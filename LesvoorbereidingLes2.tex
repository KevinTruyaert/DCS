% !TeX root = Stageportfolio.tex



\begin{landscape}
	
	\subsubsection{Les 4-5}
	\begin{tabularx}{1.56\textwidth}{|p{0.55\textwidth}|X|}\hline
		\textbf{Administratieve gegevens}\newline\newline
		Kevin Truyaert\newline\newline
		Universiteit\newline
		Handelsingenieur, 2de fase\newline
		\underline{ECTS-fiche}: De inhoud is terug te vinden op de ECTS fiche: \href{https://onderwijsaanbod.kuleuven.be/syllabi/n/D0W55AN.htm}{https://onderwijsaanbod.kuleuven.be/syllabi/n /D0W55AN.htm} \newline
		\underline{Lesonderwerp}:\newline `Oefenzitting elektromagnetisme: serie en parallel, de wet van Ohm, vermogen' & \textbf{Doelstellingen}\newline\vspace{0.5cm}
		\underline{Punt op de ECTS-fiche}
		\vspace{-0.5cm}\newline  - condensatoren: capaciteit, diëlektrische materialen, serie- en parallelschakeling, bouwvormen \newline
		- weerstanden: soortelijke weerstand, geleiders, isolatoren, serie- en parallelschakeling en elektrisch vermogen\newline
		- toepassing: elektrische veiligheid en elektrische huisinstallatie \newline
		\underline{Lesdoelen}\newline
		\vspace{-0.5cm}
		\begin{enumerate}[itemsep=0.08\baselineskip]
			\item De studenten kunnen de begrippen `spanningsverschil', `stroom', `weerstand', `energie', `vermogen' en `lading' met elkaar in verband brengen en interpreteren.
			\item De studenten kunnen de equivalente capaciteit van condensatoren in serie en parallel berekenen.
			\item De studenten kunnen het vermogen dat over weerstanden verloren gaat berekenen.
			\item De studenten kunnen het vermogen van een energiecentrale berekenen.
			\item De studenten kunnen de gegevens van een energiecentrale interpreteren en linken met de fysische variabelen.
			\item De studenten kunnen berekenen of de zekering van het circuit  bij een bepaalde belasting zal kapot gaan.
			\item  De studenten kunnen de equivalente weerstand van weerstanden in serie en parallel berekenen.
		    \item De studenten kunnen in duo over de oefening discussiëren en samen oplossingsgericht werken.
		\end{enumerate} \\\hline
	\end{tabularx}


	\begin{tabularx}{1.56\textwidth}{|p{0.55\textwidth}|X|}
		\hline
		\multirow{2}{0.55\textwidth}{\textbf{Beginsituatie}\newline De studenten hebben de theorie rond de  begrippen van in verband met condensatoren, weerstanden en vermogen twee weken voor de oefenzitting gezien in de hoorcolleges. Daar hebben ze eveneens de onderlinge relaties tussen stroom, lading, potentiaalverschil, weerstand, capaciteit en vermogen bestudeerd. \newline\newline Deze oefenzitting heeft meer raakvlakken met de interesse van de student omdat de oefeningen tastbaarder zijn. Zo gaan er oefeningen over energiecentrales en de transport van die energie tot bij je thuis en over zekeringen bij toestellen die al dan niet kapot gaan. Er zijn 28 studenten die deze sessie zouden moeten volgen, die waren er allemaal tijdens de vorige lessenreeks.\newline\newline Het lokaal kan 30 studenten plaatsen. Ik laat de banken staan in drie rijen van tien studenten. Er is een dubbel krijtbord ter beschikking en de mogelijkheid tot projectie. Wanneer er geprojecteerd wordt, hangt het projectiescherm grotendeels over beide borden.  }& \textbf{Acties}\newline  - Tijdens dit lesblok wil ik de nadruk leggen op de toepassingen en de interpretatie van de fysische begrippen rond spanningsverschil en stroom. Ik wil dat de studenten die eerst zelf formuleren om daarna terug te koppelen, afhankelijk van hun antwoord. Hierdoor laat ik ze met hun buren per twee of per drie aan de slag gaan.     \newline\newline
		- Bij het begin van de les overloop ik nog even de theorie rond de fysische begrippen en hun onderlinge relaties. Dit zet ik op één van de twee krijtborden en laat ik gedurende de hele les staan. Zo kunnen de studenten steeds makkelijk teruggrijpen naar de theorie. Ik treed eerst in gesprek met de studenten om vanuit hun antwoorden de theorie aan te reiken. \newline\newline
		- Ik werk niet met projectie, maar noteer alles op het bord, omdat het projectiescherm voor zo goed als beide borden hangt. Hierdoor houd ik een tempo aan waarop de studenten makkelijker kunnen volgen, doordat ik alles zelf ook neerschrijf.  
		
		\\ \cline{2-2}
		  & \textbf{Bronnen}\begin{itemize}
		  	\item Dudal, D., Temmerman, E., Truyaert, K., Heymans, S. (2019). Slides conceptuele natuurkunde
		  	\item Dudal, D., Temmerman, E., Truyaert, K., Heymans, S. (2019). Oefeningenbundel conceptuele natuurkunde
		  	\item Giancoli, D. C. (2008). Physics for scientists and engineers. Pearson Education International.
		  \end{itemize}\\ \hline
	\end{tabularx}


\newpage
	
	\begin{tabularx}{1.56\textwidth}{|p{1.5cm}|p{6cm}|X|p{4cm}|}
		\hline
		\textbf{Nr. lesdoel } & \textbf{Inhoud (timing)}  & \textbf{Organisatie } & \textbf{Media } \\ \hline
		1 &\underline{Herhaling theorie (15 minuten)}\newline
		De theorie rond de fysische begrippen lading, stroom, weerstand, spanningsverschil, capaciteit en vermogen worden door de studenten aangereikt. Zij interpreteren ook wat de vergelijking voorstelt en delen dit met hun medestudenten. 
		&  \underline{Onderwijsleergesprek}\newline 
		Ik start deze les met het begrip lading aan de studenten te poneren. Ik vraag hen of ze dit in verband kunnen brengen met nog andere grootheden. Vanaf de start kan ik al verschillende antwoorden krijgen. Ik plaats deze op het bord afhankelijk van hoe de studenten me dit aanreiken. Ik vraag hen niet enkel om de zaken te linken, maar ook om een uitleg waarom dit zo is. De studenten bouwen dus zelf de theorie op en interpreteren die. \newline
		Ik focus mij op het correct interpreteren van de bekomen vergelijkingen. Wanneer de student begrijpt wat er in de vergelijking staat, dan zal hij/zij deze beter begrijpen. Ik stuur de gegeven interpretatie van de student bij indien nodig, of vraag iets dieper door wanneer de student niet volledig is. 
		\newline 
		Hierna noteer ik de oefeningen op bord die gemaakt kunnen worden. Dit zijn oefeningen 58 t.e.m. 64. Ik verwacht dat deze oefeningen door iedereen gemaakt kunnen worden.  
		& Krijtbord (Bekomen bordschema wordt in bijlage toegevoegd)
		\\ \hline
	\end{tabularx}



	
	\begin{tabularx}{1.56\textwidth}{|p{1.5cm}|p{6cm}|X|p{4cm}|}
		\hline
		\textbf{Nr. lesdoel } & \textbf{Inhoud (timing)}  & \textbf{Organisatie } & \textbf{Media } \\ \hline
		1\newline 2\newline 3\newline 8&\underline{58 - 60 (1 uur)}
	    De studenten maken oefeningen 58 t.e.m. 60. Deze bespreken condensatoren en de relatie tussen de capaciteit van condensatoren, lading en spanningsverschil. De student is met deze oefeningen ook in staat om serie en parallel verbanden tussen condensatoren te bespreken. Daarnaast wordt er ook een grotere oefening gemaakt die de relaties tussen vermogen, spanningsverschil, weerstand en stroom bespreekt enerzijds en anderzijds de relaties tussen lading en stroom en tussen vermogen en energie.
	    &  \underline{Oplossingensleutel}
	    	De studenten krijgen de eindantwoorden ter beschikking en kunnen zo controleren of ze een opgave correct opgelost hebben. Ik loop ondertussen rond om vragen van studenten te beantwoorden, maar ook om vragen aan de studenten te stellen. Hierbij heb ik vooral aandacht voor de interpretaties van oefening 60, omdat deze lampen met een verschillend wattage bespreekt. In ieder huis komen er lampen met een verschillend wattage voor, waardoor het interessant is dat de studenten dit correct kunnen interpreteren. Anderzijds bespreekt deze oefening heel wat relaties tussen de fysische begrippen van deze les. 
	    
		& Oplossingenbundel\newline Oefeningenbundel + cursuspapier
		\\ \hline
	\end{tabularx}
	
	
	
	\begin{tabularx}{1.56\textwidth}{|p{1.5cm}|p{6cm}|X|p{4cm}|}
		\hline
		\textbf{Nr. lesdoel } & \textbf{Inhoud (timing)}  & \textbf{Organisatie } & \textbf{Media } \\ \hline
		&\underline{Pauze}\newline
		
		
		&    De studenten krijgen 15 minuten pauze en mogen het lokaal verlaten. Op deze manier kunnen ze het laatste uur weer met volle aandacht werken. De studenten zullen na de pauze zich focussen op het begrip vermogen en dit vooral met de relatie tussen spanningsverschil en weerstand. \newline
		Wanneer de eerste studenten het lokaal terug binnen sijpelen, sla ik een praatje met hen, waarbij ik niet over de leerstof begin. 
		& 
		\\ \hline
	\end{tabularx}
	
		\begin{tabularx}{1.56\textwidth}{|p{1.5cm}|p{6cm}|X|p{4cm}|}
		\hline
		\textbf{Nr. lesdoel } & \textbf{Inhoud (timing)}  & \textbf{Organisatie } & \textbf{Media } \\ \hline
		1\newline 3\newline 4\newline 5 \newline 8&\underline{61 (15 minuten)}\newline
		Deze oefening handelt over de generatie van energie in een energiecentrale en het transport van deze energie naar de elektriciteitscabine in de straat. Eerst berekenen de studenten het totale vermogen opgewekt in deze centrale. Daarna berekenen ze het verlies van dit vermogen vanwege  het transport naar de elektriciteitscabine in de straat. Hierbij zullen studenten de fout maken om het spanningsverschil opgewekt binnen de centrale te gebruiken in plaats van de spanningsval vanwege de transportdraad te gebruiken. 
		
		& \underline{Oplossingensleutel}
		   De studenten krijgen 15 minuten om deze oefening te maken.  Tijdens deze oefening wil ik vooral dat de studenten mij kunnen uitleggen wat hun interpretatie is bij deze oefening. Ze moeten duidelijk begrijpen dat eenzelfde begrip, hier spanningsverschil, in meerdere contexten gebruikt kan worden binnen eenzelfde oefening. Zo is er de centrale die voor een positief spanningsverschil (energiebron) zorgt en de elektriciteitskabel die een negatief spanningsverschil veroorzaakt (verbruiker). Daarom zal ik steeds een bijvraag stellen aan de studenten: `welk spanningsverschil heb je nog in de elektriciteitscabine aanwezig?'.
		& Oplossingenbundel\newline Oefeningenbundel + cursuspapier
		\\ \hline
	\end{tabularx}
	
	
	\begin{tabularx}{1.56\textwidth}{|p{1.5cm}|p{6cm}|X|p{4cm}|}
		\hline
		\textbf{Nr. lesdoel } & \textbf{Inhoud (timing)}  & \textbf{Organisatie } & \textbf{Media } \\ \hline
		1\newline 3\newline 6\newline 7\newline 8\newline &\underline{62-64,56 (45 minuten)}\newline
		De studenten richten zich bij deze oefeningen vooral op het gebruik van de relaties met betrekking tot weerstanden (of verbruiker). Ze zullen enerzijds de stroom doorheen een verbruiker moeten berekenen, om zo te constateren of de geplaatste zekering doorbrandt of niet. Tegelijkertijd leren ze dus ook de werking van een zekering, iets waar iedereen wel eens mee geconstateerd wordt. Anderzijds berekenen ze ook hoelang een verbruiker werkt gegeven een set aan batterijen. Tenslotte berekenen de studenten ook hoe lampen (verbruikers) het meest energie verbruiken, in serie of parallel.  
		& \underline{Oplossingenbundel}\newline Tijdens deze oefeningen loop ik rond om de vragen van de studenten te beantwoorden of om hen vragen te stellen bij bepaalde zaken die ik op hun blad zie. Hier wil ik hen vooral mondeling info omtrent zekeringen meegeven (waar vind je die in jullie huis?, ooit al iets mee moeten doen?, waarom zitten die daar? \ldots). Ook bij serie- en parallelschakelingen wil ik hen inzichten meegeven. De lampen zullen hier feller schijnen in parallel, maar ik wil van hen ook horen dat de batterij veel minder lang zal meegaan dan wanneer die lampen in serie staan. Dit wil ik opnieuw bereiken door vragen te stellen. Dit laatste vind ik belangrijk om mee te geven aan mijn studenten, dus breng ik deze laatste oefening nog eens klassikaal waarbij ik hen klassikaal gerichte vragen stel in verband met de relaties (lesdoel 1), om dan van een student te horen dat het verbruik bij batterijen ook afhankelijk is van de schakeling. Wanneer er nog tijd over is, kunnen de studenten oefening 56 nog maken, aangezien die vorige les niet aan bod gekomen is.
		& Oplossingenbundel\newline Oefeningenbundel + cursuspapier\newline Krijtbord
		\\ \hline
	\end{tabularx}
	
	


\begin{tabularx}{1.56\textwidth}{|p{1.5cm}|p{6cm}|X|p{4cm}|}
	\hline
	\textbf{Nr. lesdoel } & \textbf{Inhoud (timing)}  & \textbf{Organisatie } & \textbf{Media } \\ \hline
	&\underline{Afsluiten (5 minuten)}\newline 
	&  \underline{Afsluiten (5 minuten)}\newline
	Ik herhaal nog even kort wat er van de studenten verwacht werd tijdens deze les en wat ze bijgeleerd hebben. Ik zeg ook wat het onderwerp van volgende les is en vraag aan de studenten om de theorie nog even te herhalen.
	& 
	\\ \hline
\end{tabularx}
	
	
	
	
	
\end{landscape}

\subsection*{Bijlage 1.1: voorbereiding theorie}

\subsection*{Bijlage 2.2: bordschema theorie}


\includepdf[scale = 0.8,pages = 21,pagecommand=\subsection*{Bijlage 2.3: opgeloste oefeningen}]{Observaties_OpgelosteOef}
\includepdf[scale = 0.8,pages =22-23,pagecommand=]{Observaties_OpgelosteOef}

\subsection*{Bijlage 2.4: oplossingssleutels}
De oplossingssleutels zijn hieronder bijgevoegd. Deze zijn telkens zo opgesteld dat de studenten niet de oplossing uitgewerkt krijgen, maar dat er enkele gerichte vragen of hints zijn die hen op de weg kunnen helpen wanneer ze vast zitten. Wanneer ze de oefening correct hebben, kunnen ze stilstaan bij die vragen en controleren of ze de oefening ook begrijpen.\newline
De oplossingssleutels zijn hier samen gevoegd. In realiteit werden meerdere sleutels per blad afgedrukt en daarna uitgeknipt.

\subsection*{58}
\underline{Methode}
\begin{enumerate}
	\item Welke relaties met betrekking tot de capaciteit van een condensator ken je? 
	\item Met twee van deze relaties kan je het antwoord bekomen!
\end{enumerate}

\underline{Extra vragen}
\begin{enumerate}
	\item Wat betekent het dat de lucht `zuiver en droog' is? Welke factor zou wijzigen als deze situatie anders is?
	\item Wat gebeurt er wanneer er 1~$\mu$C meer op de wolk geplaatst zou worden?
	\item Hoe verhoudt de maximale lading zich tot de dimensies van een condensator?
\end{enumerate}





\subsection*{59}
\underline{Methode}
\begin{enumerate}
	\item Een equivalente condensator zoeken voor dit systeem moet in meerdere stappen gebeuren. Wat los je eerst op?
\end{enumerate}

\underline{Extra vragen}
\begin{enumerate}
	\item Stel dat iedere condensator een weerstand zou zijn. Hoe groot is de equivalente weerstand? Verklaar het verschil.
\end{enumerate}




\subsection*{60}
\underline{Methode}
\begin{enumerate}
	\item Je mag er van uit gaan dat de bron aan een constante van 120~V blijft leveren. 
	\item Wat betekent het om zwak/fel te branden?
	\item Wat brengt een verschil in ladingen teweeg? En wat gebeurt er met lading hierin?
	\item Wat stelt de eenheid kWh voor? Wat is de SI-eenheid hiervan?
\end{enumerate}


\subsection*{61}
\underline{Methode}
\begin{enumerate}
	\item Wat zijn parameters van de centrale?
	\item Welke parameters horen bij de kabel?
	\item Een bron zorgt voor een positief spanningsverschil, een verbruiker voor een negatief.
\end{enumerate}

\underline{Extra vragen}
\begin{enumerate}
	\item Welke spanning is op locatie nog beschikbaar?
\end{enumerate}



\subsection*{62}
\underline{Methode}
\begin{enumerate}
	\item Waarvoor zorgt een zekering bij een schakeling?
	\item De zekering kan een stroom van 4~A aan. Wat is de stroom in de schakeling?
\end{enumerate}

\underline{Extra vragen}
\begin{enumerate}
	\item Waar kom je zekeringen zoal tegen?
\end{enumerate}




\subsection*{63}
\underline{Methode}
\begin{enumerate}
	\item Wat gebeurt er met het spanningsverschil wanneer je batterijen in serie plaatst?
	\item Wat is opnieuw de link tussen stroom, lading en tijd? 
\end{enumerate}





\subsection*{64}
\underline{Methode}
\begin{enumerate}
	\item Bereken eerst de equivalente weerstand in beide situaties.
	\item Wat is de betekenis van emk?
	\item Wanneer geeft een lamp opnieuw het meeste licht? (Oefening 60)
\end{enumerate}












