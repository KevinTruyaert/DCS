% !TeX root = Stageportfolio.tex

\section{Persoonlijk ontwikkelingsplan}
\begin{tabularx}{\textwidth}{|p{0.15\textwidth}|p{0.795\textwidth}|}
	\hline
	\textbf{Lesdoel 1} & 
	\underline{FG 1: de leraar als begeleider van leer- en}\newline \underline{ontwikkelingsprocessen}\newline
	
	\PinkHighlight{1.8 De leraar kan observatie en evaluatie voorbereiden en uitvoeren met het oog op bijsturing en remediëring als onderdeel van het leerproces van een lerende(n) en kan die observatie-en evaluatiegegevens gebruiken om zijn eigen didactische handelen in vraag te stellen en bij te sturen waar nodig.}{12.8cm}\\ \hline
	Actie 1 & Tijdens het lesgeven wil ik problemen i.v.m. de leerstof bij de leerlingen opsporen. Dit kan ik doen door gerichte vragen te stellen, aandachtig te luisteren en te kijken naar de leerlingen terwijl ze aan het werk zijn, hun handelingen te interpreteren \ldots Vanuit dit alles wil ik bij zoveel mogelijk leerlingen een beeld schetsen in verband met hun begrip bij de behandelde leerinhouden. Ik wil me tijdens mijn stage  vooral richten op het ontwikkelen van mijn verschillende `voelsprieten' om dit te bij alle leerlingen op te sporen.	
%	Tijdens het lesgeven wil ik veel in interactie treden. Dit zou ik met zoveel mogelijk leerlingen willen doen en niet steeds dezelfde leerlingen aan bod laten komen. Het lijkt mij uitdagend om dit in realiteit om te zetten. Door hen gerichte vragen te stellen, kan ik kijken waar er mogelijke problemen zijn met de leerstof en van daaruit werken om de lesdoelen begrijpelijk te maken voor alle leerlingen. 
	\\ \hline
	Actie 2 & Wanneer ik problemen bij leerlingen ontdek, wil ik mij richten op het bijsturen van die leerlingen. Hoe kan ik hun problemen tijdens de les aanpakken om ze de leerinhouden te laten begrijpen? Tegelijkertijd wil ik mij focussen om dezelfde soort problemen bij leerlingen tijdens volgende lessen te vermijden door hen op een andere manier te benaderen.
	%Na het verbeteren van een taak, een toets of extra oefeningen wil ik die met de leerling(en) overlopen door de meest voorkomende fouten te bespreken. Zo kan ik hen bijsturen en kan ik de belangrijkste punten aanhalen waar er problemen waren. Tegelijkertijd kom ik zo te weten waar ik te weinig nadruk gelegd heb tijdens de les. Hier kan ik nu mee aan de slag om mijn toekomstige lessen aan te passen en om te verhinderen dat hetzelfde soort fouten bij soortgelijke zaken minder gemaakt worden. 
	\\ \hline
\end{tabularx}


\vspace{0.5cm}
\begin{tabularx}{\textwidth}{|p{0.15\textwidth}|p{0.795\textwidth}|}
	\hline
	\textbf{Lesdoel 2} & 
	\underline{FG 1: de leraar als begeleider van leer- en}\newline \underline{ontwikkelingsprocessen}\newline \YellowHighlight{1.2 De leraar kan zijn didactische handelen afstemmen op enerzijds de doelstellingen en anderzijds de leefwereld, de motivatie, de beginsituatie en de behoeften van de lerende(n) rekening houdend met de diversiteit van de groep.}{12.8cm} \\ \hline
	Actie 1 & Ik wil als leraar in staat zijn om de theorie interessant over te kunnen brengen. Dit wil ik doen door actuele zaken als voorbeeld van die theorie te gebruiken.  Door actuele thema's en alledaagse voorwerpen te linken met fysische verschijnselen, hoop ik dat de leerlingen de wereld rond hen beter begrijpen. \\ \hline
	Actie 2 &  Naast het binnenbrengen van de actualiteit tijdens de lessen fysica, wil ik de leerlingen ook op andere manieren gaan stimuleren en motiveren. Dit wil ik doen door de interesse van de leerlingen bij de lessen proberen te betrekken. Dit kan ik enkel doen als ik oprecht interesse toon in de leefwereld van de leerlingen en die leefwereld in de lessen probeer binnen te terkken.\\ \hline
\end{tabularx}


\vspace{0.5cm}
\begin{tabularx}{\textwidth}{|p{0.15\textwidth}|p{0.795\textwidth}|}
	\hline
	\textbf{Lesdoel 3} & 
	\underline{FG 1: de leraar als begeleider van leer- en}\newline \underline{ontwikkelingsprocessen}\newline
	\GreenHighlight{1.5 De leraar kan aangepaste werkvormen en groeperingsvormen bepalen en gebruiken.}{12.8cm}
	\\ \hline
	Actie 1 & Ik verzorg reeds drie jaar oefenzittingen aan de universiteit. Dit jaar wil ik iets nieuws proberen en de studenten actiever krijgen tijdens de oefenzittingen. Ik wil hen in groep aan de oefeningen laten werken, waardoor ze met elkaar in interactie kunnen treden om de oefeningen samen tot een goed eind te kunnen brengen.   \\ \hline
	Actie 2 & Bij de lessen die ik in het middelbaar zal verzorgen, wil ik terugkoppelen naar mijn stagelessen die ik bij DCO deed. Hier gaf ik telkens de introductieles van een nieuw stuk theorie. Die gaf ik relatief `klassiek', waarbij ik als leerkracht veel aan bod kwam. Ik wil nu proberen om de leerlingen zal actiever aan de slag te zetten bij de start van een nieuw stuk. Ik zie dit nu ook meer zitten, omdat ik meer dan één les(blok) per klas zal brengen. Dit zal als gevolg hebben dat ik een groter plan kan uitwerken en zo proberen om mijn eigen lesgeven te innoveren.    \\ \hline
\end{tabularx}





