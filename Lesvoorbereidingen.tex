% !TeX root = Stageportfolio.tex



\begin{landscape}
	\section{Lesvoorbereidingen en bijhorende media}
	\begin{tabularx}{1.56\textwidth}{|p{0.35\textwidth}|X|}\hline
		\textbf{Administratieve gegevens}\newline\newline
		Kevin Truyaert\newline\newline
		Universiteit\newline
		Handelsingenieur, 2de fase\newline
		Onderwijskoepel\newline
		Leerplannummer\newline
		Lesonderwerp & \textbf{Doelstellingen}\newline\newline
		\underline{Leerplandoelen}\newline\newline
		\underline{Lesdoelen}\newline\newline \\\hline
		\multirow{2}{0.35\textwidth}{\textbf{Beginsituatie}} & \textbf{Acties} \\ \cline{2-2}
		 & \textbf{Bronnen}\\\hline
		
	\end{tabularx}
	
	
	
\begin{table}[h]
	\begin{tabularx}{1.56\textwidth}{|p{1.5cm}|p{6cm}|X|p{4cm}|}
		\hline
		\textbf{Nr. lesdoel } & \textbf{Inhoud (timing)}  & \textbf{Organisatie } & \textbf{Media } \\ \hline
		&\underline{Inhoudelijke titel (timing)}
	    \textcolor{gray}{(Naast een inhoudelijke titel en de timing, noteer je kort en samenvattend de kerninhoud van de lesfase; uitgebreide informatie/oefeningen/… neem je op in de uitgewerkte media [verwijzen!])}
	    &  \textcolor{gray}{(Naast de benaming van de specifieke werkvorm [bv. placemat-oefening/basis-expertengroep/… en dus níet groepswerk], noteer je kernachtig het organisatorisch verloop van de lesfase. Noteer eveneens belangrijke vragen die je wil stellen.) }
		& 
		\\ \hline
	\end{tabularx}
\end{table}		
	
	
	
	
	
	
	
	
\end{landscape}