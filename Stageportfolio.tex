\documentclass[a4paper,12pt,twoside]{article}%twoside
\usepackage[utf8]{inputenc}
\usepackage[dutch]{babel}
\usepackage{fancyhdr, amsmath, color, graphicx, enumitem, tabularx, hyperref, longtable, multirow, placeins, apacite, subcaption,marvosym,multicol}
\usepackage[framemethod=tikz]{mdframed}
 
  \usepackage[margin=2.5cm,headheight=68pt]{geometry}
 %\usepackage[total={16cm, 22cm}]{geometry}
 
\pagestyle{fancy}
\fancyhf{}
\fancyhead[LE,RO]{Specifieke Lerarenopleiding voor CVO-studenten}%'E': even page, 'O': odd page
\fancyhead[RE,LO]{Didactische Competentie Stage}
\fancyfoot[RE,CO]{}
\fancyfoot[LE,RO]{KU Leuven campus Kortrijk Kulak   \thepage}
%\renewcommand{\labelitemi}{$\circ$}
 
\definecolor{CVO}{RGB}{232, 0, 97}
\setlength\parindent{0pt}
\title{Stageportfolio}
\author{Kevin Truyaert}
\date{}
 
 
 % %FOOTER IN MDFRAMED
 \usepackage{footnote} 
 \newenvironment{mdframedwithfoot}
 {   
     \savenotes
     \begin{mdframed}
     \stepcounter{footnote}
     \renewcommand{\thefootnote}{\arabic{footnote}}
     }
 {
     \end{mdframed}
     \spewnotes
 }
 
 
 %FOOTER IN PARBOX
 \makeatletter
 \newcommand{\global@insert}[2]% #1=box number, #2=vertical list
 {\bgroup
   \setbox\@tempboxa=\box#1
   \global\setbox#1=\vbox{\unvbox\@tempboxa #2}
 \egroup}
 
 \long\def\@footnotetext#1{\global@insert\footins{%
  \reset@font\footnotesize
  \interlinepenalty\interfootnotelinepenalty
  \splittopskip\footnotesep
  \splitmaxdepth \dp\strutbox \floatingpenalty \@MM
  \hsize\columnwidth \@parboxrestore
  \protected@edef\@currentlabel{%
  \csname p@footnote\endcsname\@thefnmark
  }%
  \color@begingroup
  \@makefntext{%
  \rule\z@\footnotesep\ignorespaces#1\@finalstrut\strutbox}%
  \color@endgroup}}%
 \makeatother
 %%%%%%%%%%%%%%%%%%%%%%%%%
 
 %Strikeout and highlight text
  \usepackage{soul}
  \usepackage{tikz} % only to get \foreach
  
  %\definecolor{yellow}{RGB}{255,255,0}
  \sethlcolor{yellow}

  \newcommand*{\YellowHighlight}[1]{{\hl{~#1~}}}
  % % % % % %
  
  \usepackage{tabularx,pdflscape,pdfpages}
  
  \newcolumntype{C}[1]{>{\centering\let\newline\\\arraybackslash\hspace{0pt}}m{#1}}
 
 \begin{document}
\maketitle


\section*{Identificatiegegevens}
\begin{center}
	\begin{tabular}{ll}
	\hline
	Naam: & Kevin Truyaert\\ \hline
	Adres: & Bolle-Akkerweg 4\\
		& 8800 Roeselare\\\hline
	Telefoon: & 0032495/928460\\\hline
	Mail: & kevin.truyaert@student.kuleuven.be\\\hline
	Naam stagebegeleider: & Annelies Declerck\\ \hline
\end{tabular}
\end{center}

\newpage
\tableofcontents
\newpage

\section{Observatie- en stageplanning}
% !TeX root = Stageportfolio.tex

\begin{landscape}
	
	\begin{tabularx}{1.56\textwidth}{|X|}
		\hline
		Naam stagair:  Kevin Truyaert  \\
		Tel.: 0495/928460 \hspace{3cm} e-mail: kevin.truyaert@student.kuleuven.be  \\
		Naam en adres opleidingsinstituut:  KU Leuven Campus Kulak Kortrijk, Etienne-Sabbelaan 53, 8800 Kortrijk  \\
		Naam directie: \\
		Naam stagecoördinator:  David Dudal \\
		\hline
	\end{tabularx}
	\vspace*{-0.4cm}
\section{Observatie- en stageplanning}
\vspace*{-0.3cm}\subsection{Observatieplanning}
\subsubsection{Kulak (LIO)}%
\vspace*{-0.5cm}
%\parskip 
%\vspace{\parskip}
%\begin{minipage}[b]{\textwidth}
\begin{center}
		\includegraphics[scale = 0.9,trim={7.8cm 2cm 7.3cm 2cm} ,clip,angle=-90]{OnePageObservatieKulak}
\end{center}
%\end{minipage}

\subsubsection{VISO}%\\
\vspace*{-0.5cm}
\begin{center}
	\includegraphics[scale = 0.9,trim={2.8cm 3cm 14.2cm 3cm} ,clip,angle=-90]{ObservatielesVISO}
\end{center}
%\begin{tabularx}{1.56\textwidth}{|C{0.05\textwidth}|C{0.15\textwidth}|C{0.1\textwidth}|C{0.2\textwidth}|C{0.09\textwidth}|C{0.21\textwidth}|C{0.1\textwidth}|C{0.25\textwidth}|X|}
%	\hline
%	\textbf{Nr.} & \textbf{Datum} & \textbf{Tijdstip} & \textbf{\begin{tabular}[C]{@{}l@{}}Onderwijsvorm\\ graad en lj\\ studierichting\end{tabular}} & \textbf{Lokaal} &\textbf{\begin{tabular}[C]{@{}l@{}} Leervak en\\ lesonderwerp \end{tabular}} & \textbf{\begin{tabular}[C]{@{}l@{}}AV/TV\\PV/KV\end{tabular}} & \textbf{Mentor/School} & \textbf{\begin{tabular}[C]{@{}l@{}} Handtekening\\mentor\end{tabular}}\\ \hline
%	1 & 12/02/2020 & 8:25-9:15 &\begin{tabular}[C]{@{}l@{}}tso\\3e graad 1ste jaar\\ Technieck-\\Wetenschappen\\\end{tabular} & A013 & \begin{tabular}[C]{@{}l@{}}Toegepaste\\ fysica:\\ Herhaling ERB\\en Inleiding\\ bewegins-\\vergelijking \end{tabular} & TV & \begin{tabular}[C]{@{}l@{}}Mevr. S. Schramme\\ VISO \end{tabular} & \\ \hline
%\end{tabularx}
	
\newpage
\subsection{Actieve stage}
		\subsubsection{Kulak (LIO)}
	%	
	%	\begin{minipage}[t][10cm][t]{0.5\textwidth}
			\begin{figure*}[h]
				\begin{tikzpicture}
				\node[anchor=south west] 
				at (0,0) %left bottom corner of the page
				{\includegraphics[scale = 0.85,trim={4cm 2cm 6cm 1.8cm} ,clip,angle=-90 ]{OnePageLesgevenKulak}};
				\node[fill=white] at(6.58,4.25){4-6};
				\node[fill=white] at(6.58,2.55){7-9};
				\node[fill=white] at(6.58,0.85){10-12};				
				\end{tikzpicture}
			\end{figure*}
		%\end{minipage}
		\vfill\newpage
%	\begin{tabularx}{1.56\textwidth}{|C{0.15\textwidth}|C{0.14\textwidth}|C{0.14\textwidth}|C{0.1\textwidth}|C{0.1\textwidth}|C{0.05\textwidth}|C{0.35\textwidth}|X|}
%		\hline
%		\textbf{Datum} & \textbf{Vestiging} & \textbf{\begin{tabular}[C]{@{}l@{}}Aantal\\ stage-uren\end{tabular}} & \textbf{\begin{tabular}[C]{@{}l@{}}Uur \end{tabular}}    & \textbf{Lokaal}& \textbf{\begin{tabular}[C]{@{}l@{}}AV\\TV\\PV\\KV\end{tabular}}& \textbf{\begin{tabular}[C]{@{}l@{}}Onderwijsvorm\\ graad en lj\\ Vak en lesonderwerp\end{tabular}}  &  \textbf{\begin{tabular}[C]{@{}l@{}}Naam vakmentor\\ + handtekening\end{tabular} } \\ \hline
%		27/11/2019 & Kulak & 1-3 & 10:30-13:00 & A352 & AV & Universiteit\newline 2e jaar Handelsingenieur\newline Conceptuele natuurkunde\newline werkzitting elektromagnetisme & \\ 
%\hline
%		4/12/2019 & Kulak & 4-5 & 10:30-13:00 & A352 & AV & Universiteit\newline 2e jaar Handelsingenieur\newline Conceptuele natuurkunde\newline werkzitting elektromagnetisme & \\ \hline
%		11/12/2019 & Kulak & 6-8 & 10:30-13:00 & A352 & AV & Universiteit\newline 2e jaar Handelsingenieur\newline Conceptuele natuurkunde\newline werkzitting elektromagnetisme & \\ \hline
%		19/12/2019 & Kulak & 9-10 & 10:00-12:30 & A352 & AV & Universiteit\newline 2e jaar Handelsingenieur\newline Conceptuele natuurkunde\newline werkzitting elektromagnetisme & \\ \hline
%	%	 &  &  &  &  &  &  & \\ \hline
%	\end{tabularx}
%	
\subsubsection{VISO Roeselare}

\includegraphics[scale = 0.95,trim={2.7cm 3cm 4cm 3cm} ,clip,angle=-90 ]{P1PlanningVISO}\newpage
\includegraphics[scale = 0.95,trim={2cm 3cm 8cm 3cm} ,clip,angle=-90 ]{P2PlanningVISO}
%\begin{tabularx}{1.56\textwidth}{|C{0.15\textwidth}|C{0.14\textwidth}|C{0.14\textwidth}|C{0.1\textwidth}|C{0.1\textwidth}|C{0.05\textwidth}|C{0.35\textwidth}|X|}
%	\hline
%	\textbf{Datum} & \textbf{Vestiging} & \textbf{\begin{tabular}[C]{@{}l@{}}Aantal\\ stage-uren\end{tabular}} & \textbf{\begin{tabular}[C]{@{}l@{}}Uur \end{tabular}}    & \textbf{Lokaal}& \textbf{\begin{tabular}[C]{@{}l@{}}AV\\TV\\PV\\KV\end{tabular}}& \textbf{\begin{tabular}[C]{@{}l@{}}Onderwijsvorm\\ graad en lj\\ Vak en lesonderwerp\end{tabular}}  &  \textbf{\begin{tabular}[C]{@{}l@{}}Naam vakmentor\\ + handtekening\end{tabular} } \\ \hline
%	20/02/2020 & VISO Roeselare & 11-12 & 8:25-10:05 & PA13\newline PB21 & AV & tso\newline 3e graad 1ste jaar Techniek-Wetenschappen\newline Toegepaste fysica\newline Labo M4: de stroombalans & \\ \hline
%	4/03/2020 & VISO Roeselare & 13 & 8:25-9:15 & PB25 & AV & tso\newline 3e graad 1ste jaar Techniek-Wetenschappen\newline Toegepaste fysica\newline Afwerken labo M4  \&\newline Magnetische flux& \\ \hline
%	5/03/2020 & VISO Roeselare & 14-15 & 8:25-10:05 & PA13 & AV & tso\newline 3e graad 1ste jaar Techniek-Wetenschappen\newline Toegepaste fysica\newline Bespreking labo M4 \& Magnetische fluxverandering \& Inductiespanning: wet van Faraday  & \\ \hline
%	11/03/2020 & VISO Roeselare & 16 & 8:25-9:15 & PA13 & AV & tso\newline 3e graad 1ste jaar Techniek-Wetenschappen\newline Toegepaste fysica\newline  Wet van Lenz \& algemene inductiewet: Faraday-Lenz + oefeningen  & \\ \hline
%\end{tabularx}\newpage
%\begin{tabularx}{1.56\textwidth}{|C{0.15\textwidth}|C{0.14\textwidth}|C{0.14\textwidth}|C{0.1\textwidth}|C{0.1\textwidth}|C{0.05\textwidth}|C{0.35\textwidth}|X|}
%	\hline
%	\textbf{Datum} & \textbf{Vestiging} & \textbf{\begin{tabular}[C]{@{}l@{}}Aantal\\ stage-uren\end{tabular}} & \textbf{\begin{tabular}[C]{@{}l@{}}Uur \end{tabular}}    & \textbf{Lokaal}& \textbf{\begin{tabular}[C]{@{}l@{}}AV\\TV\\PV\\KV\end{tabular}}& \textbf{\begin{tabular}[C]{@{}l@{}}Onderwijsvorm\\ graad en lj\\ Vak en lesonderwerp\end{tabular}}  &  \textbf{\begin{tabular}[C]{@{}l@{}}Naam vakmentor\\ + handtekening\end{tabular} } \\ \hline
%	12/03/2020 & VISO Roeselare & 17-18 & 8:25-10:05 & PA13 & AV & tso\newline 3e graad 1ste jaar Techniek-Wetenschappen\newline Toegepaste fysica\newline Oefeningen algemene inductiewet \& toepassingen inductie & \\ \hline
%	18/03/2020 & VISO Roeselare & 19 & 8:25-9:15 & PA13 & AV & tso\newline 3e graad 1ste jaar Techniek-Wetenschappen\newline Toegepaste fysica\newline Toepassingen inductie & \\ \hline
%	19/03/2020 & VISO Roeselare & 20-21 & 8:25-10:05 & PA13 & AV & tso\newline 3e graad 1ste jaar Techniek-Wetenschappen\newline Toegepaste fysica\newline Labo M5: de transformator  & \\ 
%\hline	
%%	 &  &  &  &  &  &  & \\ \hline
%\end{tabularx}
	


		
\end{landscape}		
		


\section{Persoonlijk ontwikkelingsplan}

\begin{tabularx}{\textwidth}{|p{0.15\textwidth}|p{0.795\textwidth}|}
	\hline
	\textbf{Lesdoel 1} & 
	\underline{FG 1: de leraar als begeleider van leer- en}\newline \underline{ontwikkelingsprocessen}\newline
	
	1.8 De leraar kan observatie en evaluatie voorbereiden en uitvoeren met het oog op bijsturing en remediëring als onderdeel van het leerproces van een lerende(n) en kan die observatie-en evaluatiegegevens gebruiken om zijn eigen didactische handelen in vraag te stellen en bij te sturen waar nodig.\\ \hline
	Actie 1 & Tijdens het lesgeven wil ik veel in interactie treden. Dit zou ik met zoveel mogelijk leerlingen willen doen en niet steeds dezelfde leerlingen aan bod laten komen. Door hen gerichte vragen te stellen, kan ik kijken waar er mogelijke problemen zijn met de leerstof en van daaruit werken om zoveel mogelijk begrijpelijk te maken voor alle leerlingen. \\ \hline
	Actie 2 & Na het verbeteren van een toets, wil ik die met de leerlingen overlopen door de meest voorkomende fouten te bespreken. Zo kan ik hen bijsturen en kan ik de belangrijkste punten aanhalen waar er problemen waren. Tegelijkertijd kom ik zo te weten waar ik te weinig nadruk gelegd heb tijdens de les. Hier kan ik nu mee aan de slag om mijn toekomstige lessen aan te passen en om te verhinderen dat hetzelfde soort fouten bij soortgelijke zaken minder gemaakt worden. \\ \hline
\end{tabularx}

\vspace{0.5cm}
\begin{tabularx}{\textwidth}{|p{0.15\textwidth}|p{0.795\textwidth}|}
\hline
\textbf{Lesdoel 2} & \underline{FG5:  de leraar als innovator - de leraar als onderzoeker}\newline
5.1 De leraar kan de kwaliteit van zijn onderwijs verder ontwikkelen. De leraar kan zijn eigen onderwijspraktijk en zijn eigen functioneren in vraag stellen en bijsturen (verbeteren) door te innoveren om zijn eigen praktijk te verbeteren.\\ \hline
Actie 1 & Ik verzorg reeds drie jaar oefenzittingen aan de universiteit. Dit jaar wil ik iets nieuws proberen en de studenten actiever de oefeningen laten maken. Ik wil hen in groep aan de oefeningen laten werken, waardoor ze met elkaar in interactie kunnen treden om de oefeningen samen tot een goed eind te kunnen brengen. Op die manier wil ik tijdens mijn oefenzittingen voor innovatie bij lessen in het hoger onderwijs zorgen. \\ \hline
Actie 2 & Bij de lessen die ik in het middelbaar zal verzorgen, wil ik terugkoppelen naar mijn stagelessen die ik bij DCO deed. Hier gaf ik telkens de introductieles van een nieuw stuk theorie. Die gaf ik relatief `klassiek', waarbij ik als leerkracht veel aan bod kwam. Ik wil nu proberen om de leerlingen zal actiever aan de slag te zetten bij de start van een nieuw stuk. Ik zie dit nu ook meer zitten, omdat ik meer dan één les(blok) per klas zal brengen. Dit zal als gevolg hebben dat ik een groter plan kan uitwerken en zo proberen om mijn eigen lesgeven te innoveren.    \\ \hline
\end{tabularx}

\vspace{0.5cm}
\begin{tabularx}{\textwidth}{|p{0.15\textwidth}|p{0.795\textwidth}|}
\hline
\textbf{Lesdoel 3} & \\ \hline
Actie 1 & \\ \hline
Actie 2 & \\ \hline
\end{tabularx}


\section{Bespreking lesobservaties}

\section{Lesvoorbereidingen en bijhorende media}

\section{Bespreking meso-activiteiten}

\section{Evaluatiedocumenten vakmentor}

\section{Evaluatie document klasbezoek stagebegeleider}

\section{Eindreflectie}

\section{Voorbereiding eindassessment}


















 \end{document}
