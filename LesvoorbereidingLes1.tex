% !TeX root = Stageportfolio.tex



\begin{landscape}
	\section{Lesvoorbereidingen en bijhorende media}
	
	\subsection{Les 1-3}
	\begin{tabularx}{1.56\textwidth}{|p{0.55\textwidth}|X|}\hline
		\textbf{Administratieve gegevens}\newline\newline
		Kevin Truyaert\newline\newline
		Universiteit\newline
		Handelsingenieur, 2de fase\newline
		\underline{ECTS-fiche}: De inhoud is terug te vinden op de ECTS fiche: \href{https://onderwijsaanbod.kuleuven.be/syllabi/n/D0W55AN.htm}{https://onderwijsaanbod.kuleuven.be/syllabi/n /D0W55AN.htm} \newline
		\underline{Lesonderwerp}: `Oefenzitting elektromagnetisme: wat zijn de relaties tussen de elektrische kracht, de  elektrische potentiaal, de elektrische flux en de elektrische capaciteit' & \textbf{Doelstellingen}\newline\vspace{0.5cm}
		\underline{Punt op de ECTS-fiche}\newline - Elektriciteit: elektrische lading, elektrisch veld (wetten van Coulomb en Gauss), elektrische flux, elektrische potentiaal, energie in een elektrisch veld \newline
		\underline{Lesdoelen}\newline
		\vspace{-0.5cm}
		\begin{enumerate}
			\item De studenten kunnen via de wet van Coulomb de elektrostatische kracht tussen ladingen berekenen.
			\item De studenten kunnen de relatie tussen de elektrostatische kracht, het elektrisch veld en een lading toepassen in een probleem.
			\item De studenten kunnen de elektrostatische kracht binnen de tweede wet van Newton herkennen.
			\item De studenten kunnen een Gaussoppervlak in een situatie opstellen.
			\item De studenten zijn in staat om de elektrische flux te bepalen met gebruik van een Gaussoppervlak.
			\item De studenten kunnen het elektrisch veld en de elektrische flux van een boloppervlak in functie van de afstand afleiden.
			\item De studenten kunnen het elektrisch veld en de elektrische flux van een opgevulde, geleidende bol in functie van de afstand afleiden.
		    \item De studenten kunnen in groep over de oefening discussiëren en samen oplossingsgericht werken.
		\end{enumerate} \\\hline
	\end{tabularx}


	\begin{tabularx}{1.56\textwidth}{|p{0.55\textwidth}|X|}
		\hline
		\multirow{2}{0.55\textwidth}{\textbf{Beginsituatie}\newline De studenten hebben de theorie rond de  begrippen van `Elektrisch veld', `Elektrische potentiaal', `Elektrische flux' en de wet van Coulomb in de week van 12-15 november gezien, twee weken voor de oefenzitting. Hierdoor zullen ze al tijd gehad hebben om de theorie te bekijken, wat aangemoedigd wordt door het maken van een voorbereidende opdracht die ik de week voor de oefenzitting op Toledo plaats.\newline\newline De minderheid van de studenten heeft  interesse bij mechanica, het eerste deel van de cursus, getoond. Het gedeelte over elektromagnetisme ervaren ze meestal interessanter. Er zijn 28 studenten die deze sessie volgen, maar gemiddeld gezien zijn er 25 studenten aanwezig geweest bij de voorbije lessen.\newline\newline Het lokaal kan 30 studenten plaatsen. Ik splits de groep in zeven tafels van vier personen. Er is een dubbel krijtbord ter beschikking en de mogelijkheid tot projectie. Wanneer er geprojecteerd wordt, hangt het projectiescherm grotendeels over beide borden.  }& \textbf{Acties}\newline  - Om de studenten te stimuleren om zelf aan de slag te gaan, wil ik hen in \GreenHighlight{groepjes van vier studenten}{5cm} aan de slag zetten. Hierdoor kan ik gerichtere feedback geven, aangezien de studenten onderling elkaar kunnen aanzetten tot het vinden van oplossingen. \PinkHighlight{Naast de ondersteunende rol, kan ik ook interacties tussen de}{12cm}\PinkHighlight{studenten onderling volgen}{5cm} en inspringen waar nodig: ofwel bij het maken van een fout, of wanneer ik hun uiteenzetting zeer goed vind en er nog dieper op in wil gaan. Dit wil ik steeds vanuit het onderwijsleergesprek proberen te realiseren.  \newline\newline
		- Bij het begin van de les overloop ik nog even de theorie rond de elektrische grootheden en hun onderlinge relaties. Dit zet ik op één van de twee krijtborden en laat ik gedurende de hele les staan. Zo kunnen de studenten steeds makkelijk teruggrijpen naar de theorie. \newline\newline
		- Ik werk niet met projectie, maar noteer alles op het bord, omdat het projectiescherm voor zo goed als beide borden hangt. Hierdoor houd ik een tempo aan waarop de studenten makkelijker kunnen volgen, doordat ik alles zelf ook neerschrijf.  
		
		\\ \cline{2-2}
		  & \textbf{Bronnen}\begin{itemize}
		  	\item Dudal, D., Temmerman, E., Truyaert, K., Heymans, S. (2019). Slides conceptuele natuurkunde
		  	\item Dudal, D., Temmerman, E., Truyaert, K., Heymans, S. (2019). Oefeningenbundel conceptuele natuurkunde
		  	\item Giancoli, D. C. (2008). Physics for scientists and engineers. Pearson Education International.
		  \end{itemize}\\ \hline
	\end{tabularx}


\newpage
	
	\begin{tabularx}{1.56\textwidth}{|p{1.5cm}|p{6cm}|X|p{4cm}|}
		\hline
		\textbf{Nr. lesdoel } & \textbf{Inhoud (timing)}  & \textbf{Organisatie } & \textbf{Media } \\ \hline
		&\underline{Herhaling theorie (15 minuten)}\newline
		De algemene student heeft op dit moment weinig voeling met de te bespreken leerstof, want het is de eerste oefenzitting over dit onderwerp. Dit heb ik zowel de voorbije jaren tijdens mijn oefenzittingen gemerkt als bij de geobserveerde theorielessen. Daarom breng ik de theorie waarop de studenten oefeningen zullen maken nog eens zelf aan bord. Deze behandelt vijf topics: lading, elektrisch veld, elektrische kracht, flux en de elektrische wet van Gauss. Vooral deze laatste vormt een struikelblok voor de studenten. Het is mijn bedoeling om die op verschillende manieren nog eens uitgelegd te hebben.
		&  \underline{Doceren}\newline 
		Ik bouw de te gebruiken theorie op door te starten vanuit de eigenschappen van een lading, dat die een elektrisch veld genereren en dat een elektrisch veld op een andere lading inwerkt door middel van de elektrische kracht. Daarna herhaal ik nog kort eens wat elektrische flux is, om dat tot het grootste probleempunt te komen: de elektrische wet van Gauss.\newline
		Ik wil vooral heel hard benadrukken wat deze wet zegt, door de aparte onderdelen uit te leggen en conceptueel voor te stellen. Ik doe dit vanuit verschillende insteken om zoveel mogelijk studenten mee te hebben. 
		\newline 
		Hierna noteer ik de oefeningen op bord die gemaakt kunnen worden. Dit zijn oefeningen 51 t.e.m. 57. Ik verwacht dat deze oefeningen door de betere studenten allemaal gemaakt kunnen worden. Ik verwacht dat de meesten zullen vast komen te zitten bij oefening 54 en 55. Deze gaan namelijk over de elektrische wet van Gauss. Oefeningen 56 en 57 kunnen tijdens de volgende les ook nog aan bod komen. 
		& Krijtbord (Bordschema, zie bijlage)
		\\ \hline
	\end{tabularx}

\begin{tabularx}{1.56\textwidth}{|p{1.5cm}|p{6cm}|X|p{4cm}|}
	\hline
	\textbf{Nr. lesdoel } & \textbf{Inhoud (timing)}  & \textbf{Organisatie } & \textbf{Media } \\ \hline
	&\underline{Oefeningen 51-54 (1 uur)}\newline
	Tijdens deze lesfase maken de studenten oefeningen. Studenten moeten de geziene theorie kunnen omzetten in het toepassen van oefeningen. Introductie oefeningen kunnen moeilijker gelinkt worden met fysische concepten die in het dagelijkse leven aanwezig zijn.  Daarom zijn deze oefeningen heel algemeen.\newline
	Tijdens de lesfase loop ik rond en bezoek ik alle zeven tafels van vier personen. Ik stel actief vragen aan de studenten, zeker wanneer ik problemen denk waar te nemen. Tegelijkertijd help ik studenten die actief vragen stellen door middel van een onderwijsleergesprek.  
	&  \underline{Check-in duo / check-in quatro}\newline 
	De studenten maken oefeningen door eerst zelf kort na te denken over hoe ze de oefening kunnen aanpakken. Daarna overleggen ze per twee of per vier (hun keuze) hoe ze de oefening tot een goed eind kunnen brengen.
	& De studenten gebruiken hun oefeningenbundel en lossen oefeningen op cursusbladen op.
	\\ \hline
\end{tabularx}


	
	
	\begin{tabularx}{1.56\textwidth}{|p{1.5cm}|p{6cm}|X|p{4cm}|}
		\hline
		\textbf{Nr. lesdoel } & \textbf{Inhoud (timing)}  & \textbf{Organisatie } & \textbf{Media } \\ \hline
		&\underline{Inhoudelijke titel (timing)}
	    \textcolor{gray}{(Naast een inhoudelijke titel en de timing, noteer je kort en samenvattend de kerninhoud van de lesfase; uitgebreide informatie/oefeningen/… neem je op in de uitgewerkte media [verwijzen!])}
	    &  \textcolor{gray}{(Naast de benaming van de specifieke werkvorm [bv. placemat-oefening/basis-expertengroep/… en dus níet groepswerk], noteer je kernachtig het organisatorisch verloop van de lesfase. Noteer eveneens belangrijke vragen die je wil stellen.) }
		& 
		\\ \hline
	\end{tabularx}
	
	
	
	
	
	
	
	
\end{landscape}