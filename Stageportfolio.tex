\documentclass[a4paper,12pt,twoside]{article}%twoside
\usepackage[utf8]{inputenc}
\usepackage[dutch]{babel}
\usepackage{fancyhdr, amsmath, color, graphicx, enumitem, tabularx, hyperref, longtable, multirow, placeins, apacite, subcaption,marvosym,multicol}
\usepackage[framemethod=tikz]{mdframed}
\usepackage{hyphenat}
\hyphenation{achter-grond weten-schaps-vakken stage-be-ge-lei-ding oor-spronke-lijk door-lopen stage-be-ge-leid-ster oefen-zittingen ver-antwoordelijke sem-es-ter ver-ant-woorde-lijke we-ten-schappen elektro-me-cha-nica}
\usepackage[europeanresistors,americaninductors,americancurrents,siunitx]{circuitikz}

\usepackage[margin=2.5cm,headheight=68pt]{geometry}
%\usepackage[total={16cm, 22cm}]{geometry}

\pagestyle{fancy}
\fancyhf{}
\fancyhead[LE,RO]{Specifieke Lerarenopleiding voor CVO-studenten}%'E': even page, 'O': odd page
\fancyhead[RE,LO]{Didactische Competentie Stage}
\fancyfoot[RE,LO]{\thepage}
\fancyfoot[LE,RO]{KU Leuven campus Kortrijk Kulak}
%\renewcommand{\labelitemi}{$\circ$}

\definecolor{CVO}{RGB}{232, 0, 97}
\setlength\parindent{0pt}
\title{Stageportfolio}
\author{Kevin Truyaert}
\date{}



%Strikeout and highlight text
\usepackage{soul}
\usepackage{tikz} % only to get \foreach

%\definecolor{yellow}{RGB}{255,255,0}
\sethlcolor{yellow}

% \newcommand*{\YellowHighlight}[1]{{\hl{~#1~}}}
\newcommand*{\PinkHighlight}[2]{{\hspace{-1.5mm}\colorbox{pink}{\parbox{#2}{~#1~}}}}
\newcommand*{\GreenHighlight}[2]{{\hspace{-1.5mm}\colorbox{green}{\parbox{#2}{~#1~}}}}
\newcommand*{\YellowHighlight}[2]{{\hspace{-1.5mm}\colorbox{yellow}{\parbox{#2}{~#1~}}}}
% % % % % %

\usepackage{tabularx,pdflscape,pdfpages}

\newcolumntype{C}[1]{>{\centering\let\newline\\\arraybackslash\hspace{0pt}}m{#1}}
\newcolumntype{Q}[1]{>{\flushleft\let\newline\\\arraybackslash\hspace{0pt}}p{#1}}

\begin{document}
	\maketitle
	
	
	\section*{Identificatiegegevens}
	\begin{center}
		\begin{tabular}{ll}
			\hline
			Naam: & Kevin Truyaert\\ \hline
			Adres: & Bolle-Akkerweg 4\\
			& 8800 Roeselare\\\hline
			Telefoon: & 0032495/928460\\\hline
			Mail: & kevin.truyaert@student.kuleuven.be\\\hline
			Naam stagebegeleider: & Cato De Baets\\ \hline
		\end{tabular}
	\end{center}
	
	\newpage
	\tableofcontents
	\newpage
	
	
	% !TeX root = Stageportfolio.tex

\begin{landscape}
	
	\begin{tabularx}{1.56\textwidth}{|X|}
		\hline
		Naam stagair:  Kevin Truyaert  \\
		Tel.: 0495/928460 \hspace{3cm} e-mail: kevin.truyaert@student.kuleuven.be  \\
		Naam en adres opleidingsinstituut:  KU Leuven Campus Kulak Kortrijk, Etienne-Sabbelaan 53, 8800 Kortrijk  \\
		Naam directie: \\
		Naam stagecoördinator:  David Dudal \\
		\hline
	\end{tabularx}
	\vspace*{-0.4cm}
\section{Observatie- en stageplanning}
\vspace*{-0.3cm}\subsection{Observatieplanning}
\subsubsection{Kulak (LIO)}%
\vspace*{-0.5cm}
%\parskip 
%\vspace{\parskip}
%\begin{minipage}[b]{\textwidth}
\begin{center}
		\includegraphics[scale = 0.9,trim={7.8cm 2cm 7.3cm 2cm} ,clip,angle=-90]{OnePageObservatieKulak}
\end{center}
%\end{minipage}

\subsubsection{VISO}%\\
\vspace*{-0.5cm}
\begin{center}
	\includegraphics[scale = 0.9,trim={2.8cm 3cm 14.2cm 3cm} ,clip,angle=-90]{ObservatielesVISO}
\end{center}
%\begin{tabularx}{1.56\textwidth}{|C{0.05\textwidth}|C{0.15\textwidth}|C{0.1\textwidth}|C{0.2\textwidth}|C{0.09\textwidth}|C{0.21\textwidth}|C{0.1\textwidth}|C{0.25\textwidth}|X|}
%	\hline
%	\textbf{Nr.} & \textbf{Datum} & \textbf{Tijdstip} & \textbf{\begin{tabular}[C]{@{}l@{}}Onderwijsvorm\\ graad en lj\\ studierichting\end{tabular}} & \textbf{Lokaal} &\textbf{\begin{tabular}[C]{@{}l@{}} Leervak en\\ lesonderwerp \end{tabular}} & \textbf{\begin{tabular}[C]{@{}l@{}}AV/TV\\PV/KV\end{tabular}} & \textbf{Mentor/School} & \textbf{\begin{tabular}[C]{@{}l@{}} Handtekening\\mentor\end{tabular}}\\ \hline
%	1 & 12/02/2020 & 8:25-9:15 &\begin{tabular}[C]{@{}l@{}}tso\\3e graad 1ste jaar\\ Technieck-\\Wetenschappen\\\end{tabular} & A013 & \begin{tabular}[C]{@{}l@{}}Toegepaste\\ fysica:\\ Herhaling ERB\\en Inleiding\\ bewegins-\\vergelijking \end{tabular} & TV & \begin{tabular}[C]{@{}l@{}}Mevr. S. Schramme\\ VISO \end{tabular} & \\ \hline
%\end{tabularx}
	
\newpage
\subsection{Actieve stage}
		\subsubsection{Kulak (LIO)}
	%	
	%	\begin{minipage}[t][10cm][t]{0.5\textwidth}
			\begin{figure*}[h]
				\begin{tikzpicture}
				\node[anchor=south west] 
				at (0,0) %left bottom corner of the page
				{\includegraphics[scale = 0.85,trim={4cm 2cm 6cm 1.8cm} ,clip,angle=-90 ]{OnePageLesgevenKulak}};
				\node[fill=white] at(6.58,4.25){4-6};
				\node[fill=white] at(6.58,2.55){7-9};
				\node[fill=white] at(6.58,0.85){10-12};				
				\end{tikzpicture}
			\end{figure*}
		%\end{minipage}
		\vfill\newpage
%	\begin{tabularx}{1.56\textwidth}{|C{0.15\textwidth}|C{0.14\textwidth}|C{0.14\textwidth}|C{0.1\textwidth}|C{0.1\textwidth}|C{0.05\textwidth}|C{0.35\textwidth}|X|}
%		\hline
%		\textbf{Datum} & \textbf{Vestiging} & \textbf{\begin{tabular}[C]{@{}l@{}}Aantal\\ stage-uren\end{tabular}} & \textbf{\begin{tabular}[C]{@{}l@{}}Uur \end{tabular}}    & \textbf{Lokaal}& \textbf{\begin{tabular}[C]{@{}l@{}}AV\\TV\\PV\\KV\end{tabular}}& \textbf{\begin{tabular}[C]{@{}l@{}}Onderwijsvorm\\ graad en lj\\ Vak en lesonderwerp\end{tabular}}  &  \textbf{\begin{tabular}[C]{@{}l@{}}Naam vakmentor\\ + handtekening\end{tabular} } \\ \hline
%		27/11/2019 & Kulak & 1-3 & 10:30-13:00 & A352 & AV & Universiteit\newline 2e jaar Handelsingenieur\newline Conceptuele natuurkunde\newline werkzitting elektromagnetisme & \\ 
%\hline
%		4/12/2019 & Kulak & 4-5 & 10:30-13:00 & A352 & AV & Universiteit\newline 2e jaar Handelsingenieur\newline Conceptuele natuurkunde\newline werkzitting elektromagnetisme & \\ \hline
%		11/12/2019 & Kulak & 6-8 & 10:30-13:00 & A352 & AV & Universiteit\newline 2e jaar Handelsingenieur\newline Conceptuele natuurkunde\newline werkzitting elektromagnetisme & \\ \hline
%		19/12/2019 & Kulak & 9-10 & 10:00-12:30 & A352 & AV & Universiteit\newline 2e jaar Handelsingenieur\newline Conceptuele natuurkunde\newline werkzitting elektromagnetisme & \\ \hline
%	%	 &  &  &  &  &  &  & \\ \hline
%	\end{tabularx}
%	
\subsubsection{VISO Roeselare}

\includegraphics[scale = 0.95,trim={2.7cm 3cm 4cm 3cm} ,clip,angle=-90 ]{P1PlanningVISO}\newpage
\includegraphics[scale = 0.95,trim={2cm 3cm 8cm 3cm} ,clip,angle=-90 ]{P2PlanningVISO}
%\begin{tabularx}{1.56\textwidth}{|C{0.15\textwidth}|C{0.14\textwidth}|C{0.14\textwidth}|C{0.1\textwidth}|C{0.1\textwidth}|C{0.05\textwidth}|C{0.35\textwidth}|X|}
%	\hline
%	\textbf{Datum} & \textbf{Vestiging} & \textbf{\begin{tabular}[C]{@{}l@{}}Aantal\\ stage-uren\end{tabular}} & \textbf{\begin{tabular}[C]{@{}l@{}}Uur \end{tabular}}    & \textbf{Lokaal}& \textbf{\begin{tabular}[C]{@{}l@{}}AV\\TV\\PV\\KV\end{tabular}}& \textbf{\begin{tabular}[C]{@{}l@{}}Onderwijsvorm\\ graad en lj\\ Vak en lesonderwerp\end{tabular}}  &  \textbf{\begin{tabular}[C]{@{}l@{}}Naam vakmentor\\ + handtekening\end{tabular} } \\ \hline
%	20/02/2020 & VISO Roeselare & 11-12 & 8:25-10:05 & PA13\newline PB21 & AV & tso\newline 3e graad 1ste jaar Techniek-Wetenschappen\newline Toegepaste fysica\newline Labo M4: de stroombalans & \\ \hline
%	4/03/2020 & VISO Roeselare & 13 & 8:25-9:15 & PB25 & AV & tso\newline 3e graad 1ste jaar Techniek-Wetenschappen\newline Toegepaste fysica\newline Afwerken labo M4  \&\newline Magnetische flux& \\ \hline
%	5/03/2020 & VISO Roeselare & 14-15 & 8:25-10:05 & PA13 & AV & tso\newline 3e graad 1ste jaar Techniek-Wetenschappen\newline Toegepaste fysica\newline Bespreking labo M4 \& Magnetische fluxverandering \& Inductiespanning: wet van Faraday  & \\ \hline
%	11/03/2020 & VISO Roeselare & 16 & 8:25-9:15 & PA13 & AV & tso\newline 3e graad 1ste jaar Techniek-Wetenschappen\newline Toegepaste fysica\newline  Wet van Lenz \& algemene inductiewet: Faraday-Lenz + oefeningen  & \\ \hline
%\end{tabularx}\newpage
%\begin{tabularx}{1.56\textwidth}{|C{0.15\textwidth}|C{0.14\textwidth}|C{0.14\textwidth}|C{0.1\textwidth}|C{0.1\textwidth}|C{0.05\textwidth}|C{0.35\textwidth}|X|}
%	\hline
%	\textbf{Datum} & \textbf{Vestiging} & \textbf{\begin{tabular}[C]{@{}l@{}}Aantal\\ stage-uren\end{tabular}} & \textbf{\begin{tabular}[C]{@{}l@{}}Uur \end{tabular}}    & \textbf{Lokaal}& \textbf{\begin{tabular}[C]{@{}l@{}}AV\\TV\\PV\\KV\end{tabular}}& \textbf{\begin{tabular}[C]{@{}l@{}}Onderwijsvorm\\ graad en lj\\ Vak en lesonderwerp\end{tabular}}  &  \textbf{\begin{tabular}[C]{@{}l@{}}Naam vakmentor\\ + handtekening\end{tabular} } \\ \hline
%	12/03/2020 & VISO Roeselare & 17-18 & 8:25-10:05 & PA13 & AV & tso\newline 3e graad 1ste jaar Techniek-Wetenschappen\newline Toegepaste fysica\newline Oefeningen algemene inductiewet \& toepassingen inductie & \\ \hline
%	18/03/2020 & VISO Roeselare & 19 & 8:25-9:15 & PA13 & AV & tso\newline 3e graad 1ste jaar Techniek-Wetenschappen\newline Toegepaste fysica\newline Toepassingen inductie & \\ \hline
%	19/03/2020 & VISO Roeselare & 20-21 & 8:25-10:05 & PA13 & AV & tso\newline 3e graad 1ste jaar Techniek-Wetenschappen\newline Toegepaste fysica\newline Labo M5: de transformator  & \\ 
%\hline	
%%	 &  &  &  &  &  &  & \\ \hline
%\end{tabularx}
	


		
\end{landscape}		
		

	
	
	
	% !TeX root = Stageportfolio.tex

\section{Persoonlijk ontwikkelingsplan}
\begin{tabularx}{\textwidth}{|p{0.15\textwidth}|p{0.795\textwidth}|}
	\hline
	\textbf{Lesdoel 1} & 
	\underline{FG 1: de leraar als begeleider van leer- en}\newline \underline{ontwikkelingsprocessen}\newline
	
	\PinkHighlight{1.8 De leraar kan observatie en evaluatie voorbereiden en uitvoeren met het oog op bijsturing en remediëring als onderdeel van het leerproces van een lerende(n) en kan die observatie-en evaluatiegegevens gebruiken om zijn eigen didactische handelen in vraag te stellen en bij te sturen waar nodig.}{12.8cm}\\ \hline
	Actie 1 & Tijdens het lesgeven wil ik problemen i.v.m. de leerstof bij de leerlingen opsporen. Dit kan ik doen door gerichte vragen te stellen, aandachtig te luisteren en te kijken naar de leerlingen terwijl ze aan het werk zijn, hun handelingen te interpreteren \ldots Vanuit dit alles wil ik bij zoveel mogelijk leerlingen een beeld schetsen in verband met hun begrip bij de behandelde leerinhouden. Ik wil me tijdens mijn stage  vooral richten op het ontwikkelen van mijn verschillende `voelsprieten' om dit te bij alle leerlingen op te sporen.	
%	Tijdens het lesgeven wil ik veel in interactie treden. Dit zou ik met zoveel mogelijk leerlingen willen doen en niet steeds dezelfde leerlingen aan bod laten komen. Het lijkt mij uitdagend om dit in realiteit om te zetten. Door hen gerichte vragen te stellen, kan ik kijken waar er mogelijke problemen zijn met de leerstof en van daaruit werken om de lesdoelen begrijpelijk te maken voor alle leerlingen. 
	\\ \hline
	Actie 2 & Wanneer ik problemen bij leerlingen ontdek, wil ik mij richten op het bijsturen van die leerlingen. Hoe kan ik hun problemen tijdens de les aanpakken om ze de leerinhouden te laten begrijpen? Tegelijkertijd wil ik mij focussen om dezelfde soort problemen bij leerlingen tijdens volgende lessen te vermijden door hen op een andere manier te benaderen.
	%Na het verbeteren van een taak, een toets of extra oefeningen wil ik die met de leerling(en) overlopen door de meest voorkomende fouten te bespreken. Zo kan ik hen bijsturen en kan ik de belangrijkste punten aanhalen waar er problemen waren. Tegelijkertijd kom ik zo te weten waar ik te weinig nadruk gelegd heb tijdens de les. Hier kan ik nu mee aan de slag om mijn toekomstige lessen aan te passen en om te verhinderen dat hetzelfde soort fouten bij soortgelijke zaken minder gemaakt worden. 
	\\ \hline
\end{tabularx}


\vspace{0.5cm}
\begin{tabularx}{\textwidth}{|p{0.15\textwidth}|p{0.795\textwidth}|}
	\hline
	\textbf{Lesdoel 2} & 
	\underline{FG 1: de leraar als begeleider van leer- en}\newline \underline{ontwikkelingsprocessen}\newline \YellowHighlight{1.2 De leraar kan zijn didactische handelen afstemmen op enerzijds de doelstellingen en anderzijds de leefwereld, de motivatie, de beginsituatie en de behoeften van de lerende(n) rekening houdend met de diversiteit van de groep.}{12.8cm} \\ \hline
	Actie 1 & Ik wil als leraar in staat zijn om de theorie interessant over te kunnen brengen. Dit wil ik doen door actuele zaken als voorbeeld van die theorie te gebruiken.  Door actuele thema's en alledaagse voorwerpen te linken met fysische verschijnselen, hoop ik dat de leerlingen de wereld rond hen beter begrijpen. \\ \hline
	Actie 2 &  Naast het binnenbrengen van de actualiteit tijdens de lessen fysica, wil ik de leerlingen ook op andere manieren gaan stimuleren en motiveren. Dit wil ik doen door de interesse van de leerlingen bij de lessen proberen te betrekken. Dit kan ik enkel doen als ik oprecht interesse toon in de leefwereld van de leerlingen en die leefwereld in de lessen probeer binnen te terkken.\\ \hline
\end{tabularx}


\vspace{0.5cm}
\begin{tabularx}{\textwidth}{|p{0.15\textwidth}|p{0.795\textwidth}|}
	\hline
	\textbf{Lesdoel 3} & 
	\underline{FG 1: de leraar als begeleider van leer- en}\newline \underline{ontwikkelingsprocessen}\newline
	\GreenHighlight{1.5 De leraar kan aangepaste werkvormen en groeperingsvormen bepalen en gebruiken.}{12.8cm}
	\\ \hline
	Actie 1 & Ik verzorg reeds drie jaar oefenzittingen aan de universiteit. Dit jaar wil ik iets nieuws proberen en de studenten actiever krijgen tijdens de oefenzittingen. Ik wil hen in groep aan de oefeningen laten werken, waardoor ze met elkaar in interactie kunnen treden om de oefeningen samen tot een goed eind te kunnen brengen.   \\ \hline
	Actie 2 & Bij de lessen die ik in het middelbaar zal verzorgen, wil ik terugkoppelen naar mijn stagelessen die ik bij DCO deed. Hier gaf ik telkens de introductieles van een nieuw stuk theorie. Die gaf ik relatief `klassiek', waarbij ik als leerkracht veel aan bod kwam. Ik wil nu proberen om de leerlingen zal actiever aan de slag te zetten bij de start van een nieuw stuk. Ik zie dit nu ook meer zitten, omdat ik meer dan één les(blok) per klas zal brengen. Dit zal als gevolg hebben dat ik een groter plan kan uitwerken en zo proberen om mijn eigen lesgeven te innoveren.    \\ \hline
\end{tabularx}






	
	
	\begin{landscape}
		
		\section{Bespreking lesobservaties}
		
		\subsection{Bespreking observatieles 1 Kulak  (LIO)}
		\begin{tabularx}{1.56\textwidth}{|Q{0.25\textwidth}|Q{0.1\textwidth}|Q{0.25\textwidth}|Q{0.85\textwidth}|}\hline
			\textbf{Naam student: Kevin Truyaert} & & Aandachtspunten (o.b.v. POP) & Reflectie:\newline -Wat leerde ik uit mijn observatie over mijn aandachtspunten? \newline -Wat doe ik ermee tijdens mijn stage?\\\hline
			Observatieles 1 \newline \underline{Datum:} 14/11/2019 \underline{Klas:} 2e bachelor Handelsingenieurs \newline \underline{Lesonderwerp:} De wet van Gauss bij geleiders & 1 & Hoe evalueert de prof tijdens de les om te zien of bepaalde lesdoelen tijdens de les bereikt zijn? & Na het zien van een stuk theorie, overloopt de prof samen met de studenten een toepassing op het net geziene stuk theorie. De toepassing wordt door de studenten uitgelegd. De prof stuurt hen door gericht vragen te stellen.  Daarnaast worden de studenten ook bij het zien van de afleiding van nieuwe theorie vaak betrokken. Ze hebben vaak de net geziene onderwerpen nodig om het nieuwe te begrijpen. Ook hier stelt de prof opnieuw gerichte vragen om het antwoord van de leerlingen te krijgen.\newline Ik zie dit als een snelle manier om te polsen of je publiek, of toch een deel ervan, de lesdoelen bereikt heeft. Ik zou echter graag nog een breder gamma aan snelle evaluatietools in mijn bezit krijgen. Tijdens de les merkte ik ook wel op dat het steeds dezelfde handen waren die antwoordden op de vragen van de prof. Dit zou je kunnen vermijden door iemand aan te duiden.  \\
			& 2 & Hoe reageert de prof bij het krijgen van een goed antwoord? En hoe bij het krijgen van een fout antwoord? & Wanneer de prof een vraag stelt aan de studenten, al dan niet evaluerend, becommentarieert hij steeds het gegeven antwoord. Indien het correct is, dan geeft hij ofwel positieve feedback, ofwel vraagt hij door, om te zien of de student, of andere studenten, ook de diepere begrippen vatten en kunnen uitleggen. Indien de student fout antwoordt, dan vraagt de prof verder. 
		\end{tabularx}
		
		
		\begin{tabularx}{1.56\textwidth}{|Q{0.25\textwidth}|Q{0.1\textwidth}|Q{0.25\textwidth}|Q{0.85\textwidth}|}
			& & & Hij doet dit meestal door de vragen te herformuleren of de vraag in deelvragen op te delen. Ook wanneer de prof geen antwoord krijgt, deelt hij vaak de vraag in kleinere deelvragen op.  \\
			& & & Het opsplitsen van vragen in kleinere deelvragen vind ik een positief gegeven wanneer er geen antwoord komt. De leerlingen denken op die manier nog steeds na over het probleem en ook de leerlingen die nog niet zoveel begrijpen kunnen  bij complexere problemen beter volgen door de kleinere stappen. De prof benadert de studenten positief, zowel bij correcte als foutieve antwoorden. Hij vraagt even door bij de persoon die foutief antwoordde, maar trekt die vragen ook open naar alle studenten. Op die manier voelt de persoon die foutief geantwoord heeft zich niet geviseerd, wat ik als positief ervoer. \\\hline
		\end{tabularx}
		
	\end{landscape}

\includepdf[scale = 0.8,pages = 1,pagecommand=\subsection*{Notities observatieles 1 Kulak (LIO)}]{Observaties_OpgelosteOef}
\includepdf[scale = 0.8,pages =2-8,pagecommand=]{Observaties_OpgelosteOef}

\begin{landscape}
	
		\subsection{Bespreking observatieles 2 Kulak}
		\begin{tabularx}{1.56\textwidth}{|Q{0.25\textwidth}|Q{0.1\textwidth}|Q{0.25\textwidth}|Q{0.85\textwidth}|}\hline
			Observatieles 2 \newline \underline{Datum:} 20/11/2019\newline \underline{Klas:} 2e bachelor Handelsingenieurs \newline \underline{Lesonderwerp:}  Elektrische stroom, weerstanden in serie en parallel, wetten van Kirchhoff& 1 & Hoe kan de prof niet eenvoudige fysische begrippen toch conceptueel uitleggen, zonder dat de studenten de harde fysica zien? & Bij het volledige elektromagnetisme gedeelte van deze cursus is het moeilijk om de inhoud conceptueel, met zo weinig mogelijk vergelijkingen, te benaderen. Toch slaagt de prof hier grotendeels in. De harde afleidingen blijven achterwege, terwijl er wel nog een voldoende samenspel is tussen uitleg en vergelijkingen. Zo worden de afleidingen van nieuwe vergelijkingen (bijvoorbeeld van de stroomdichtheid tijdens deze les) wel nog gedaan. De uitwerking staat echter al volledig op de slide en de prof begeleidt de studenten door middel van het stellen van vragen (Wat is stroom weer juist? Hoe tellen we de totale lading in een volume? ...). Op die manier is het mogelijk dat de studenten bepaalde afleidingen wel zien en conceptueel kunnen vatten wat er juist gebeurt, door middel van een sterke begeleiding. Naast afleidingen, gebruikt de prof vaak ook analogieën. Wanneer het over de wet van Ohm gaat en de eigenschappen van de stroom (ladingen) in een draad bespreekt, legt hij vaak de analogie met mensen die door een deur moeten. \newline Omdat fysica in het middelbaar nog minder afleidingen heeft en vaak ook conceptueel uitgelegd wordt, vind ik het belangrijk om correcte analogieën te hanteren bij mijn lessen in het middelbaar. De fysische afleidingen zou ik ook eens proberen samen met de leerlingen op te bouwen, zonder dat de volledige afleiding al te zien is. Door middel van die vraagstelling zie ik wel mogelijk dat de leerlingen een veel beter begrip zullen hebben van de afleiding zelf, dan wanneer je als leerkracht gewoon vergelijkingen in elkaar invult. \\
			
		\end{tabularx}
		
		\begin{tabularx}{1.56\textwidth}{|Q{0.25\textwidth}|Q{0.1\textwidth}|Q{0.25\textwidth}|Q{0.85\textwidth}|}	& 2 & Durven de studenten vragen te stellen? Hoe zorgt de prof ervoor dat de studenten dit durven te doen? & Tijdens de observatie van beide lessen viel het mij op dat er meerdere studenten waren die vragen aan de prof durfden te stellen. Dit gebeurde zowel nadat de prof hen uitdrukkelijk gevraagd had of ze nog met vragen zitten na een moeilijker stuk gezien hebben als op zelfstandige basis. Ondanks de grotere groep van 60 studenten heeft het merendeel geen probleem om bij dit vak vragen te stellen. Ik kan hieruit besluiten dat er een positieve klassfeer heerst tijdens deze lessen. Dit komt vooral denk ik omdat er respect uit beide kampen komt. Als leerkracht wil ik ook werken aan een positief klasklimaat waarin de studenten durven te leren. \\\hline	
		\end{tabularx}
	\end{landscape}


\includepdf[scale = 0.8,pages = 9,pagecommand=\subsection*{Notities observatieles 2 Kulak (LIO)}]{Observaties_OpgelosteOef}
\includepdf[scale = 0.8,pages =10-16,pagecommand=]{Observaties_OpgelosteOef}


\begin{landscape}
	
	\subsection{Bespreking observatieles VISO}
	\begin{tabularx}{1.56\textwidth}{|Q{0.25\textwidth}|Q{0.1\textwidth}|Q{0.25\textwidth}|Q{0.85\textwidth}|}\hline
		Observatieles VISO \newline \underline{Datum:} 12/02/2020\newline \underline{Klas:} 5e Techniek-Wetenschappen \newline \underline{Lesonderwerp:}  Herhalingsoefeningen ERB en inleiding de een-dimensionale bewegingsvergelijking & 1 & Hoe handelt de leerkracht naar de leerlingen toe? Hoe voorziet ze output op hun input en input op hun output? & De leerkracht pikt lichamelijke signalen van de leerlingen zeer vlot op en spreekt hen hierop ook aan. Indien een leerling drukt aan het schrijven is bij een oefening die eigenlijk al gemaakt had moeten zijn, spreekt ze hen hierover ook aan. Anderzijds blijft ze die leerling ook actief bij de les betrekken, wanneer die bij een volgende oefening ook aan het werk is. Een andere leerling gaf het antwoord, de leerkracht speelt dan de vraag hoe dit antwoord tot stand gekomen is aan de eerste leerling. \newline Een andere leerling had haar cursus niet bij. De leerkracht merkt dit meteen op en vraagt hier verder om. Blijkbaar is dit al enkele maal voorgekomen en de leerkracht besluit daarom om een nota in de agenda van de leerling te noteren. Ze zegt tegen de leerling in kwestie dat dit nog een duidelijk werkpunt is voor haar. \newline Leerlingen antwoorden spontaan op vragen die de leerkracht stelt. Hierdoor krijg je heel vaak dat dezelfde leerlingen eigenlijk het antwoord geven. Hier houdt de leerkracht zelf ook rekening mee en speelt daarom vaak verdere vragen door naar de andere leerlingen, of ze duidt meteen iemand aan om haar vraag te beantwoorden. Dit doet ze steeds via hun naam. Op die manier test ze ook of de leerlingen voldoende mee zijn, iets wat zeker doenbaar is in een klas van acht leerlingen.  \\
	\end{tabularx}
	\begin{tabularx}{1.56\textwidth}{|Q{0.25\textwidth}|Q{0.1\textwidth}|Q{0.25\textwidth}|Q{0.85\textwidth}|}	& 2 & Betrekt de leerkracht de wereld, de actualiteit, de interessevelden van de leerlingen \ldots bij de les?  & Tijdens de observatie van deze les rond de eenparige rechtlijnige beweging werden de oefeningen duidelijk vertaald naar een `verhaaltje' en konden de leerlingen zich steeds iets `voorstellen' bij de oefening. Zo gaan er oefeningen over de gemiddelde snelheid van een auto die een rit van Oostenrijk naar Belgi\"e maakt. Andere oefeningen gaan dan over twee fietsers die hetzelfde traject afleggen, maar die verschillende gemiddelde snelheden hebben over verschillende deeltrajecten.\newline Tegelijkertijd maakt de leerkracht de connectie met het vak wiskunde wanneer het over de functie en de grafiek van de bewegingsvergelijking gaat. Ze haalt aan dat bij fysica er altijd maar een bepaald stuk van de grafiek `nuttig' en `interessant' is, terwijl bij wiskunde de volledige grafiek nodig is. Ze duidt dit duidelijk door meermaals te herhalen dat de grafieken binnen fysica als lijnstukken gerepresenteerd worden in plaats van rechten in wiskunde.  \\\hline	
	\end{tabularx}
\end{landscape}

\includepdf[scale = 0.8,pages = 1,pagecommand=\subsection*{Notities observatieles 1 VISO}]{ObservatieVISO}
\includepdf[scale = 0.8,pages =2-,pagecommand=]{ObservatieVISO}


%%Even in comments zetten
%%	
	% !TeX root = Stageportfolio.tex



\begin{landscape}
	\section{Lesvoorbereidingen en bijhorende media}
	
	\subsection{Les 1-3}
	\begin{tabularx}{1.56\textwidth}{|p{0.55\textwidth}|X|}\hline
		\textbf{Administratieve gegevens}\newline\newline
		Kevin Truyaert\newline\newline
		Universiteit\newline
		Handelsingenieur, 2de fase\newline
		Leerplannummer: De inhoud is terug te vinden op de ECTS fiche: \href{https://onderwijsaanbod.kuleuven.be/syllabi/n/D0W55AN.htm}{https://onderwijsaanbod.kuleuven.be/syllabi/n /D0W55AN.htm} \newline
		Lesonderwerp: `Oefenzitting elektromagnetisme: wat zijn de relaties tussen de elektrische kracht, de  elektrische potentiaal, de elektrische flux en de elektrische capaciteit' & \textbf{Doelstellingen}\newline
		\newline\newline 
		\underline{Leerplandoelen}\newline - Elektriciteit: elektrische lading, elektrisch veld (wetten van Coulomb en Gauss), elektrische flux, elektrische potentiaal, energie in een elektrisch veld \newline\newline
		\underline{Lesdoelen}\newline
		\vspace{-0.5cm}
		\begin{enumerate}
			\item De studenten kunnen via de wet van Coulomb de elektrostatische kracht tussen ladingen berekenen.
			\item De studenten kunnen de relatie tussen de elektrostatische kracht, het elektrisch veld en een lading toepassen in een probleem.
			\item De studenten kunnen de elektrostatische kracht binnen de tweede wet van Newton herkennen.
			\item De studenten kunnen een Gaussoppervlak in een situatie opstellen.
			\item De studenten zijn in staat om de elektrische flux te bepalen met gebruik van een Gaussoppervlak.
			\item De studenten kunnen het elektrisch veld en de elektrische flux van een boloppervlak in functie van de afstand afleiden.
			\item De studenten kunnen het elektrisch veld en de elektrische flux van een opgevulde, geleidende bol in functie van de afstand afleiden.
			%\item De studenten kunnen.
		\end{enumerate} \\\hline
	\end{tabularx}


	\begin{tabularx}{1.56\textwidth}{|p{0.55\textwidth}|X|}
		\hline
		\multirow{2}{0.55\textwidth}{\textbf{Beginsituatie}\newline De studenten hebben de theorie rond de  begrippen van `Elektrisch veld', `Elektrische potentiaal', `Elektrische flux' en de wet van Coulomb in de week van 12-15 november gezien, twee weken voor de oefenzitting. Hierdoor zullen ze al tijd gehad hebben om de theorie te bekijken, wat aangemoedigd wordt door het maken van een voorbereidende opdracht die ik de week voor de oefenzitting op Toledo plaats.\newline\newline De minderheid van de studenten heeft  interesse bij mechanica, het eerste deel van de cursus, getoond. Het gedeelte over elektromagnetisme ervaren ze meestal interessanter. Er zijn 28 studenten die deze sessie volgen, maar gemiddeld gezien zijn er 25 studenten aanwezig geweest bij de voorbije lessen.\newline\newline Het lokaal kan 30 studenten plaatsen. Er is een dubbel krijtbord ter beschikking en de mogelijkheid tot projectie. Wanneer er geprojecteerd wordt, hangt het projectiescherm grotendeels over beide borden.  }& \textbf{Acties}\newline  - Om de studenten te stimuleren om zelf aan de slag te gaan, wil ik hen in groepjes van vier tot zes studenten aan de slag zetten. Hierdoor kan ik gerichtere feedback geven, aangezien de studenten onderling elkaar kunnen aanzetten tot het vinden van oplossingen. Naast de helpende rol, kan ik ook interacties tussen de studenten onderling volgen en inspringen waar nodig: ofwel bij het maken van een fout, of wanneer ik hun uiteenzetting zeer goed vind en er nog dieper op in wil gaan. Dit wil ik steeds vanuit het onderwijsleergesprek proberen te realiseren.  \newline\newline
		- Bij het begin van de les overloop ik nog even de theorie rond de elektrische grootheden en hun onderlinge relaties. Dit zet ik op één van de twee krijtborden en laat ik gedurende de hele les staan. Zo kunnen de studenten steeds makkelijk teruggrijpen naar de theorie. \newline\newline
		- Ik werk niet met projectie, maar noteer alles op het bord, omdat het projectiescherm voor zo goed als beide borden hangt. Hierdoor houd ik een tempo aan waarop de studenten makkelijker kunnen volgen, doordat ik alles zelf ook neerschrijf.  
		
		\\ \cline{2-2}
		  & \textbf{Bronnen}\begin{itemize}
		  	\item Dudal, D., Temmerman, E., Truyaert, K., Heymans, S. (2019). Slides conceptuele natuurkunde
		  	\item Dudal, D., Temmerman, E., Truyaert, K., Heymans, S. (2019). Oefeningenbundel conceptuele natuurkunde
		  	\item Giancoli, D. C. (2008). Physics for scientists and engineers. Pearson Education International.
		  \end{itemize}\\ \hline
	\end{tabularx}


\newpage
	
	
	
	\begin{tabularx}{1.56\textwidth}{|p{1.5cm}|p{6cm}|X|p{4cm}|}
		\hline
		\textbf{Nr. lesdoel } & \textbf{Inhoud (timing)}  & \textbf{Organisatie } & \textbf{Media } \\ \hline
		&\underline{Inhoudelijke titel (timing)}
	    \textcolor{gray}{(Naast een inhoudelijke titel en de timing, noteer je kort en samenvattend de kerninhoud van de lesfase; uitgebreide informatie/oefeningen/… neem je op in de uitgewerkte media [verwijzen!])}
	    &  \textcolor{gray}{(Naast de benaming van de specifieke werkvorm [bv. placemat-oefening/basis-expertengroep/… en dus níet groepswerk], noteer je kernachtig het organisatorisch verloop van de lesfase. Noteer eveneens belangrijke vragen die je wil stellen.) }
		& 
		\\ \hline
	\end{tabularx}
	
	
	
	
	
	
	
	
\end{landscape}
	% !TeX root = Stageportfolio.tex



\begin{landscape}
	
	\subsubsection{Les 4-5}
	\begin{tabularx}{1.56\textwidth}{|p{0.55\textwidth}|X|}\hline
		\textbf{Administratieve gegevens}\newline\newline
		Kevin Truyaert\newline\newline
		Universiteit\newline
		Handelsingenieur, 2de fase\newline
		\underline{ECTS-fiche}: De inhoud is terug te vinden op de ECTS fiche: \href{https://onderwijsaanbod.kuleuven.be/syllabi/n/D0W55AN.htm}{https://onderwijsaanbod.kuleuven.be/syllabi/n /D0W55AN.htm} \newline
		\underline{Lesonderwerp}:\newline `Oefenzitting elektromagnetisme: serie en parallel, de wet van Ohm, vermogen' & \textbf{Doelstellingen}\newline\vspace{0.5cm}
		\underline{Punt op de ECTS-fiche}
		\vspace{-0.5cm}\newline  - condensatoren: capaciteit, diëlektrische materialen, serie- en parallelschakeling, bouwvormen \newline
		- weerstanden: soortelijke weerstand, geleiders, isolatoren, serie- en parallelschakeling en elektrisch vermogen\newline
		- toepassing: elektrische veiligheid en elektrische huisinstallatie \newline
		\underline{Lesdoelen}\newline
		\vspace{-0.5cm}
		\begin{enumerate}[itemsep=0.08\baselineskip]
			\item De studenten kunnen de begrippen `spanningsverschil', `stroom', `weerstand', `energie', `vermogen' en `lading' met elkaar in verband brengen en interpreteren.
			\item De studenten kunnen de equivalente capaciteit van condensatoren in serie en parallel berekenen.
			\item De studenten kunnen het vermogen dat over weerstanden verloren gaat berekenen.
			\item De studenten kunnen het vermogen van een energiecentrale berekenen.
			\item De studenten kunnen de gegevens van een energiecentrale interpreteren en linken met de fysische variabelen.
			\item De studenten kunnen berekenen of de zekering van het circuit  bij een bepaalde belasting zal kapot gaan.
			\item  De studenten kunnen de equivalente weerstand van weerstanden in serie en parallel berekenen.
		    \item De studenten kunnen in duo over de oefening discussiëren en samen oplossingsgericht werken.
		\end{enumerate} \\\hline
	\end{tabularx}


	\begin{tabularx}{1.56\textwidth}{|p{0.55\textwidth}|X|}
		\hline
		\multirow{2}{0.55\textwidth}{\textbf{Beginsituatie}\newline De studenten hebben de theorie rond de  begrippen van in verband met condensatoren, weerstanden en vermogen twee weken voor de oefenzitting gezien in de hoorcolleges. Daar hebben ze eveneens de onderlinge relaties tussen stroom, lading, potentiaalverschil, weerstand, capaciteit en vermogen bestudeerd. \newline\newline Deze oefenzitting heeft meer raakvlakken met de interesse van de student omdat de oefeningen tastbaarder zijn. Zo gaan er oefeningen over energiecentrales en de transport van die energie tot bij je thuis en over zekeringen bij toestellen die al dan niet kapot gaan. Er zijn 28 studenten die deze sessie zouden moeten volgen, die waren er allemaal tijdens de vorige lessenreeks.\newline\newline Het lokaal kan 30 studenten plaatsen. Ik laat de banken staan in drie rijen van tien studenten. Er is een dubbel krijtbord ter beschikking en de mogelijkheid tot projectie. Wanneer er geprojecteerd wordt, hangt het projectiescherm grotendeels over beide borden.  }& \textbf{Acties}\newline  - Tijdens dit lesblok wil ik de nadruk leggen op de toepassingen en de interpretatie van de fysische begrippen rond spanningsverschil en stroom. Ik wil dat de studenten die eerst zelf formuleren om daarna terug te koppelen, afhankelijk van hun antwoord. Hierdoor laat ik ze met hun buren per twee of per drie aan de slag gaan.     \newline\newline
		- Bij het begin van de les overloop ik nog even de theorie rond de fysische begrippen en hun onderlinge relaties. Dit zet ik op één van de twee krijtborden en laat ik gedurende de hele les staan. Zo kunnen de studenten steeds makkelijk teruggrijpen naar de theorie. Ik treed eerst in gesprek met de studenten om vanuit hun antwoorden de theorie aan te reiken. \newline\newline
		- Ik werk niet met projectie, maar noteer alles op het bord, omdat het projectiescherm voor zo goed als beide borden hangt. Hierdoor houd ik een tempo aan waarop de studenten makkelijker kunnen volgen, doordat ik alles zelf ook neerschrijf.  
		
		\\ \cline{2-2}
		  & \textbf{Bronnen}\begin{itemize}
		  	\item Dudal, D., Temmerman, E., Truyaert, K., Heymans, S. (2019). Slides conceptuele natuurkunde
		  	\item Dudal, D., Temmerman, E., Truyaert, K., Heymans, S. (2019). Oefeningenbundel conceptuele natuurkunde
		  	\item Giancoli, D. C. (2008). Physics for scientists and engineers. Pearson Education International.
		  \end{itemize}\\ \hline
	\end{tabularx}


\newpage
	
	\begin{tabularx}{1.56\textwidth}{|p{1.5cm}|p{6cm}|X|p{4cm}|}
		\hline
		\textbf{Nr. lesdoel } & \textbf{Inhoud (timing)}  & \textbf{Organisatie } & \textbf{Media } \\ \hline
		1 &\underline{Herhaling theorie (15 minuten)}\newline
		De theorie rond de fysische begrippen lading, stroom, weerstand, spanningsverschil, capaciteit en vermogen worden door de studenten aangereikt. Zij interpreteren ook wat de vergelijking voorstelt en delen dit met hun medestudenten. 
		&  \underline{Onderwijsleergesprek}\newline 
		Ik start deze les met het begrip lading aan de studenten te poneren. Ik vraag hen of ze dit in verband kunnen brengen met nog andere grootheden. Vanaf de start kan ik al verschillende antwoorden krijgen. Ik plaats deze op het bord afhankelijk van hoe de studenten me dit aanreiken. Ik vraag hen niet enkel om de zaken te linken, maar ook om een uitleg waarom dit zo is. De studenten bouwen dus zelf de theorie op en interpreteren die. \newline
		Ik focus mij op het correct interpreteren van de bekomen vergelijkingen. Wanneer de student begrijpt wat er in de vergelijking staat, dan zal hij/zij deze beter begrijpen. Ik stuur de gegeven interpretatie van de student bij indien nodig, of vraag iets dieper door wanneer de student niet volledig is. 
		\newline 
		Hierna noteer ik de oefeningen op bord die gemaakt kunnen worden. Dit zijn oefeningen 58 t.e.m. 64. Ik verwacht dat deze oefeningen door iedereen gemaakt kunnen worden.  
		& Krijtbord (Bekomen bordschema wordt in bijlage toegevoegd)
		\\ \hline
	\end{tabularx}



	
	\begin{tabularx}{1.56\textwidth}{|p{1.5cm}|p{6cm}|X|p{4cm}|}
		\hline
		\textbf{Nr. lesdoel } & \textbf{Inhoud (timing)}  & \textbf{Organisatie } & \textbf{Media } \\ \hline
		1\newline 2\newline 3\newline 8&\underline{58 - 60 (1 uur)}
	    De studenten maken oefeningen 58 t.e.m. 60. Deze bespreken condensatoren en de relatie tussen de capaciteit van condensatoren, lading en spanningsverschil. De student is met deze oefeningen ook in staat om serie en parallel verbanden tussen condensatoren te bespreken. Daarnaast wordt er ook een grotere oefening gemaakt die de relaties tussen vermogen, spanningsverschil, weerstand en stroom bespreekt enerzijds en anderzijds de relaties tussen lading en stroom en tussen vermogen en energie.
	    &  \underline{Oplossingensleutel}
	    	De studenten krijgen de eindantwoorden ter beschikking en kunnen zo controleren of ze een opgave correct opgelost hebben. Ik loop ondertussen rond om vragen van studenten te beantwoorden, maar ook om vragen aan de studenten te stellen. Hierbij heb ik vooral aandacht voor de interpretaties van oefening 60, omdat deze lampen met een verschillend wattage bespreekt. In ieder huis komen er lampen met een verschillend wattage voor, waardoor het interessant is dat de studenten dit correct kunnen interpreteren. Anderzijds bespreekt deze oefening heel wat relaties tussen de fysische begrippen van deze les. 
	    
		& Oplossingenbundel\newline Oefeningenbundel + cursuspapier
		\\ \hline
	\end{tabularx}
	
	
	
	\begin{tabularx}{1.56\textwidth}{|p{1.5cm}|p{6cm}|X|p{4cm}|}
		\hline
		\textbf{Nr. lesdoel } & \textbf{Inhoud (timing)}  & \textbf{Organisatie } & \textbf{Media } \\ \hline
		&\underline{Pauze}\newline
		
		
		&    De studenten krijgen 15 minuten pauze en mogen het lokaal verlaten. Op deze manier kunnen ze het laatste uur weer met volle aandacht werken. De studenten zullen na de pauze zich focussen op het begrip vermogen en dit vooral met de relatie tussen spanningsverschil en weerstand. \newline
		Wanneer de eerste studenten het lokaal terug binnen sijpelen, sla ik een praatje met hen, waarbij ik niet over de leerstof begin. 
		& 
		\\ \hline
	\end{tabularx}
	
		\begin{tabularx}{1.56\textwidth}{|p{1.5cm}|p{6cm}|X|p{4cm}|}
		\hline
		\textbf{Nr. lesdoel } & \textbf{Inhoud (timing)}  & \textbf{Organisatie } & \textbf{Media } \\ \hline
		1\newline 3\newline 4\newline 5 \newline 8&\underline{61 (15 minuten)}\newline
		Deze oefening handelt over de generatie van energie in een energiecentrale en het transport van deze energie naar de elektriciteitscabine in de straat. Eerst berekenen de studenten het totale vermogen opgewekt in deze centrale. Daarna berekenen ze het verlies van dit vermogen vanwege  het transport naar de elektriciteitscabine in de straat. Hierbij zullen studenten de fout maken om het spanningsverschil opgewekt binnen de centrale te gebruiken in plaats van de spanningsval vanwege de transportdraad te gebruiken. 
		
		& \underline{Oplossingensleutel}
		   De studenten krijgen 15 minuten om deze oefening te maken.  Tijdens deze oefening wil ik vooral dat de studenten mij kunnen uitleggen wat hun interpretatie is bij deze oefening. Ze moeten duidelijk begrijpen dat eenzelfde begrip, hier spanningsverschil, in meerdere contexten gebruikt kan worden binnen eenzelfde oefening. Zo is er de centrale die voor een positief spanningsverschil (energiebron) zorgt en de elektriciteitskabel die een negatief spanningsverschil veroorzaakt (verbruiker). Daarom zal ik steeds een bijvraag stellen aan de studenten: `welk spanningsverschil heb je nog in de elektriciteitscabine aanwezig?'.
		& Oplossingenbundel\newline Oefeningenbundel + cursuspapier
		\\ \hline
	\end{tabularx}
	
	
	\begin{tabularx}{1.56\textwidth}{|p{1.5cm}|p{6cm}|X|p{4cm}|}
		\hline
		\textbf{Nr. lesdoel } & \textbf{Inhoud (timing)}  & \textbf{Organisatie } & \textbf{Media } \\ \hline
		1\newline 3\newline 6\newline 7\newline 8\newline &\underline{62-64,56 (45 minuten)}\newline
		De studenten richten zich bij deze oefeningen vooral op het gebruik van de relaties met betrekking tot weerstanden (of verbruiker). Ze zullen enerzijds de stroom doorheen een verbruiker moeten berekenen, om zo te constateren of de geplaatste zekering doorbrandt of niet. Tegelijkertijd leren ze dus ook de werking van een zekering, iets waar iedereen wel eens mee geconstateerd wordt. Anderzijds berekenen ze ook hoelang een verbruiker werkt gegeven een set aan batterijen. Tenslotte berekenen de studenten ook hoe lampen (verbruikers) het meest energie verbruiken, in serie of parallel.  
		& \underline{Oplossingenbundel}\newline Tijdens deze oefeningen loop ik rond om de vragen van de studenten te beantwoorden of om hen vragen te stellen bij bepaalde zaken die ik op hun blad zie. Hier wil ik hen vooral mondeling info omtrent zekeringen meegeven (waar vind je die in jullie huis?, ooit al iets mee moeten doen?, waarom zitten die daar? \ldots). Ook bij serie- en parallelschakelingen wil ik hen inzichten meegeven. De lampen zullen hier feller schijnen in parallel, maar ik wil van hen ook horen dat de batterij veel minder lang zal meegaan dan wanneer die lampen in serie staan. Dit wil ik opnieuw bereiken door vragen te stellen. Dit laatste vind ik belangrijk om mee te geven aan mijn studenten, dus breng ik deze laatste oefening nog eens klassikaal waarbij ik hen klassikaal gerichte vragen stel in verband met de relaties (lesdoel 1), om dan van een student te horen dat het verbruik bij batterijen ook afhankelijk is van de schakeling. Wanneer er nog tijd over is, kunnen de studenten oefening 56 nog maken, aangezien die vorige les niet aan bod gekomen is.
		& Oplossingenbundel\newline Oefeningenbundel + cursuspapier\newline Krijtbord
		\\ \hline
	\end{tabularx}
	
	


\begin{tabularx}{1.56\textwidth}{|p{1.5cm}|p{6cm}|X|p{4cm}|}
	\hline
	\textbf{Nr. lesdoel } & \textbf{Inhoud (timing)}  & \textbf{Organisatie } & \textbf{Media } \\ \hline
	&\underline{Afsluiten (5 minuten)}\newline 
	&  \underline{Afsluiten (5 minuten)}\newline
	Ik herhaal nog even kort wat er van de studenten verwacht werd tijdens deze les en wat ze bijgeleerd hebben. Ik zeg ook wat het onderwerp van volgende les is en vraag aan de studenten om de theorie nog even te herhalen.
	& 
	\\ \hline
\end{tabularx}
	
	
	
	
	
\end{landscape}

\subsection*{Bijlage: oplossingsleutels}
De oplossingssleutels zijn hieronder bijgevoegd. Deze zijn telkens zo opgesteld dat de studenten niet de oplossing uitgewerkt krijgen, maar dat er enkele gerichte vragen of hints zijn die hen op de weg kunnen helpen wanneer ze vast zitten. Wanneer ze de oefening correct hebben, kunnen ze stilstaan bij die vragen en controleren of ze de oefening ook begrijpen.\newline
De oplossingssleutels zijn hier samen gevoegd. In realiteit werden meerdere sleutels per blad afgedrukt en daarna uitgeknipt.

\subsection*{58}
\underline{Methode}
\begin{enumerate}
	\item Welke relaties met betrekking tot de capaciteit van een condensator ken je? 
	\item Met twee van deze relaties kan je het antwoord bekomen!
\end{enumerate}

\underline{Extra vragen}
\begin{enumerate}
	\item Wat betekent het dat de lucht `zuiver en droog' is? Welke factor zou wijzigen als deze situatie anders is?
	\item Wat gebeurt er wanneer er 1~$\mu$C meer op de wolk geplaatst zou worden?
	\item Hoe verhoudt de maximale lading zich tot de dimensies van een condensator?
\end{enumerate}





\subsection*{59}
\underline{Methode}
\begin{enumerate}
	\item Een equivalente condensator zoeken voor dit systeem moet in meerdere stappen gebeuren. Wat los je eerst op?
\end{enumerate}

\underline{Extra vragen}
\begin{enumerate}
	\item Stel dat iedere condensator een weerstand zou zijn. Hoe groot is de equivalente weerstand? Verklaar het verschil.
\end{enumerate}




\subsection*{60}
\underline{Methode}
\begin{enumerate}
	\item Je mag er van uit gaan dat de bron aan een constante van 120~V blijft leveren. 
	\item Wat betekent het om zwak/fel te branden?
	\item Wat brengt een verschil in ladingen teweeg? En wat gebeurt er met lading hierin?
	\item Wat stelt de eenheid kWh voor? Wat is de SI-eenheid hiervan?
\end{enumerate}


\subsection*{61}
\underline{Methode}
\begin{enumerate}
	\item Wat zijn parameters van de centrale?
	\item Welke parameters horen bij de kabel?
	\item Een bron zorgt voor een positief spanningsverschil, een verbruiker voor een negatief.
\end{enumerate}

\underline{Extra vragen}
\begin{enumerate}
	\item Welke spanning is op locatie nog beschikbaar?
\end{enumerate}



\subsection*{62}
\underline{Methode}
\begin{enumerate}
	\item Waarvoor zorgt een zekering bij een schakeling?
	\item De zekering kan een stroom van 4~A aan. Wat is de stroom in de schakeling?
\end{enumerate}

\underline{Extra vragen}
\begin{enumerate}
	\item Waar kom je zekeringen zoal tegen?
\end{enumerate}




\subsection*{63}
\underline{Methode}
\begin{enumerate}
	\item Wat gebeurt er met het spanningsverschil wanneer je batterijen in serie plaatst?
	\item Wat is opnieuw de link tussen stroom, lading en tijd? 
\end{enumerate}





\subsection*{64}
\underline{Methode}
\begin{enumerate}
	\item Bereken eerst de equivalente weerstand in beide situaties.
	\item Wat is de betekenis van emk?
	\item Wanneer geeft een lamp opnieuw het meeste licht? (Oefening 60)
\end{enumerate}













	% !TeX root = Stageportfolio.tex



\begin{landscape}
	
	\subsubsection{Les 7-9}
	\begin{tabularx}{1.56\textwidth}{|p{0.55\textwidth}|X|}\hline
		\textbf{Administratieve gegevens}\newline\newline
		Kevin Truyaert\newline\newline
		Universiteit\newline
		Handelsingenieur, 2de fase\newline
		\underline{ECTS-fiche}: De inhoud is terug te vinden op de ECTS fiche: \href{https://onderwijsaanbod.kuleuven.be/syllabi/n/D0W55AN.htm}{https://onderwijsaanbod.kuleuven.be/syllabi/n /D0W55AN.htm} \newline
		\underline{Lesonderwerp}: `DC netwerken met weerstanden' & \textbf{Doelstellingen}\newline\vspace{0.5cm}
		\underline{Punt op de ECTS-fiche}
		\vspace{-0.5cm}\newline  - DC netwerken, wetten van Kirchhoff, elektrische meettoestellen \newline - toepassing: elektrische veiligheid en elektrische huisinstallatie \newline
		\underline{Lesdoelen}\newline
		\vspace{-0.5cm}
		\begin{enumerate}[itemsep=0.08\baselineskip]
			\item De studenten kennen de wetten van Kirchhoff.
			\item De studenten kunnen de wetten van Kirchhoff conceptueel uitleggen.
			\item De studenten kunnen de wetten van Kirchhoff opstellen voor een open netwerk met bronnen en weerstanden.
			\item De studenten kunnen de wetten van Kirchhoff opstellen voor een gesloten netwerk met bronnen en weerstanden.
			\item De studenten kunnen interpreteren dat een voltmeter gebruiken zorgt voor een wijziging in de spanningsval over een component. 
		    \item De studenten kunnen in groep over de oefening discussiëren en samen oplossingsgericht werken.
		\end{enumerate} \\\hline
	\multicolumn{2}{c}{ }\\
	\multicolumn{2}{c}{ }\\
	\multicolumn{2}{c}{ }\\
	\multicolumn{2}{c}{ }\\
	\multicolumn{2}{c}{ }\\
	\multicolumn{2}{c}{ }\\
	\multicolumn{2}{c}{ }\\
	\multicolumn{2}{c}{ }\\
	\multicolumn{2}{c}{ }\\
	\multicolumn{2}{c}{ }\\
	\end{tabularx}


	\begin{tabularx}{1.56\textwidth}{|p{0.55\textwidth}|X|}
		\hline
		\multirow{2}{0.55\textwidth}{\textbf{Beginsituatie}\newline De studenten hebben de theorie rond de wetten van Kirchhoff drie weken voor de oefenzitting gezien. Hierdoor zullen ze al tijd gehad hebben om de theorie te bekijken. Rond deze tijd hebben de studenten echter meerdere deadlines voor andere vakken en een examen Frans. Hierdoor plaats ik geen voorbereidende oefening online, maar vraag ik hen om enkel de wetten van Kirchhoff nog eens goed te bekijken. \newline\newline Er zijn 28 studenten die deze sessie volgen, maar vorige sessie waren slechts 18 studenten aanwezig. \newline\newline Het lokaal kan 30 studenten plaatsen. Ik splits de groep in zeven tafels van vier personen. Er is een dubbel krijtbord ter beschikking en de mogelijkheid tot projectie. Wanneer er geprojecteerd wordt, hangt het projectiescherm grotendeels over beide borden.  }& \textbf{Acties}\newline  - Net zoals tijdens de eerste lessenreeks, wil ik de studenten in \GreenHighlight{groepjes van vier studenten}{5cm} aan de slag zetten. Als examenvraag stel ik namelijk een oefening op rond de wetten van Kirchhoff, die aansluit bij wat ze deze en volgende les te zien krijgen. Ik vind het van essentieel belang dat ze de wetten van Kirchhoff niet allen goed en veel kunnen oefenen, maar dat ze die ook conceptueel begrijpen. Bij de eerste lessenreeks merkte ik op dat ik op deze manier gerichtere feedback aan de studenten kon geven. Ik ervoer ook dat ze gemotiveerd waren om per twee `beter' te doen dan hun overburen, terwijl ze toch steevast elkaar hielpen wanneer de andere vast zaten. Ik wil hier opnieuw een steunende rol spelen tijdens hun leer- en ervaringsproces.  \newline\newline
		
		- Bij het begin van de les overloop ik samen met de studenten de wetten van Kirchhoff. Zij reiken mij de twee wetten aan, die ik op het bord neerschrijf. Verder noteer ik ook samen met hen een stappenplan om dit soort oefeningen op te lossen. Dit laat ik op het bord staan. Zo kunnen de studenten steeds makkelijk teruggrijpen naar de theorie. \newline\newline
		- Ik werk niet met projectie, maar noteer alles op het bord, omdat het projectiescherm voor zo goed als beide borden hangt. Hierdoor houd ik een tempo aan waarop de studenten makkelijker kunnen volgen, doordat ik alles zelf ook neerschrijf.  
		
		\\ \cline{2-2}
		  & \textbf{Bronnen}\begin{itemize}
		  	\item Dudal, D., Temmerman, E., Truyaert, K., Heymans, S. (2019). Slides conceptuele natuurkunde
		  	\item Dudal, D., Temmerman, E., Truyaert, K., Heymans, S. (2019). Oefeningenbundel conceptuele natuurkunde
		  	\item Giancoli, D. C. (2008). Physics for scientists and engineers. Pearson Education International.
		  \end{itemize}\\ \hline
	\end{tabularx}


\newpage

\begin{tabularx}{1.56\textwidth}{|p{1.5cm}|p{6cm}|X|p{3cm}|}
	\hline
	\textbf{Nr. lesdoel } & \textbf{Inhoud (timing)}  & \textbf{Organisatie } & \textbf{Media } \\ \hline
	1\newline 2 &\underline{Herhaling theorie (20 minuten)}\newline
	De theorie rond de wetten van Kirchhoff worden door de studenten aangereikt. Zij interpreteren ook wat de vergelijkingen voorstellen en delen dit met hun medestudenten. Hierna bouw ik samen met de studenten een stappenplan op om dit soort oefeningen aan te pakken. We bespreken ook nog kort even welke voorwaarden voldaan moeten zijn om een stroom te hebben (gesloten kring, geen condensatoren). 
	&  \underline{Onderwijsleergesprek}\newline 
	Ik start deze les met aan de studenten te vragen om mij de twee wetten van Kirchhoff uit te leggen. Ik noteer de wiskunde vertaling hiervan op bord. Ik probeer verschillende studenten aan het woord te laten.\newline Hierna stel ik samen met de studenten een stappenplan op om DC schakelingen te kunnen interpreteren. Ik vermoed dat de studenten dit zich niet meer goed zullen herinneren vanuit de theorieles. Daardoor zal ik zelf eerst een hint per stap geven of de stap(pen) zelf op het bord zetten, waarna ik telkens nog eens een student aan het woord laat om deze stap uit te leggen in eigen woorden. Hierna schets ik een kleine kring op het bord waar we dit klassikaal op toepassen.\newline
	Ik focus mij ook hier weer op het correct interpreteren van beide vergelijkingen. Dit is goed mogelijk door een vergelijking te maken waarbij de stroom een rij mensen of een rij auto's is en een spanningsverschil een helling. De eerste wet wordt dan  dat je bij ieder kruispunt slecht één richting kan kiezen waardoor het totaal aantal inkomende mensen/auto's hetzelfde moet zijn aan het totaal vertrekkende auto's. De tweede wet van Kirchhoff stelt voor dat je in iedere kring op hetzelfde niveau moet terugkomen: als je een kring doorlopen hebt, dan ben je terug op dezelfde hoogte.
	\newline 
	Hierna noteer ik de oefeningen op bord die gemaakt kunnen worden. Dit zijn oefeningen 66, 65, 69 en 71 in die volgorde. Ik verwacht dat de alle oefeningen door iedereen gemaakt kunnen worden.\newline Ik zal de nadruk tijdens deze les vooral leggen op het zelfstandig inoefenen van dit soort oefeningen. Na het stappenplan op het bord genoteerd te hebben en met een minimaal voorbeeld gelinkt te hebben, heb ik uit ervaring van de voorbije jaren gemerkt dat de studenten er geen meerwaarde aan hebben om nog eerst een extra oefening te maken. Daarom laat ik hen meteen aan de slag gaan met de oefeningenreeks.
	& Krijtbord (Bordschema in bijlage)
	\\ \hline
\end{tabularx}




\begin{tabularx}{1.56\textwidth}{|p{1.5cm}|p{6cm}|X|p{4cm}|}
	\hline
	\textbf{Nr. lesdoel } & \textbf{Inhoud (timing)}  & \textbf{Organisatie } & \textbf{Media } \\ \hline
	3\newline 4\newline 6 &\underline{Oefening 66 (30 minuten)}\newline
	Ik laat de studenten eerst zelf aan de slag gaan om per twee/vier aan deze oefening te beginnen. Ondanks dat ze in de theorieles al enkele DC netwerken overlopen hebben en nu een stappenplan nog eens expliciet opgesteld hebben, zullen ze nog niet volledig begrijpen hoe dit soort oefeningen aan te pakken. De voorbije jaren moest ik telkens expliciet minstens één oefening helemaal klassikaal begeleiden, nadat ze er zelf aan gewerkt hebben. Hier zal ik ook nu voor zorgen.
	&  \underline{Check-in duo / check-in quatro}\newline
	De eerste 15 minuten wil ik de studenten laten in hun groep per twee of per vier aan de slag laten gaan. Het is de eerste maal dat ze zelfstandig dit soort oefening maken, dus ik verwacht een trage start. Ondanks het stappenplan, zullen de studenten enkele vragen hebben. Daarom geef ik duidelijk mee dat ik de eerste vijf minuten niet zal rondlopen. Deze tijd gebruik ik om de schakeling op het nog vrije bord te tekenen.\newline
	Daarna loop ik rond en observeer ik de studenten, stel ik vragen of beantwoord die. Na een kwartier verwacht ik dat er weinig oplossingen en veel vragen uit de bus gekomen zijn. Daarom bespreek ik deze oefening nog eens klassikaal, door het stappenplan met hen te overlopen. Ik lees nog eens duidelijk de stap en vraag aan de studenten wat ik moet doen.\newline
	Ik verwacht dat de studenten niet allemaal alle inzichten hebben, waardoor ik dit stappenplan klassikaal `bevraag'. Zo zullen niet alle studenten er bij stil gestaan hebben dat er geen stroom doorheen de open takken stromen, waardoor ze een vergelijking minder nodig hebben. Ik verwacht dan ook dat de studenten hier vragen over zullen hebben tijdens het oplossen van de opdracht. Hiervoor reken ik in totaal ook een kwartier uit.
	& Bord\newline Oefeningenbundel + cursuspapier
	\\ \hline
\end{tabularx}


\begin{tabularx}{1.56\textwidth}{|p{1.5cm}|p{6cm}|X|p{4cm}|}
	\hline
	\textbf{Nr. lesdoel } & \textbf{Inhoud (timing)}  & \textbf{Organisatie } & \textbf{Media } \\ \hline
	4\newline 5\newline 6 &\underline{Oefening 65 (25 minuten)}\newline
	Deze oefening toont aan dat wanneer je de spanningsval wil meten over een verbruiker, de interne weerstand heel groot moet zijn om de reële situatie op te meten. Want wanneer je iets meet, verander je steeds de uitkomst. Bij deze oefening heeft de voltmeter geen hoge interne weerstand, waardoor het spanningsverschil gemeten over beide componenten lager ligt dan het spanningsverschil van de bron.
	& 	\underline{Check-in duo / check-in quatro}\newline
	De oefening zelf zal deze keer beter gaan, in vergelijking met de vorige. De studenten hebben nu het stappenplan op drie verschillende manieren ervaren: bij een minimaal voorbeeld, zelf geprobeerd en klassikaal. Ik verwacht deze keer de grootste problemen bij het verwerken van de voltmeter in de schakeling. Verder zie ik hen de oefening goed oplossen.\newline
	Aan de studenten die vlotter weg zijn met deze oefening, zal ik extra vragen hoe het komt dat de som van de twee opgemeten potentiaalverschillen niet die van de bron is. Dit zou fysisch moeten kloppen, aangezien je anders geen behoud van energie zou hebben. Deze vraag wil ik vooral conceptueel behandelen: doordat je een weerstand in parallel plaatst, verandert de equivalente weerstand en verandert het systeem. Om dit zoveel mogelijk te beperken, zorg je dat je voltmeter een hoge interne weerstand heeft. Deze vraag stel ik op het eind van dit lesonderdeel klassikaal aan de overige studenten, omdat ik het concept `Meten is weten' wil verduidelijken naar `Meten is veranderen en interpreteren'.
	& Oefeningenbundel + cursuspapier
	\\ \hline
\end{tabularx}






\begin{tabularx}{1.56\textwidth}{|p{1.5cm}|p{6cm}|X|p{4cm}|}
	\hline
	\textbf{Nr. lesdoel } & \textbf{Inhoud (timing)}  & \textbf{Organisatie } & \textbf{Media } \\ \hline
		&\underline{Pauze}\newline
	
	
	&    De studenten krijgen 15 minuten pauze en mogen het lokaal verlaten. Op deze manier kunnen ze het laatste uur weer met volle aandacht werken. 
	Wanneer de eerste studenten het lokaal terug binnen sijpelen, sla ik een praatje met hen, waarbij ik niet over de leerstof begin. 
	& 
	\\ \hline
\end{tabularx}


\begin{tabularx}{1.56\textwidth}{|p{1.5cm}|p{6cm}|X|p{4cm}|}
	\hline
	\textbf{Nr. lesdoel } & \textbf{Inhoud (timing)}  & \textbf{Organisatie } & \textbf{Media } \\ \hline
	4\newline 6&\underline{Oefeningen  69 en 71 (55 minuten)}\newline
	Deze oefeningen behandelen iets complexere schakelingen met weerstanden. De studenten leren tijdens deze lesfase verder de wetten van Kirchhoff inoefenen op iet complexere systemen.
 &  \underline{Check-in duo / check-in quatro}\newline
 De studenten lossen deze complexere oefeningen in overleg met elkaar op. Ik loop rond om vragen te beantwoorden en om in te pikken op hun reeds gevonden oplossing, zowel wanneer die correct of foutief blijkt te zijn. Ik wil hier vooral de nadruk blijven leggen op het conceptuele van de fysische concepten, naast het correct oplossen van de vraag. Hoe de studenten de richtingen van hun stroom en potentiaalverschillen definiëren maakt enkel uit over de bronnen (indien er meerdere aanwezig zijn), maar niet bij verbruikers (dan komen ze gewoon negatief uit). 
	& Oefeningenbundel + cursuspapier
	\\ \hline
\end{tabularx}



\begin{tabularx}{1.56\textwidth}{|p{1.5cm}|p{6cm}|X|p{4cm}|}
	\hline
	\textbf{Nr. lesdoel } & \textbf{Inhoud (timing)}  & \textbf{Organisatie } & \textbf{Media } \\ \hline
	&\underline{Afsluiten (5 minuten)}\newline 
	&  \underline{Afsluiten (5 minuten)}\newline
	Ik herhaal nog even kort wat er van de studenten verwacht werd tijdens deze les en wat ze bijgeleerd hebben. Ik zeg ook dat volgende oefenzitting de laatste is en dat ze, indien gewenst, dan vragen kunnen stellen.
	& 
	\\ \hline
\end{tabularx}


	
	
	
	
	
	
\end{landscape}


\subsection*{Bijlage 4.1.3.1: bordschema theorie}
\begin{center}
	\includegraphics[width=0.9\textwidth]{Bord3}
\end{center}
\newpage


\includepdf[scale = 0.8,pages = 25,pagecommand=\subsection*{Bijlage 4.1.3.2: opgeloste oefeningen}]{Observaties_OpgelosteOef}
\includepdf[scale = 0.8,pages =26-29,pagecommand=]{Observaties_OpgelosteOef}





	% !TeX root = Stageportfolio.tex



\begin{landscape}	
	\subsubsection{Les 9-10}
	\begin{tabularx}{1.56\textwidth}{|p{0.55\textwidth}|X|}\hline
		\textbf{Administratieve gegevens}\newline\newline
		Kevin Truyaert\newline\newline
		Universiteit\newline
		Handelsingenieur, 2de fase\newline
		\underline{ECTS-fiche}: De inhoud is terug te vinden op de ECTS fiche: \href{https://onderwijsaanbod.kuleuven.be/syllabi/n/D0W55AN.htm}{https://onderwijsaanbod.kuleuven.be/syllabi/n /D0W55AN.htm} \newline
		\underline{Lesonderwerp}: `DC netwerken met weerstanden en condensatoren' & \textbf{Doelstellingen}\newline\vspace{0.5cm}
		\underline{Punt op de ECTS-fiche}
		\vspace{-0.5cm}\newline  - DC netwerken, wetten van Kirchhoff, elektrische meettoestellen \newline - toepassing: elektrische veiligheid en elektrische huisinstallatie \newline
		\underline{Lesdoelen}\newline
		\vspace{-0.5cm}
		\begin{enumerate}[itemsep=0.08\baselineskip]
			\item De studenten kunnen de wetten van Kirchhoff wiskundig formuleren.
			\item De studenten kunnen de werking en de invloed van een condensator in een elektrische schakeling conceptueel uitleggen.
			\item De studenten kunnen de wetten van Kirchhoff opstellen voor een gesloten netwerk met bronnen en condensatoren.
			\item De studenten kunnen de equivalente capaciteit van condensatoren in serie en parallel berekenen. (Herhaling lesdoel 2, les 4-5)
			\item De studenten kunnen de wetten van Kirchhoff opstellen voor een gesloten netwerk met bronnen, weerstanden en condensatoren.
			\item De studenten kunnen in duo/trio over de oefening discussiëren en samen oplossingsgericht werken.
		%	\item De studenten kunnen de wetten van Kirchhoff individueel gebruiken om een elektrische schakeling uit te werken.
		\end{enumerate} \\\hline
		\multicolumn{2}{c}{ }\\
		\multicolumn{2}{c}{ }\\
		\multicolumn{2}{c}{ }\\
		\multicolumn{2}{c}{ }\\
		\multicolumn{2}{c}{ }\\
		\multicolumn{2}{c}{ }\\
		\multicolumn{2}{c}{ }\\
		\multicolumn{2}{c}{ }\\
		\multicolumn{2}{c}{ }\\
	\end{tabularx}
	
	
	\begin{tabularx}{1.56\textwidth}{|p{0.55\textwidth}|X|}
		\hline
		\multirow{2}{0.55\textwidth}{\textbf{Beginsituatie}\newline De studenten hebben vorige week een oefenzitting rond de wetten van Kirchhoff gehad. Tijdens deze les hebben ze de wiskundige uitdrukking herhaald en hebben ze de wetten van Kirchhoff toegepast in schakelingen met bronnen en weerstanden. Rond deze tijd hebben de studenten echter meerdere deadlines voor andere vakken en een examen Frans. Hierdoor plaats ik geen voorbereidende oefening online, maar vraag ik hen om enkel eigenschappen van condensatoren nog eens goed te bekijken. \newline\newline Er zijn 28 studenten die deze sessie volgen, maar vorige sessie waren 22 studenten aanwezig. \newline\newline Het lokaal kan 30 studenten plaatsen. Ik laat de banken in drie rijen van tien staan. Er is een dubbel krijtbord ter beschikking en de mogelijkheid tot projectie. Wanneer er geprojecteerd wordt, hangt het projectiescherm grotendeels over beide borden.  }& \textbf{Acties}\newline  -  Als examenvraag stel ik een oefening op rond de wetten van Kirchhoff, die aansluit bij wat ze deze en vorige les gezien hebben. Ik vind het van essentieel belang dat ze de wetten van Kirchhoff niet allen goed en veel kunnen oefenen, maar dat ze die ook conceptueel begrijpen. Vorige lessenreeks zette ik er op in dat zoveel mogelijk studenten de basis van het toepassen van de wetten van Kirchhoff onder de knie hadden. Bij deze les wil ik er voor zorgen dat de studenten individueler ook aan de slag kunnen gaan bij deze soort oefeningen. Ze kunnen wel nog steeds ten rade bij hun buren of bij mij wanneer er problemen opduiken. \newline\newline
		
		- Bij het begin van de les overloop ik samen met de studenten de wetten van Kirchhoff. Zij reiken mij de twee wetten aan, die ik op het bord neerschrijf. Verder noteer ik ook samen met hen een stappenplan om dit soort oefeningen op te lossen. Dit laat ik op het bord staan. Zo kunnen de studenten steeds makkelijk teruggrijpen naar de theorie. \newline\newline
		- Ik werk niet met projectie, maar noteer alles op het bord, omdat het projectiescherm voor zo goed als beide borden hangt. Hierdoor houd ik een tempo aan waarop de studenten makkelijker kunnen volgen, doordat ik alles zelf ook neerschrijf.  
		
		\\ \cline{2-2}
		& \textbf{Bronnen}\begin{itemize}
			\item Dudal, D., Temmerman, E., Truyaert, K., Heymans, S. (2019). Slides conceptuele natuurkunde
			\item Dudal, D., Temmerman, E., Truyaert, K., Heymans, S. (2019). Oefeningenbundel conceptuele natuurkunde
			\item Giancoli, D. C. (2008). Physics for scientists and engineers. Pearson Education International.
		\end{itemize}\\ \hline
	\end{tabularx}
\newpage
	

\begin{tabularx}{1.56\textwidth}{|p{1.5cm}|p{6cm}|X|p{3cm}|}
	\hline
	\textbf{Nr. lesdoel } & \textbf{Inhoud (timing)}  & \textbf{Organisatie } & \textbf{Media } \\ \hline
	1\newline 2 &\underline{Herhaling theorie (15 minuten)}\newline
	Ik vraag net zoals vorige les opnieuw aan de studenten om de wetten van Kirchhoff zowel in hun eigen woorden als wiskundig te formuleren. Verder bespreken we nog de eigenschappen van een condensator en wat dit betekent wanneer ze in een schakeling geplaatst worden. Ik leg er de nadruk op dat de studenten in oefeningen enkel maar schakelingen met condensatoren in stationaire toestand moeten kunnen oplossen. Dit wil zeggen dat de condensator ofwel volledig opgeladen ofwel volledig ontladen is.
	&  \underline{Onderwijsleergesprek}\newline 
	Ik start deze les met aan de studenten te vragen om mij de twee wetten van Kirchhoff nog eens opnieuw uit te leggen, conceptueel en de wiskundige vertaling ervan. Ik probeer verschillende studenten aan het woord te laten. Daarna laat ik de studenten nog eens het stappenplan herhalen en schrijf ik dit ook op het bord ter referentie voor de oefeningen. \newline Hierna overloop ik met de studenten de eigenschappen van een condensator in stationaire toestand: geen stroom meer door een tak met een condensator, betekenis van het potentiaalverschil bij een opgeladen condensator, het ontladen van een condensator (conceptueel). Hierna schets ik een kleine kring (bron-condensator) op het bord waar we dit klassikaal op toepassen.\newline 
	Hierna noteer ik de oefeningen op bord die gemaakt kunnen worden. Dit zijn oefeningen 68, 70 en 67 in die volgorde. Ik verwacht dat de eerste drie oefeningen door iedereen gemaakt kunnen worden en de laatste door de betere studenten.\newline Ik zal de nadruk tijdens deze les vooral leggen op het zelfstandig inoefenen van dit soort oefeningen. 
	& Krijtbord 
	\\ \hline
\end{tabularx}

\begin{tabularx}{1.56\textwidth}{|p{1.5cm}|p{6cm}|X|p{4cm}|}
	\hline
	\textbf{Nr. lesdoel } & \textbf{Inhoud (timing)}  & \textbf{Organisatie } & \textbf{Media } \\ \hline
	3 \newline 4\newline 6	&\underline{Oefening 68 (20 minuten)}\newline
	Tijdens deze oefening ervaren studenten om spanningsverschillen over condensatoren te berekenen. Hiervoor zullen ze eerst een equivalente capaciteit voor de condensatoren moeten berekenen. Op dit moment evalueer ik of lesdoel 2 van les 4-5 wel degelijk bereikt is (hier lesdoel 4). Ik verwacht geen problemen met de evaluatie van vorig lesdoel. 
	
	&   \underline{Oplossingensleutel}\newline
		De studenten kunnen hun antwoord controleren aan de hand van een controlesleutel. Deze zijn zo opgesteld dat alle stappen benoemd zijn, maar niet uitgewerkt. Bij vragen kunnen de studenten mij raadplegen.\newline
		De grootste problemen zullen ontstaan bij het berekenen van het potentiaalverschil over iedere condensator. De studenten zullen niet meteen inzien dat de lading die per condensator opgeslagen is dezelfde moet zijn. Dit zal ik beantwoorden door hen te vragen naar de werking van een opladende condensator in een circuit.
	&  Oplossingensleutel (Bijlage)\newline Oefeningenbundel + cursuspapier
	\\ \hline
\end{tabularx}




\begin{tabularx}{1.56\textwidth}{|p{1.5cm}|p{6cm}|X|p{4cm}|}
	\hline
	\textbf{Nr. lesdoel } & \textbf{Inhoud (timing)}  & \textbf{Organisatie } & \textbf{Media } \\ \hline
	5\newline 6	&\underline{Oefening 70 (30 minuten)}\newline
	Deze schakeling bevat zowel weerstanden als condensatoren als verbruiker. Deze oefening is de eerste gecombineerde oefening die de studenten krijgen.  
	
	&   \underline{Oplossingensleutel}\newline
	De studenten kunnen hun antwoord controleren aan de hand van een controlesleutel. Deze zijn zo opgesteld dat alle stappen benoemd zijn, maar niet uitgewerkt. Bij vragen kunnen de studenten mij raadplegen.\newline
	De grootste problemen bij deze oefening zullen ontstaan omdat er geen rekening gehouden wordt met een bepaalde eigenschap van de condensator. Wanneer deze volledig opgeladen is, dan is de stroom doorheen die tak gelijk aan $0$~A. Sommige studenten zullen dit niet meteen inzien, waardoor er een vergelijking tekort is. Bij deze vraag stel ik aan de studenten de vraag om eens na te denken over de eigenschappen van condensatoren. Een andere vraag kan zijn waarom de weerstand van R$_3$ niet gekend is. Ook voor die vraag geldt hetzelfde antwoord.
	&  Oplossingensleutel (Bijlage)\newline Oefeningenbundel + cursuspapier
	\\ \hline
\end{tabularx}




\begin{tabularx}{1.56\textwidth}{|p{1.5cm}|p{6cm}|X|p{4cm}|}
	\hline
	\textbf{Nr. lesdoel } & \textbf{Inhoud (timing)}  & \textbf{Organisatie } & \textbf{Media } \\ \hline
	&\underline{Pauze}\newline
	
	
	&    De studenten krijgen 15 minuten pauze en mogen het lokaal verlaten. \newline
	Wanneer de eerste studenten het lokaal terug binnen sijpelen, sla ik een praatje met hen, waarbij ik niet over de leerstof begin. 
	& 
	\\ \hline
\end{tabularx}



\begin{tabularx}{1.56\textwidth}{|p{1.5cm}|p{6cm}|X|p{4cm}|}
	\hline
	\textbf{Nr. lesdoel } & \textbf{Inhoud (timing)}  & \textbf{Organisatie } & \textbf{Media } \\ \hline
	 5\newline 6 & \underline{Oefening 67 (30 minuten)}\newline Deze schakeling bevat zowel weerstanden als condensatoren als verbruiker. Deze schakeling is complexer dan de vorige.
	 
	 
	 &   \underline{Oplossingensleutel}\newline
	 De studenten kunnen hun antwoord controleren aan de hand van een controlesleutel. Deze zijn zo opgesteld dat alle stappen benoemd zijn, maar niet uitgewerkt. Bij vragen kunnen de studenten mij raadplegen.\newline
	 Opnieuw hier geldt dat wanneer de condensator volledig opgeladen is, de stroom doorheen die tak gelijk aan $0$~A is. Sommige studenten zullen dit opnieuw niet meteen inzien, waardoor er opnieuw een vergelijking tekort is. Bij deze vraag stel ik dan ook opnieuw dezelfde vraag aan de studenten.  Mogelijks ontstaan er ook problemen bij het sluiten van de schakelaar waardoor er een extra lus ontstaat.
	 &  Oplossingensleutel (Bijlage)\newline Oefeningenbundel + cursuspapier
	 \\ \hline
\end{tabularx}



\begin{tabularx}{1.56\textwidth}{|p{1.5cm}|p{6cm}|X|p{4cm}|}
	\hline
	\textbf{Nr. lesdoel } & \textbf{Inhoud (timing)}  & \textbf{Organisatie } & \textbf{Media } \\ \hline
	5\newline 6	&\underline{Vragen + extra oefening (1 uur)}\newline 
	Ik projecteer eerst een extra oefening (oude examenvraag) die over een DC netwerk gaat. Hierna kunnen de studenten mij individueel tijdens deze lesfase vragen stellen. Ondertussen kunnen de overige studenten de oude examenoefening proberen op te lossen.
	
	&  Ik projecteer een oude examen oefening met de numerieke oplossing op het projectiescherm. De studenten kunnen deze oplossen. Ondertussen is er mogelijkheid tot vragen in verband met alle delen van de cursus. Ik vraag de studenten om mij geen vragen te stellen over de examenoefening gedurende de eerste 40 minuten, tenzij er geen vragen over de cursus meer zouden komen. Ik meld hen ook dat ik enkel meer mondelinge feedback zal geven over deze oefening. 
	& Projectiescherm \newline cursuspapier 
	\\ \hline
\end{tabularx}


	
\begin{tabularx}{1.56\textwidth}{|p{1.5cm}|p{6cm}|X|p{4cm}|}
	\hline
	\textbf{Nr. lesdoel } & \textbf{Inhoud (timing)}  & \textbf{Organisatie } & \textbf{Media } \\ \hline
	&\underline{Afsluiten (5 minuten)}\newline 
	&  \underline{Afsluiten (5 minuten)}\newline
	Ik herhaal nog even kort wat er van de studenten verwacht werd tijdens deze les en wat ze bijgeleerd hebben. Ik herhaal nog eens de info in verband met de oefeningen voor het examen, het deel waarvoor ik verantwoordelijk ben. Ik herhaal hen ook nog eens de afspraken rond het stellen van vragen voor het examen en dat er tijdens de kerstperiode zowel niet door de prof als door mij geantwoord zal worden.  Ik wens de studenten een fijn oudjaar en veel succes bij het studeren.
	& 
	\\ \hline
\end{tabularx}
	
	
	
	
	
	
	
\end{landscape}

\subsection*{Bijlage: oplossingsleutels}
De oplossingssleutels zijn hieronder bijgevoegd. Deze zijn telkens zo opgesteld dat de studenten niet de oplossing uitgewerkt krijgen, maar dat er enkele gerichte vragen of hints zijn die hen op de weg kunnen helpen wanneer ze vast zitten. Wanneer ze de oefening correct hebben, kunnen ze stilstaan bij die vragen en controleren of ze de oefening ook begrijpen.\newline
De oplossingssleutels zijn hier samen gevoegd. In realiteit werden meerdere sleutels per blad afgedrukt en daarna uitgeknipt.

\subsection*{68}
\underline{Methode}
\begin{enumerate}
\item Hoe bereken je de equivalente capaciteit opnieuw? (Oefening 59) 
\item Wat is er indentiek bij  volledig opgeladen condensatoren in serie?
\item Je weet nu de lading en de capaciteit. Hoe haal je hieruit het potentiaalverschil?
\end{enumerate}

\subsection*{70}
\underline{Methode}
\begin{enumerate}
\item Waarom is de numerieke waarde van R$_3$ niet gekend? Dit is bewust.
\item Wat gebeurt er met een tak die een volledig opgeladen condensator bevat?
\end{enumerate}

\underline{Extra vragen}
\begin{enumerate}
\item Heeft de tak met de condensator nog een invloed op de schakeling eens die volledig opgeladen is? Verklaar.
\end{enumerate}




\subsection*{67}
\underline{Methode}
\begin{enumerate}
	\item Wat gebeurt er met een tak die een volledig opgeladen condensator bevat?
	\item Gebruik opnieuw de hoogte als equivalent voor spanning. Welke verbruikers leveren hier een spanningsverschil? Welke niet? Gebruik dit om conceptueel na te gaan welk punt, b of c, een hogere potentiaal heeft.
	\item Wanneer de schakelaar gesloten is, hebben de condensatoren een \textbf{nieuw} evenwicht gevonden! Er zal dus een ladingsverschil zijn met de `open' situatie.
\end{enumerate}


\underline{Extra vragen}
\begin{enumerate}
\item Is dit ladingsverschil steeds gelijk aan elkaar? Waarom is dit hier zo?
\item Wanneer de schakelaar gesloten is, wat is nu het potentiaalverschil tussen b en c? Verklaar.
\item Beschrijf conceptueel wat er in de eerste tijdseenheden na het sluiten van de schakelaar  gebeurt, na lange tijd open te zijn geweest.
\end{enumerate}



\subsection*{Examenoefening 2018-2019}
	
Onderstaande figuur toont een Kirchhoff netwerk bestaande uit vier weerstanden, een spanningsbron, een condensator en een schakelaar. De waardes van de verschillende componenten zijn: $\varepsilon_1=15$~V, $C=4$~$\mu F$ en $R=5~\Omega$. De waarde van de weerstand $\tilde{R}$ is ongekend. 

In de situatie waar de schakelaar al lange tijd \textbf{gesloten} is, weet je dat de stroom in de tak met $\varepsilon_1$ gelijk is aan $I_1 = 1A$. De stroom $I_1$ loopt zoals op de figuur aangegeven.

\begin{enumerate}
	\item Waar ligt de hoogste potentiaal bij de condensator, in punt a of b?
	\item Bereken de lading op de condensator wanneer de schakelaar lange tijd gesloten is.
	\item Bereken het vermogen dat over $\tilde{R}$ verloren gaat wanneer de schakelaar gesloten is.
	\item De condensator is volledig opgeladen op het moment dat de schakelaar \textbf{open} gezet wordt. De condensator zal beginnen te ontladen. In de theorie werd gezien dat de lading op de condensator in een RC-keten exponentieel daalt als: \[Q=Q_0e^{-\frac{t}{R_{eq}C}},\] waarbij $R_{eq}$ de equivalente weerstand van het systeem is en $Q_0$ is de beginlading op de condensator. Bepaal de stroom in de tak van de condensator in het resterende systeem in functie van de tijd.
\end{enumerate}
Formuleer telkens  een duidelijke antwoordzin. 

\def\x{6}
\def\y{6}
% Size of the bridge
\def\dx{3}
\def\dy{3}

\begin{figure}[h]
	\centering
	\begin{circuitikz}[american resistors, european voltages]
		%	% Voltage source
		\draw (0,\y) to [battery1, l_=$\varepsilon_1$,i=$I_1$,*-*]
		(0, 0) to [cspst] (\dx,0) to (\x,0) to (\x,1) to ({\x/6*4},1) to [R,l_=$3R$,*-*] ({\x/6*4},{\y/2}); \draw  ({\x/6*4},{\y/2}) to [R=$R$,*-*] ({\x/6*4},5) to (\x,5) to (\x,\y) to (0,\y);
		\draw (\x,5) to ({\x/6*8},5) to [R, l_=$\tilde{R}$,*-*] ({\x/3*4}, {\y/2});
		\draw ({\x/3*4}, {\y/2}) to [R,l_=$2R$,*-*] ({\x/3*4}, 1) to (\x,1);
		\draw ({\x/6*4},{\y/2}) node[left]{a} to [C,l_=$C$,*-] ({\x/3*4},{\y/2}) node[right]{b};
	\end{circuitikz}
	\caption*{Een Kirchoff netwerk met vier weerstanden, een bron, een condensator en een schakelaar.}
	\label{Fig:KichoffNetwerk1V4R}
\end{figure}

\underline{\textbf{Antwoord:}}
\begin{enumerate}
	\item punt a ligt 8.75~V hoger
	\item 35$\mu$C
	\item 3.125W 
	\item $I = -0.5091\exp(-14545t)~A$
\end{enumerate}


	
%	
	
% !TeX root = Stageportfolio.tex



\begin{landscape}
	\subsection{Lessen aan VISO}
	De bijlagen per lesblok (e.g. bordschema, uitgeschreven oplossingen, labo's \ldots) zijn na elk lesblok terug te vinden. 
	\subsubsection{Les 1-2}
	\begin{tabularx}{1.56\textwidth}{|p{0.35\textwidth}|X|}\hline
		\textbf{Administratieve gegevens}\newline\newline
		Kevin Truyaert\newline\newline
		technisch secundair onderwijs\newline
		3e graad, 1ste jaar, Techniek-Wetenschappen\newline
		VVKSO: \href{http://ond.vvkso-ict.com/leerplannen/doc/Toegepaste\%20fysica-2014-041.pdf}{http://ond.vvkso-ict.com/leerplannen /doc/Toegepaste\%20fysica-2014-041.pdf} \newline
		\underline{Lesonderwerp}:\newline `Labo M4: De stroombalans' & \textbf{Doelstellingen}
		\begin{itemize}[itemsep=0.08\baselineskip]
			\item B24: De richting, de zin en de grootte van de Lorentzkracht op een rechte stroomvoerende geleider aangeven en hiermee de magnetische veldsterkte omschrijven. 
			\item AD4 Reflecteren: Over een waarnemingsopdracht/experiment/onderzoek en het resultaat reflecteren.
			\item AD5 Rapporteren: Over een waarnemingsopdracht/experiment/onderzoek en het resultaat rapporteren.
			\item AD10 Meettoestellen en meetnauwkeurigheid: De gepaste toestellen kiezen voor het meten van de behandelde grootheden en de meetresultaten correct aflezen en noteren.
			\item AD 12 Grafieken: Meetresultaten grafisch voorstellen in een diagram en deze interpreteren.
		\end{itemize}
		\underline{Lesdoelen}\newline
		\vspace{-0.75cm}
		\begin{enumerate}[itemsep=0.08\baselineskip]
			\item De leerlingen kunnen de Lorentzkracht toepassen op de specifieke situatie van de stroombalans.
			\item De leerlingen werken samen om een wetenschappelijke opstelling te kunnen bouwen.
			\item De leerlingen voeren de beschreven handelingen uit om tot resultaten te komen.
			\item De leerlingen meten de resultaten nauwkeurig op.
			\item De leerlingen reflecteren over de resultaten.
			\item De leerlingen rapporteren over hun resultaten.
			\item De leerlingen houden bij hun berekeningen rekening met de nauwkeurigheid.
			\item De leerlingen stellen de meetresultaten grafisch voor.
		\end{enumerate} \\\hline
	\end{tabularx}


	\begin{tabularx}{1.56\textwidth}{|p{0.55\textwidth}|X|}
		\hline
		\multirow{2}{0.55\textwidth}{\textbf{Beginsituatie}\newline  
		Er zijn acht leerlingen binnen 5TW. Er heerst een algemene klassfeer. De leerlingen hebben al theorie gekregen  rond en oefeningen gemaakt op de magnetische krachtwerking. \newline\newline Bij dit labo beschikt de school over drie opstellingen van de stroombalans, wat in evenveel groepjes resulteert. De mentor voorziet de verdeling van de groepjes, aangezien dit in een rotatiesysteem verwerkt zit voor andere labo's.\newline\newline Het lokaal is het fysicalokaal waar mogelijkheid is om elektriciteit uit  pilaren te halen die centraal per rij staan. Er zijn voldoende voorzieningen voor het experimentele materiaal. Er is ook een krijtbord en een beamer aanwezig. \newline\newline Dit is mijn eerste les in het tso en is meteen een labo. Als leerkracht ben ik hierdoor wat nerveus (eerste les) maar zeer benieuwd naar het kunnen van de leerlingen, iets wat ik meteen kan testen vanaf mijn eerste les, vanwege het onderwerp. } & \textbf{Acties}\newline  
		
		\\ \cline{2-2}
		  & \textbf{Bronnen}\begin{itemize}
		  	\item Schramme, S. (2018) De stroombalans, labo magnetisme 4
		  	\item Frederiksen (2014), Current Balance 4565.00
		  \end{itemize}\\ \hline
	\end{tabularx}


\newpage
	
%	\begin{tabularx}{1.56\textwidth}{|p{1.5cm}|p{6cm}|X|p{4cm}|}
%		\hline
%		\textbf{Nr. lesdoel } & \textbf{Inhoud (timing)}  & \textbf{Organisatie } & \textbf{Media } \\ \hline
%		&\underline{Herhaling theorie (15 minuten)}\newline
%		De algemene student heeft op dit moment weinig voeling met de te bespreken leerstof, want het is de eerste oefenzitting over dit onderwerp. Dit heb ik zowel de voorbije jaren tijdens mijn oefenzittingen gemerkt als bij de geobserveerde theorielessen. Daarom breng ik de theorie waarop de studenten oefeningen zullen maken nog eens zelf aan bord. Deze behandelt vijf topics: lading, elektrisch veld, elektrische kracht, flux en de elektrische wet van Gauss. Vooral deze laatste vormt een struikelblok voor de studenten. Het is mijn bedoeling om die op verschillende manieren nog eens uitgelegd te hebben.
%		&  \underline{Doceren}\newline 
%		Ik bouw de te gebruiken theorie op door te starten vanuit de eigenschappen van een lading, dat die een elektrisch veld genereren en dat een elektrisch veld op een andere lading inwerkt door middel van de elektrische kracht. Daarna herhaal ik nog kort eens wat elektrische flux is, om dat tot het grootste probleempunt te komen: de elektrische wet van Gauss.\newline
%		Ik wil vooral heel hard benadrukken wat deze wet zegt, door de aparte onderdelen uit te leggen en conceptueel voor te stellen. Ik doe dit vanuit verschillende insteken om zoveel mogelijk studenten mee te hebben. 
%		\newline 
%		Hierna noteer ik de oefeningen op bord die gemaakt kunnen worden. Dit zijn oefeningen 51 t.e.m. 55, 57 en 56, in die volgorde. Ik verwacht dat deze oefeningen door de betere studenten allemaal gemaakt kunnen worden. Ik verwacht dat de meesten zullen vast komen te zitten bij oefening 54 en 55. Deze gaan namelijk over de elektrische wet van Gauss. Oefeningen 57 en 56 kunnen tijdens de volgende les eventueel ook nog aan bod komen. 
%		& Krijtbord (Bordschema, zie bijlage)
%		\\ \hline
%	\end{tabularx}
%
%	
%
%\begin{tabularx}{1.56\textwidth}{|p{1.5cm}|p{6cm}|X|p{4cm}|}
%	\hline
%	\textbf{Nr. lesdoel } & \textbf{Inhoud (timing)}  & \textbf{Organisatie } & \textbf{Media } \\ \hline
%	&\underline{Afsluiten (5 minuten)}\newline 
%	&  \underline{Afsluiten}\newline
%	Ik herhaal nog even kort wat er van de studenten verwacht werd tijdens deze les en wat ze bijgeleerd hebben. Ik zeg ook wat het onderwerp van volgende les is en vraag aan de studenten om de theorie nog even te herhalen.
%	& 
%	\\ \hline
%\end{tabularx}
%	
%	
	
	
\end{landscape}


%\subsection*{Bijlage 1.1: voorbereiding theorie}
%
%\subsection*{Bijlage 1.2: bordschema theorie}
%\begin{center}
%	\includegraphics[width=0.9\textwidth]{Bord1a}
%\includegraphics[width=0.9\textwidth]{Bord1b}
%\end{center}
%\newpage
%
%
%\includepdf[scale = 0.8,pages = 17,pagecommand=\subsection*{Bijlage 1.3: opgeloste oefeningen}]{Observaties_OpgelosteOef}
%\includepdf[scale = 0.8,pages =18-20,pagecommand=]{Observaties_OpgelosteOef}
%
%
%
%\includepdf[scale = 0.95,pages = 1,pagecommand=\subsection*{Bijlage 1.4: oefeningenbundel elektromagnetisme}]{OefeningenBundel}
%\includepdf[scale = 0.95,pages =2-,pagecommand=]{OefeningenBundel}
% !TeX root = Stageportfolio.tex



\begin{landscape}
	\subsubsection{Les 13}
	\begin{tabularx}{1.56\textwidth}{|p{0.35\textwidth}|X|}\hline
		\textbf{Administratieve gegevens}\newline\newline
		Kevin Truyaert\newline\newline
		technisch secundair onderwijs\newline
		3e graad, 1ste jaar, Techniek-Wetenschappen\newline
		VVKSO: \href{http://ond.vvkso-ict.com/leerplannen/doc/Toegepaste\%20fysica-2014-041.pdf}{http://ond.vvkso-ict.com/leerplannen /doc/Toegepaste\%20fysica-2014-041.pdf} \newline
		\underline{Lesonderwerp}:\newline Afwerken `Labo M4: De stroombalans' & \textbf{Doelstellingen}
		\begin{itemize}[itemsep=0.08\baselineskip]
			\item B24: De richting, de zin en de grootte van de Lorentzkracht op een rechte stroomvoerende geleider aangeven en hiermee de magnetische veldsterkte omschrijven. 
			\item AD4 Reflecteren: Over een waarnemingsopdracht/experiment/onderzoek en het resultaat reflecteren.
			\item AD5 Rapporteren: Over een waarnemingsopdracht/experiment/onderzoek en het resultaat rapporteren.
			\item AD10 Meettoestellen en meetnauwkeurigheid: De gepaste toestellen kiezen voor het meten van de behandelde grootheden en de meetresultaten correct aflezen en noteren.
			\item AD 12 Grafieken: Meetresultaten grafisch voorstellen in een diagram en deze interpreteren.
		\end{itemize}
		\underline{Lesdoelen}\newline
		\vspace{-0.75cm}
		\begin{enumerate}[itemsep=0.08\baselineskip]
			\item De leerlingen kunnen de Lorentzkracht toepassen op de specifieke situatie van de stroombalans.
			\item De leerlingen reflecteren over de resultaten.
			\item De leerlingen rapporteren over hun resultaten.
			\item De leerlingen werken samen bij het opbouwen van hun verslag.
			\item De leerlingen houden bij hun berekeningen rekening met de nauwkeurigheid.
			\item De leerlingen stellen de meetresultaten grafisch voor.
			\item De leerlingen berekenen de magnetische veldsterkte van de magneet.
			\item De leerlingen begrijpen conceptueel wat de magnetische flux is.
			\item De leerlingen kunnen de drie factoren die de magnetische flux bepalen.
		\end{enumerate} \\\hline
	\end{tabularx}


	\begin{tabularx}{1.56\textwidth}{|p{0.55\textwidth}|X|}
		\hline
		\multirow{2}{0.55\textwidth}{\textbf{Beginsituatie}\newline  
		Er zijn acht leerlingen binnen 5TW. Er heerst een algemene klassfeer. De leerlingen hebben al theorie gekregen  rond en oefeningen gemaakt op de magnetische krachtwerking. \newline\newline De leerlingen hebben de week voor de krokusvakantie aan dit labo mogen beginnen en hebben toen de metingen uitgevoerd. Ze zijn ook al begonnen met de verwerking van hun data. \newline\newline Hannah was vorige les afwezig. Ik zal zorgen dat ze de opstelling nog even ziet voor ze de data van haar groepje verwerkt.   \newline\newline Mijn vorige lessen (11-12) zijn algemeen gezien goed verlopen. Dit labo mag drie lessen in beslag nemen en dit zal zeker lukken. Het begin van volgende laboles zal ik echter wel wat anders aanpakken. Ook in het algemeen zal ik mij wat meer op de zwakkere leerlingen in de groep moeten richten en de sterkere vooral zelfstandig laten bezig zijn, in plaats van extra verdiepende vragen aan die laatste groep te stellen.} & \textbf{Acties}\newline\newline  
		- \YellowHighlight{Ik herhaal de inhoud van het labo nog eens kort: waarover deden jullie onderzoek}{15cm}  \YellowHighlight{en wat waren de onderzoeksvragen?}{7cm} Door deze herhaling zorg ik ervoor dat de beginsituatie voor iedereen terug gelijk is en hoop ik dat de leerlingen terug kunnen inpikken na de krokusvakantie. \newline\newline
		- \GreenHighlight{Bij een labo is het de bedoeling om in groep een resultaat op de gestelde onderzoeks-}{15cm} \GreenHighlight{vragen te bekomen.}{3.6cm} Hier worden de leerlingen in drie groepen onder verdeeld. Hierdoor zal er een goede wisselwerking kunnen zijn tussen de leerlingen onderling en zijn er voldoende kritische blikken per groep om de vragen op te lossen. Tegelijkertijd kan ik als leerkracht vlot tussen de groepjes laveren wanneer ik zowel goede als minder goede zaken observeer.
		\newline\newline\newline\newline\newline
		
		\\ \cline{2-2}
		  & \textbf{Bronnen}\begin{itemize}
		  	\item Schramme, S. (2018) De stroombalans, labo magnetisme 4
		  	\item Frederiksen (2014), Current Balance 4565.00
		  \end{itemize}\\ \hline
	\end{tabularx}


\newpage
	
	\begin{tabularx}{1.56\textwidth}{|p{1.5cm}|p{9cm}|X|p{4cm}|}
		\hline
		\textbf{Nr. lesdoel } & \textbf{Inhoud (timing)}  & \textbf{Organisatie } & \textbf{Media } \\ \hline
		&\underline{Herhaling onderzoeksvragen} \underline{labo (10 minuten)}\newline De leerlingen nemen plaats aan hun computer en krijgen hun bundel terug. Daarna overloop ik via vraagstelling aan de leerlingen nog eens de onderzoeksvragen van het labo.
		&  \underline{Onderwijsleergesprek}\newline 
			De leerlingen krijgen hun bundel terug van mij en we overlopen de onderzoeksvragen van het labo nog even gezamenlijk. We gaan nog even dieper in op de effecten van de Lorentzkracht op de draad en de reactiekracht op de magneet. \newline\newline Ik overloop samen met Hannah de labobundel en overloop nog even kort met haar de theorie via onderwijsleergesprek (Hoe loopt de stroom, magnetisch veld, hoe is de Lorentzkracht gericht \ldots ) en bouw de opstelling. Samen testen we één configuratie en vraag ik wat er zou gebeuren wanneer de magneet omgekeerd zou staan.  De anderen werken ondertussen verder aan hun werkbundel.
		&  Labobundel
		\\ \hline
	\end{tabularx}\vspace{5mm}

\begin{tabularx}{1.56\textwidth}{|p{1.5cm}|p{9cm}|X|p{4cm}|}
	\hline
	\textbf{Nr. lesdoel } & \textbf{Inhoud (timing)}  & \textbf{Organisatie } & \textbf{Media } \\ \hline
	1\newline\newline 2\newline\newline 3\newline\newline 4\newline\newline 5\newline\newline 6\newline\newline 7&\underline{Afwerken laboverslag magnetisme 4:} \underline{de stroombalans (40 minuten)}\newline
	De leerlingen werken per groep verder aan hun individueel verslag. Ze  dienen individueel een verslag in, maar ze mogen nog met hun groepsleden samenwerken.
	&  \underline{Onderzoekspracticum}\newline De leerlingen zullen deze tijd nodig hebben om hun laboverslag af te werken. Ze zijn bezig met het afronden van onderdeel 6. Hierna bouwen ze een algemene conclusie op rond de Lorentzkracht, de stroomsterkte en de lengte van de geleider. Vanuit dit besluit, berekenen ze dan de veldsterkte van de magneet. Vijf van de acht leerlingen hebben op dit moment de fout gemaakt om niet met SI-eenheden te werken. Ik heb hier bewust nog niets over gezegd, zodat ze dit in hun besluit zullen ondervinden en bij zichzelf de vraag zullen moeten stellen wat er nu precies fout gelopen is.\newline\newline Hannah begint aan de dataverwerking met de data van het groepje waarbij ze ingedeeld zat. Ik probeer om extra bij haar langs te gaan.\newline\newline Iedereen dient hun verslag in de bundel in en zorgt ervoor dat hun dataverwerking in excel geüpload en geprint wordt. Indien er leerlingen vroeger klaar zijn, dan heb ik een bundel met extra oefeningen over de Lorentzkracht waar ze in groep kunnen aan werken. \newline\newline Op het eind van de les zeg ik hen nog eens dat ze morgen zeker hoofdstuk 5 moeten mee hebben.
	&  Computers (computerlokaal)\newline\newline Labobundel
	\\ \hline
\end{tabularx}


	
\end{landscape}


%\subsection*{Bijlage 5.1: slides introductie}

%
%\subsection*{Bijlage 1.2: bordschema theorie}
%\begin{center}
%	\includegraphics[width=0.9\textwidth]{Bord1a}
%\includegraphics[width=0.9\textwidth]{Bord1b}
%\end{center}
%\newpage
%
%
%\includepdf[scale = 0.8,pages = 17,pagecommand=\subsection*{Bijlage 1.3: opgeloste oefeningen}]{Observaties_OpgelosteOef}
%\includepdf[scale = 0.8,pages =18-20,pagecommand=]{Observaties_OpgelosteOef}
%
%
%
%\includepdf[scale = 0.95,pages = 1,pagecommand=\subsection*{Bijlage 1.4: oefeningenbundel elektromagnetisme}]{OefeningenBundel}
%\includepdf[scale = 0.95,pages =2-,pagecommand=]{OefeningenBundel}
% !TeX root = Stageportfolio.tex



\begin{landscape}
	\subsubsection{Les 14-15}
	\begin{tabularx}{1.56\textwidth}{|p{0.35\textwidth}|X|}\hline
		\textbf{Administratieve gegevens}\newline\newline
		Kevin Truyaert\newline\newline
		technisch secundair onderwijs\newline
		3e graad, 1ste jaar, Techniek-Wetenschappen\newline
		VVKSO: \href{http://ond.vvkso-ict.com/leerplannen/doc/Toegepaste\%20fysica-2014-041.pdf}{http://ond.vvkso-ict.com/leerplannen /doc/Toegepaste\%20fysica-2014-041.pdf} \newline
		\underline{Lesonderwerp}:\newline Bespreking labo M4 \& Magnetische fluxverandering & \textbf{Doelstellingen}
		\begin{itemize}[itemsep=0.08\baselineskip]
			\item B27: Fluxverandering als oorzaak van inductiespanning toelichten
		\end{itemize}
		\underline{Lesdoelen}\newline
		\vspace{-0.75cm}
		\begin{enumerate}[itemsep=0.08\baselineskip]
			\item De leerlingen zien in hoe de lorentzkracht op de geleider een effect op een balans kan hebben.
			\item De leerlingen begrijpen conceptueel wat de magnetische flux is.ver
			\item De leerlingen kennen de drie factoren die de magnetische flux bepalen.
			\item De leerlingen kunnen de magnetische flux berekenen.
			\item De leerlingen begrijpen hoe een fluxverandering kan ontstaan.
			\item De leerlingen kunnen de fluxverandering berekenen.
			\item De leerlingen ervaren het ontstaan van een inductiespanning door middel van een fluxverandering via demo's.
			\item De leerlingen ervaren de analogie tussen elektriciteit en magnetisme bij het bespreken van inductieverschijnselen.
			\item De leerlingen kunnen de wet van Faraday zelf, via een demo, samenstellen.
		\end{enumerate} \\\hline
	\end{tabularx}\vfill \textcolor{white}{.} 


	\begin{tabularx}{1.56\textwidth}{|p{0.55\textwidth}|X|}
		\hline
		\multirow{2}{0.55\textwidth}{\textbf{Beginsituatie}\newline  
		Er zijn acht leerlingen binnen 5TW. Er heerst een algemene tot positieve klassfeer. De leerlingen hebben al theorie gekregen  rond en oefeningen gemaakt op de magnetische krachtwerking. \newline\newline De leerlingen hebben de dag hiervoor hun labo ingediend. Vandaag worden de belangrijkste aspecten hiervan nog even overlopen. Daarnaast zijn we gisteren ook begonnen aan een nieuw stuk theorie rond elektromagnetische inductie. Hiervan hebben we de basis rond magnetische flux al besproken. Dit wordt nog even terug aangehaald. \newline\newline Deze les omvat een volledig nieuw stuk theorie. Nieuwe theorie is voor mij een valkuil om in doceermodus te gaan. Ik wil mij hiervan bewust zijn en hier op letten om meer vraag gesteld les te geven en over te gaan op een onderwijsleergesprek-modus.} & \textbf{Acties}\newline\newline  
		- Magnetische flux is een concept dat niet voor te stellen valt. De gevolgen van deze flux kan je wel voorstellen en worden uitvoerig in hoofdstuk 6 besproken. Vooraleer je deze concepten echter kan beginnen te bespreken, moeten de leerlingen de basis van elektromagnetische inductie beheersen, waarvoor ze magnetische flux(verandering) moeten begrijpen. Toch verwacht ik dat de leerlingen hier vlot mee weg zullen zijn en wil ik veel in interactie treden met de leerlingen. \newline\newline
		- \GreenHighlight{Via demo's wil ik bepaalde onderwerpen starten.}{9cm}	Op die manier kan ik de interesse van de leerlingen wekken en kan ik fysische wetmatigheden hen effectief aantonen. Zo kunnen leerlingen op een klassikale manier zelfstandig dingen ontdekken.	
		\newline\newline\newline\newline\newline
		
		\\ \cline{2-2}
		  & \textbf{Bronnen}\begin{itemize}
		  	\item Schramme, S. (2018) De stroombalans, labo magnetisme 4
		  	\item Frederiksen (2014), Current Balance 4565.00
		  	\item Giancoli, D. C. (2008). Physics for scientists and engineers. Pearson Education International.
		  \end{itemize}\\ \hline
	\end{tabularx}


\newpage
	
	\begin{tabularx}{1.56\textwidth}{|p{1.5cm}|p{9cm}|X|p{4cm}|}
		\hline
		\textbf{Nr. lesdoel } & \textbf{Inhoud (timing)}  & \textbf{Organisatie } & \textbf{Media } \\ \hline
		1	&\underline{Bespreking Labo M4:} \underline{de stroombalans (15 minuten)}\newline
			 We gaan nog even dieper in op de effecten van de Lorentzkracht op de draad en de reactiekracht op de magneet om duidelijk te verklaren waarom de balans verschillen kon opmeten.
		&  \underline{Onderwijsleergesprek}\newline 
			De leerlingen krijgen hun door mij verbeterde labobundel terug en we overlopen de onderzoeksvragen van het labo nog even gezamenlijk. Ik vraag aan de leerlingen wat zij als essentie van het labo ervaren hebben. Vanuit dat standpunt wordt het labo besproken. Hierna wordt er niet meer terug gekomen op dit labo. Een duidelijk begrip van de Lorentzkracht is nodig voor de laatste twee hoofdstukken van magnetisme.
		&  Labobundel\newline\newline Slides (zie bijlage)
		\\ \hline
	\end{tabularx}\vspace{5mm}



\begin{tabularx}{1.56\textwidth}{|p{1.5cm}|p{9cm}|X|p{4cm}|}
	\hline
	\textbf{Nr. lesdoel } & \textbf{Inhoud (timing)}  & \textbf{Organisatie } & \textbf{Media } \\ \hline
    2\newline\newline 3& \underline{Magnetische flux (5 minuten)}\newline
    Het concept van magnetische flux, wat vorige les ingeleid werd, wordt nu kort herhaald.
	&  \underline{Onderwijsleergesprek}\newline  
	Ik teken een situatie met een magnetische flux op bord en de leerlingen zeggen mij wat de belangrijke eigenschappen zijn in verband met magnetische flux.	Welke drie componenten zijn er belangrijk en hoe is de flux hiervan afhankelijk?
	&  Cursus hoofdstuk 5 p1-2\newline\newline Krijtbord
	\\ \hline
\end{tabularx}\vspace{5mm}


\begin{tabularx}{1.56\textwidth}{|p{1.5cm}|p{8cm}|X|p{3cm}|}
	\hline
	\textbf{Nr. lesdoel } & \textbf{Inhoud (timing)}  & \textbf{Organisatie } & \textbf{Media } \\ \hline
	4& \underline{Magnetische flux: Oefeningen (20 minuten)}\newline
	Aangezien de leerlingen net de eigenschappen van magnetische flux gezien hebben, maken we eerst wat oefeningen hierop. Zo kunnen de leerlingen een beter begrip hierover krijgen.
	&  \underline{Onderwijsleergesprek + oefeningen}\newline  Oefening 2 a en b maak ik klassikaal met de leerlingen samen. Hierna werken de leerlingen oefening 2 individueel af. Ik schrijf ook op bord dat de leerlingen verder kunnen gaan met oefeningen 4, 5 en 6. De eindoplossing van die oefeningen komen op bord, de werkwijze zal enkel van oefening 5 op bord komen indien nodig.
	&  Cursus hoofdstuk 5 p4-5\newline\newline Krijtbord
	\\ \hline
\end{tabularx}\vspace{5mm}


\begin{tabularx}{1.56\textwidth}{|p{1.5cm}|p{8cm}|X|p{3cm}|}
\hline
\textbf{Nr. lesdoel } & \textbf{Inhoud (timing)}  & \textbf{Organisatie } & \textbf{Media } \\ \hline
5& \underline{Magnetische fluxverandering:} \underline{theorie (15 minuten)}\newline
Aangezien de leerlingen net de eigenschappen van magnetische flux gezien hebben, definieer ik nu de magnetische fluxverandering. De leerlingen komen te weten hoe die verandering tot stand kan komen.
&  \underline{Onderwijsleergesprek}\newline 
Ik teken een situatie met een magnetische flux op bord en de leerlingen zeggen mij wat de belangrijke eigenschappen zijn in verband met magnetische flux. Ik vraag de leerlingen hoe die fluxverandering kan ontstaan (net vanuit één van die drie eigenschappen). Zo bespreken we alle mogelijkheden van de opwekking van een fluxverandering.	
&  Cursus hoofdstuk 5 p2-3\newline\newline Krijtbord
\\ \hline
\end{tabularx}\vspace{5mm}



\begin{tabularx}{1.56\textwidth}{|p{1.5cm}|p{8cm}|X|p{3cm}|}
	\hline
	\textbf{Nr. lesdoel } & \textbf{Inhoud (timing)}  & \textbf{Organisatie } & \textbf{Media } \\ \hline
	5\newline\newline 6& \underline{Magnetische fluxverandering:} \underline{Oefeningen (15 minuten)}\newline
	Aangezien de leerlingen net de eigenschappen van magnetische fluxverandering gezien hebben, maken we eerst wat oefeningen hierop, om een beter begrip van de fluxverandering te krijgen. Dat is essentieel om aan inductiespanning te kunnen beginnen.
	&  \underline{Onderwijsleergesprek + oefeningen}\newline  Oefening 7 maak ik klassikaal, met de leerlingen samen, via vraagstelling aan de leerlingen. Hierna werken de leerlingen individueel oefening 8 en 9. Ik schrijf enkel de tussenoplossingen en de eindoplossing op bord. Ondertussen plaats ik het materiaal voor de demo rond inductiespanning en -stroom op tafel.
	&  Cursus hoofdstuk 5 p5-6\newline\newline Krijtbord
	\\ \hline
\end{tabularx}\vspace{5mm}



\begin{tabularx}{1.56\textwidth}{|p{1.5cm}|p{9cm}|X|p{4cm}|}
	\hline
	\textbf{Nr. lesdoel } & \textbf{Inhoud (timing)}  & \textbf{Organisatie } & \textbf{Media } \\ \hline
	7\newline\newline 8& \underline{Inductiespanning en -stroom:} \underline{Inleiding (10 minuten)}\newline
	Er werd in vorige lessen (Hoofdstuk 2) door de leerlingen ondervonden dat een elektrische stroom een magnetisch veld veroorzaakt. Hier onderzoeken we of het omgekeerde ook waar is: induceert een magnetisch veld een elektrische stroom in een gesloten circuit? Dit zal een interactie tussen elektriciteit en magnetisme aan de leerlingen tonen.
	&  \underline{Demonstratie + Onderwijsleergesprek}\newline 
	Ik begin de les met een applicatie van Walter Fendt die aantoont dat een magnetisch veld door een stroomgeleidende draad kan ontstaan. Hier vraag ik aan de leerlingen wat zij zien. Daarna vertel ik hen dat we gaan onderzoeken of het omgekeerde ook waar is.\newline De opstelling bestaat uit een spoel die aangesloten is aan een milliampèremeter, die zowel negatieve als positieve stromen kan meten. Daarna beweeg ik een magneet naar de spoel. Ik zeg niets en vraag aan de leerlingen wat zij beschrijven wat er gebeurt. Hier speel ik op in en treed ik in interactie met de leerlingen om van hen te horen wat zij ervaren wat er gebeurt.	Op basis hiervan interageer ik met de leerlingen om hen in hun bewoording te begeleiden. Hierna vullen we samen op basis van de demo pagina's 7 en 8 in.
	&  Cursus hoofdstuk 5 p7-8\newline\newline Krijtbord\newline\newline App
	\\ \hline
\end{tabularx}\vspace{5mm}




\begin{tabularx}{1.56\textwidth}{|p{1.5cm}|p{9cm}|X|p{4cm}|}
	\hline
	\textbf{Nr. lesdoel } & \textbf{Inhoud (timing)}  & \textbf{Organisatie } & \textbf{Media } \\ \hline
	7\newline\newline 8\newline\newline 9& \underline{De wet van Faraday:} \underline{Verbanden (15 minuten)}\newline
	Er is nu aangetoond dat in een gesloten circuit er een inductie stroom is. We kunnen ditzelfde experiment doen, maar in plaats van een milliampèremeter aan de spoel aan te sluiten, sluit ik nu een voltagemeter aan. Hieruit is het mogelijk om de inductiespanning te meten door middel van een fluxverandering. Door verschillende wijzigingen aan de situatie toe te brengen, kunnen de leerlingen zelf bepaalde evenredigheden uit de demonstratie halen. Samen met de leerlingen leid ik de wet van Faraday af.
	&  \underline{Demonstratie + Onderwijsleergesprek}\newline 
	De opstelling bestaat uit een spoel die aangesloten is aan een voltagemeter, die zowel negatieve als positieve spanningen kan meten. Daarna beweeg ik een magneet naar de spoel. Ik zeg niets en vraag aan de leerlingen wat zij beschrijven wat er gebeurt. Hier speel ik op in en treed ik in interactie met de leerlingen om van hen te horen wat zij ervaren wat er gebeurt.	Deze resultaten zullen ze snel begrijpen, gezien we net de situatie van de inductiestroom gezien hebben. Nu verander ik de windingen van de spoel, de flux (door middel van de magneet) en de tijdspanne waarin ik de magneet dichter breng. Deze relaties leiden uiteindelijk tot de wet van Faraday. Hierna vullen we samen op basis van de demo pagina's 9 en 10 in.
	&  Cursus hoofdstuk 5 p9-10\newline\newline Krijtbord
	\\ \hline
\end{tabularx}\vspace{5mm}


\begin{tabularx}{1.56\textwidth}{|p{1.5cm}|p{9cm}|X|p{4cm}|}
	\hline
	\textbf{Nr. lesdoel } & \textbf{Inhoud (timing)}  & \textbf{Organisatie } & \textbf{Media } \\ \hline
	& \underline{Slot (5 minuten)}\newline
	Ik herhaal nog even kort samen met de leerlingen de wet van Faraday. We bespreken samen wat deze voorstelt, een inductiespanning, en waarop deze steunt, een fluxverandering. Op deze manier probeer ik de leerlingen te evalueren.
	&  \underline{Vertellen}\newline 
	Bespreken van de wet van Faraday en fluxverandering
	&  
	\\ \hline
\end{tabularx}




	
\end{landscape}


%\subsection*{Bijlage 5.1: slides introductie}

%
%\subsection*{Bijlage 1.2: bordschema theorie}
%\begin{center}
%	\includegraphics[width=0.9\textwidth]{Bord1a}
%\includegraphics[width=0.9\textwidth]{Bord1b}
%\end{center}
%\newpage
%
%
%\includepdf[scale = 0.8,pages = 17,pagecommand=\subsection*{Bijlage 1.3: opgeloste oefeningen}]{Observaties_OpgelosteOef}
%\includepdf[scale = 0.8,pages =18-20,pagecommand=]{Observaties_OpgelosteOef}
%
%
%
%\includepdf[scale = 0.95,pages = 1,pagecommand=\subsection*{Bijlage 1.4: oefeningenbundel elektromagnetisme}]{OefeningenBundel}
%\includepdf[scale = 0.95,pages =2-,pagecommand=]{OefeningenBundel}
% !TeX root = Stageportfolio.tex



\begin{landscape}
	\subsubsection{Les 16}
	\begin{tabularx}{1.56\textwidth}{|p{0.35\textwidth}|X|}\hline
		\textbf{Administratieve gegevens}\newline\newline
		Kevin Truyaert\newline\newline
		technisch secundair onderwijs\newline
		3e graad, 1ste jaar, Techniek-Wetenschappen\newline
		VVKSO: \href{http://ond.vvkso-ict.com/leerplannen/doc/Toegepaste\%20fysica-2014-041.pdf}{http://ond.vvkso-ict.com/leerplannen /doc/Toegepaste\%20fysica-2014-041.pdf} \newline
		\underline{Lesonderwerp}:\newline Wet van Lenz \& algemene inductiewet: Faraday-Lenz + oefeningen & \textbf{Doelstellingen}
		\begin{itemize}[itemsep=0.08\baselineskip]
			\item B27: Fluxverandering als oorzaak van inductiespanning toelichten
			\item B28: Met behulp van de wet van Lenz de zin van de inductiespanning vinden
			\item B29: De algemene inductiewet hanteren.
		\end{itemize}
		\underline{Lesdoelen}\newline
		\vspace{-0.75cm}
		\begin{enumerate}[itemsep=0.08\baselineskip]
			\item De leerlingen kunnen de zin van de inductiestroom bepalen.
			\item De leerlingen kunnen de wet van Lenz verwoorden.
			\item De leerlingen zien de samenhang tussen de wet van Faraday en de wet van Lenz in.
			\item De leerlingen kennen de wet van Faraday-Lenz.
			\item De leerlingen kunnen de wet van Faraday-Lenz op een rechte, bewegende geleider toepassen.
			\item De leerlingen kennen de relatie tussen inductiespanning, magnetisch veld, lengte van de geleider, snelheid van de geleider en aantal windingen.
			\item De leerlingen kunnen de algemene inductiewet tijdens oefeningen hanteren.
		\end{enumerate} \\\hline
	\end{tabularx}\vfill \textcolor{white}{.} 


	\begin{tabularx}{1.56\textwidth}{|p{0.55\textwidth}|X|}
		\hline
		\multirow{2}{0.55\textwidth}{\textbf{Beginsituatie}\newline  
		Er zijn acht leerlingen binnen 5TW. Er heerst een algemene klassfeer. De leerlingen hebben al theorie gekregen  rond en oefeningen gemaakt op de magnetische krachtwerking. \newline\newline De leerlingen hebben vorige week de wet van Faraday gezien. Hiermee kunnen ze de grootte van de inductiespanning bepalen. Ook met de begrippen flux en fluxverandering zijn ze gekend. \newline\newline NOG AANVULLEN MET LERAARKENMERKEN.} & \textbf{Acties}\newline\newline 
		- \GreenHighlight{Via demo's wil ik bepaalde onderwerpen starten.}{9cm}	Op die manier kan ik de interesse van de leerlingen wekken en kan ik fysische wetmatigheden hen effectief aantonen. Zo kunnen leerlingen op een klassikale manier zelfstandig dingen ontdekken.	 \newline\newline 
		- Ik wil oefeningen op zo'n wijze brengen dat ze steeds dezelfde structuur hebben. Die structuur bouw ik eerst samen met de leerlingen op, om ze daarna zelfstandig aan de slag te laten gaan met oefeningen die steeds wat complexer worden. \PinkHighlight{Tijdens het zelfstandig maken van de oefeningen probeer ik toch zeker}{13cm} \PinkHighlight{de zwakkere leerlingen in de gaten te houden en hen individueler te coachen bij het}{15cm} \PinkHighlight{maken van oefeningen.}{5cm}
		\newline\newline\newline\newline\newline\newline
		
		\\ \cline{2-2}
		  & \textbf{Bronnen}\begin{itemize}
		  	\item Schramme, S. (2018) De stroombalans, labo magnetisme 4
		  	\item Frederiksen (2014), Current Balance 4565.00
		  	\item Giancoli, D. C. (2008). Physics for scientists and engineers. Pearson Education International.
		  \end{itemize}\\ \hline
	\end{tabularx}


\newpage
	
	\begin{tabularx}{1.56\textwidth}{|p{1.5cm}|p{8cm}|X|p{4cm}|}
		\hline
		\textbf{Nr. lesdoel } & \textbf{Inhoud (timing)}  & \textbf{Organisatie } & \textbf{Media } \\ \hline
		1\newline\newline 2&\underline{De wet van Lenz (20 minuten)}\newline
			De leerlingen bepalen de zin van de inductiestroom  aan de hand van een demo.
		&  \underline{Demo + Onderwijsleergesprek}\newline 
			Doormiddel van een elektromagneet en een spoel met een weekijzeren kern, wordt een stroom geïnduceerd in een ring die rond de weekijzeren kern hangt. De leerlingen zullen die zien bewegen. Samen leiden we af hoe de inductiestroom loopt, op basis van de beweging van de ring t.o.v. het magnetisch veld van de spoel. Zo worden pagina's 11 en 12 in de bundel aangevuld, aan de hand van wat de leerlingen observeren wat er met de ring gebeurt. Daarna besluiten we met de definitie van de wet van Lenz. Ik zorg dat verschillende leerlingen deze eens in eigen woorden ook zeggen, omdat dit een belangrijk gegeven is voor de toepassingen die in hoofdstuk 6 aan bod komen. 
		&   Cursus hoofdstuk 5 p11-12\newline\newline Krijtbord \newline\newline Elekromagneet, weekijzeren kern, hangende ring
		\\ \hline
	\end{tabularx}\vspace{5mm}

	
\begin{tabularx}{1.56\textwidth}{|p{1.5cm}|p{8cm}|X|p{4cm}|}
	\hline
	\textbf{Nr. lesdoel } & \textbf{Inhoud (timing)}  & \textbf{Organisatie } & \textbf{Media } \\ \hline
	3\newline\newline 4\newline\newline &\underline{De algemene inductiewet:} \underline{Faraday-Lenz (5 minuten)}\newline
	We kunnen de wet van Faraday-Lenz besluiten als combinatie van de wet van Faraday en de wet van Lenz.
	&  \underline{Onderwijsleergesprek}\newline 
	De algemene inductiewet schrijf ik nu op bord. Ik vraag aan de leerlingen om ieder deel te verklaren.
	&   Cursus hoofdstuk 5 p13\newline\newline Krijtbord 
	\\ \hline
\end{tabularx}\vspace{5mm}


\begin{tabularx}{1.56\textwidth}{|p{1.5cm}|p{8cm}|X|p{4cm}|}
	\hline
	\textbf{Nr. lesdoel } & \textbf{Inhoud (timing)}  & \textbf{Organisatie } & \textbf{Media } \\ \hline
	5\newline\newline 6&\underline{Faraday-Lenz op een rechte, bewegende} \underline{geleider (5 minuten)}\newline
	Toepassen van Faraday-Lenz op een bewegende geleider.
	&  \underline{Onderwijsleergesprek}\newline 
	Ik schets de situatie van een bewegende, rechte geleider die aan spanningsmeter verbonden is. We leiden de spanning die door de beweging geïnduceerd is af, door middel van de gekende formules. Ik laat de leerlingen hier zelfstandig mee starten, gezien de verschillende stappen al op hun blad aanwezig zijn, waarna ik inpik.
	&   Cursus hoofdstuk 5 p13\newline\newline Krijtbord 
	\\ \hline
\end{tabularx}\vspace{5mm}



\begin{tabularx}{1.56\textwidth}{|p{1.5cm}|p{8cm}|X|p{4cm}|}
	\hline
	\textbf{Nr. lesdoel } & \textbf{Inhoud (timing)}  & \textbf{Organisatie } & \textbf{Media } \\ \hline
    1\newline\newline 4 \newline\newline 7& \underline{Faraday-Lenz: oefeningen (18 minuten)}\newline
    De leerlingen maken oefeningen op Faraday-Lenz.	
	&  \underline{Oefeningen + onderwijsleergesprek}\newline 
	Ik vraag aan de leerlingen om zelf even oefening 1 te maken. Hiervoor moeten ze gebruik maken van de wet van Lenz. Ik treed na een minuut in interactie met de leerlingen om mij het antwoord op de vragen te geven. Daarna maak ik, door middel van vraagstelling aan de leerlingen, oefening 2. Ik bouw alle stappen op die ze bij dit soort oefeningen zullen moeten doen. Daarna maken ze zelf oefeningen 3, 4 en 5.
	&  Cursus hoofdstuk 5 p14-15\newline\newline Krijtbord
	\\ \hline
\end{tabularx}\vspace{5mm}


\begin{tabularx}{1.56\textwidth}{|p{1.5cm}|p{8cm}|X|p{4cm}|}
	\hline
	\textbf{Nr. lesdoel } & \textbf{Inhoud (timing)}  & \textbf{Organisatie } & \textbf{Media } \\ \hline
	& \underline{Slot (2 minuten)}\newline
	Ik herhaal nog even kort de wet van Lenz en de algemene inductiewet. Deze zullen belangrijk zijn bij volgende lessen gezien die nog meer oefeningen hierover bevatten en er ook toepassingen van magnetische inductie besproken zullen worden.	
	&  \underline{Onderwijsleergesprek + vertellen}\newline 
	&  Cursus hoofdstuk 5 \newline\newline Krijtbord
	\\ \hline
\end{tabularx}

	
\end{landscape}


%\subsection*{Bijlage 5.1: slides introductie}

%
%\subsection*{Bijlage 1.2: bordschema theorie}
%\begin{center}
%	\includegraphics[width=0.9\textwidth]{Bord1a}
%\includegraphics[width=0.9\textwidth]{Bord1b}
%\end{center}
%\newpage
%
%
%\includepdf[scale = 0.8,pages = 17,pagecommand=\subsection*{Bijlage 1.3: opgeloste oefeningen}]{Observaties_OpgelosteOef}
%\includepdf[scale = 0.8,pages =18-20,pagecommand=]{Observaties_OpgelosteOef}
%
%
%
%\includepdf[scale = 0.95,pages = 1,pagecommand=\subsection*{Bijlage 1.4: oefeningenbundel elektromagnetisme}]{OefeningenBundel}
%\includepdf[scale = 0.95,pages =2-,pagecommand=]{OefeningenBundel}	
% !TeX root = Stageportfolio.tex



\begin{landscape}
	\subsubsection{Les 17-18}
	\begin{tabularx}{1.56\textwidth}{|p{0.35\textwidth}|X|}\hline
		\textbf{Administratieve gegevens}\newline\newline
		Kevin Truyaert\newline\newline
		technisch secundair onderwijs\newline
		3e graad, 1ste jaar, Techniek-Wetenschappen\newline
		VVKSO: \href{http://ond.vvkso-ict.com/leerplannen/doc/Toegepaste\%20fysica-2014-041.pdf}{http://ond.vvkso-ict.com/leerplannen /doc/Toegepaste\%20fysica-2014-041.pdf} \newline
		\underline{Lesonderwerp}:\newline Oefeningen op de algemene inductiewet \& Toepassingen op inductie & \textbf{Doelstellingen}
		\begin{itemize}[itemsep=0.08\baselineskip]
			\item B27: Fluxverandering als oorzaak van inductiespanning toelichten
			\item B28: Met behulp van de wet van Lenz de zin van de inductiespanning vinden
			\item B29: De algemene inductiewet hanteren.
			\item B30: Het werkingsprincipe van een generator weergeven
		\end{itemize}
		\underline{Lesdoelen}\newline
		\vspace{-0.75cm}
		\begin{enumerate}[itemsep=0.08\baselineskip]
			\item De leerlingen kunnen de wet van Faraday-Lenz op een rechte, bewegende geleider toepassen.
			\item De leerlingen kennen de relatie tussen inductiespanning, magnetisch veld, lengte van de geleider, snelheid van de geleider en aantal windingen.
			\item De leerlingen kunnen de algemene inductiewet tijdens oefeningen hanteren.
			\item De leerlingen kunnen de werking van een wisselspanningsgenerator weergeven.
			
		\end{enumerate} \\\hline
	\end{tabularx}\vfill \textcolor{white}{.} 


	\begin{tabularx}{1.56\textwidth}{|p{0.55\textwidth}|X|}
		\hline
		\multirow{2}{0.55\textwidth}{\textbf{Beginsituatie}\newline  
		Er zijn acht leerlingen binnen 5TW. Er heerst een algemene klassfeer. De leerlingen hebben al theorie gekregen rond en basisoefeningen gemaakt op magnetische inductie.  \newline\newline NOG AANVULLEN MET LERAARKENMERKEN.} & \textbf{Acties}\newline\newline  
		- Ik wil oefeningen op zo'n wijze brengen dat ze steeds dezelfde structuur hebben. Die structuur bouw ik eerst samen met de leerlingen op, om ze daarna zelfstandig aan de slag te laten gaan met oefeningen die steeds wat complexer worden. \PinkHighlight{Tijdens het zelfstandig maken van de oefeningen probeer ik toch zeker}{13cm} \PinkHighlight{de zwakkere leerlingen in de gaten te houden en hen individueler te coachen bij het}{15cm} \PinkHighlight{maken van oefeningen.}{4.5cm}
		\newline\newline\newline\newline\newline\newline\newline\newline
		
		\\ \cline{2-2}
		  & \textbf{Bronnen}\begin{itemize}
		  	\item Schramme, S. (2018) De stroombalans, labo magnetisme 4
		  	\item Frederiksen (2014), Current Balance 4565.00
		  	\item Giancoli, D. C. (2008). Physics for scientists and engineers. Pearson Education International.
		  \end{itemize}\\ \hline
	\end{tabularx}


\newpage
	
	\begin{tabularx}{1.56\textwidth}{|p{1.5cm}|p{8cm}|X|p{4cm}|}
		\hline
		\textbf{Nr. lesdoel } & \textbf{Inhoud (timing)}  & \textbf{Organisatie } & \textbf{Media } \\ \hline
		1\newline\newline2\newline\newline3	&\underline{Oefeningen: de algemene} \underline{inductiewet (40 minuten)}\newline
			Tijdens deze lesfase focussen de leerlingen zich op het maken van oefeningen in verband met de algemene inductiewet. Deze combineert de wet van Faraday en de wet van Lenz, dus leerlingen moeten beide begrijpen om de oefeningen tot een goed einde te brengen. Tijdens deze oefeningensessie wil ik gebruik maken van correctiesleutels om de leerlingen hun oefeningen zelfstandig te laten corrigeren.
		&  \underline{Zelfstandig oefeningen maken} \underline{Bespreking via correctiesleutel}\newline 
			De leerlingen maken zelfstandig oefeningen 6 t.e.m. 11. Na het maken van iedere oefening kunnen ze een correctiesleutel ophalen waarin alle stappen beschreven staan. Zo kunnen de sterkere leerlingen zelfstandig meerdere (en complexere) oefeningen maken, terwijl ik mij concentreer op de zwakkere leerlingen. Wanneer ik zie dat er bij een bepaalde oefening klassikaal problemen zijn, kan ik bepaalde stappen op het bord brengen.
		&   Cursus hoofdstuk 5 p15-16\newline\newline Krijtbord
		\\ \hline
	\end{tabularx}\vspace{5mm}



\begin{tabularx}{1.56\textwidth}{|p{1.5cm}|p{8cm}|X|p{4cm}|}
	\hline
	\textbf{Nr. lesdoel } & \textbf{Inhoud (timing)}  & \textbf{Organisatie } & \textbf{Media } \\ \hline
    4 & \underline{Werking wisselspanningsgenerator:} \underlin{inleiding (5 minuten)}\newline
    	Opzet van de wisselspanningsgenerator verduidelijken en kennismaken met de werking ervan.
	&  \underline{Onderwijsleergesprek}\newline 
	Ik schets de opstelling van de wisselspanningsgenerator en vraag de leerlingen om alle componenten aan te duiden, te benoemen, \ldots 
	&  Cursus hoofdstuk 6 p6\newline\newline Slides
	\\ \hline
\end{tabularx}\vspace{5mm}


\begin{tabularx}{1.56\textwidth}{|p{1.5cm}|p{9cm}|X|p{4cm}|}
	\hline
	\textbf{Nr. lesdoel } & \textbf{Inhoud (timing)}  & \textbf{Organisatie } & \textbf{Media } \\ \hline
	4& \underline{Werking wisselspanningsgenerator:} \underline{eerste kwartdraai (5 minuten)}\newline
	De werking van de wisselspanningsgenerator wordt verduidelijkt. Dit zal in verschillende stappen gebeuren.
	&  \underline{Onderwijsleergesprek}\newline  De leerlingen kennen de werking van de algemene inductiewet. Via vraagstelling wil ik samen met hen de eerste kwartdraai van de wisselspanningsgenerator bespreken. Zo vullen we samen het eerste kader op pagina 7 in.
	&  Cursus hoofdstuk 6 p7\newline\newline Krijtbord
	\\ \hline
\end{tabularx}\vspace{5mm}


\begin{tabularx}{1.56\textwidth}{|p{1.5cm}|p{9cm}|X|p{4cm}|}
\hline
\textbf{Nr. lesdoel } & \textbf{Inhoud (timing)}  & \textbf{Organisatie } & \textbf{Media } \\ \hline
4& \underline{Werking wisselspanningsgenerator:} \underline{eerste kwartdraai (5 minuten)}\newline
De werking van de wisselspanningsgenerator wordt verduidelijkt. Dit zal in verschillende stappen gebeuren.
&  \underline{Onderwijsleergesprek}\newline  De leerlingen kennen de werking van de algemene inductiewet. Via vraagstelling wil ik samen met hen de eerste kwartdraai van de wisselspanningsgenerator bespreken. Zo vullen we samen het eerste kader op pagina 7 in.
&  Cursus hoofdstuk 6 p7\newline\newline Krijtbord
\\ \hline
\end{tabularx}\vspace{5mm}



\begin{tabularx}{1.56\textwidth}{|p{1.5cm}|p{9cm}|X|p{4cm}|}
	\hline
	\textbf{Nr. lesdoel } & \textbf{Inhoud (timing)}  & \textbf{Organisatie } & \textbf{Media } \\ \hline
	5\newline\newline 6& \underline{Magnetische fluxverandering:} \underline{Oefeningen (15 minuten)}\newline
	Aangezien de leerlingen net de eigenschappen van magnetische fluxverandering gezien hebben, maken we eerst wat oefeningen hierop. Die zijn essentieel om aan inductiespanning te kunnen beginnen.
	&  \underline{Onderwijsleergesprek + oefeningen}\newline  Oefening 7 maak ik klassikaal, met de leerlingen samen, via vraagstelling aan de leerlingen. Hierna werken de leerlingen individueel oefening 8 en 9. Ik schrijf enkel de tussenoplossingen en de eindoplossing op bord. Ondertussen plaats ik het materiaal voor de demo rond inductiespanning en -stroom op tafel.
	&  Cursus hoofdstuk 5 p5-6\newline\newline Krijtbord
	\\ \hline
\end{tabularx}\vspace{5mm}



\begin{tabularx}{1.56\textwidth}{|p{1.5cm}|p{9cm}|X|p{4cm}|}
	\hline
	\textbf{Nr. lesdoel } & \textbf{Inhoud (timing)}  & \textbf{Organisatie } & \textbf{Media } \\ \hline
	7\newline\newline 8& \underline{Inductiespanning en -stroom:} \underline{Inleiding (10 minuten)}\newline
	Er werd in vorige lessen (Hoofdstuk 2) door de leerlingen ondervonden dat een elektrische stroom een magnetisch veld veroorzaakt. Hier onderzoeken we of het omgekeerde ook waar is: induceert een magnetisch veld een elektrische stroom in een gesloten circuit? Dit zal een interactie tussen elektriciteit en magnetisme aan de leerlingen tonen.
	&  \underline{Demonstratie + Onderwijsleergesprek}\newline 
	De opstelling bestaat uit een spoel die aangesloten is aan een milliampèremeter, die zowel negatieve als positieve stromen kan meten. Daarna beweeg ik een magneet naar de spoel. Ik zeg niets en vraag aan de leerlingen wat zij beschrijven wat er gebeurt. Hier speel ik op in en treed ik in interactie met de leerlingen om van hen te horen wat zij ervaren wat er gebeurt.	Op basis hiervan interageer ik met de leerlingen om hen in hun bewoording te begeleiden. Hierna vullen we samen op basis van de demo pagina's 7 en 8 in.
	&  Cursus hoofdstuk 5 p7-8\newline\newline Krijtbord
	\\ \hline
\end{tabularx}\vspace{5mm}




\begin{tabularx}{1.56\textwidth}{|p{1.5cm}|p{9cm}|X|p{4cm}|}
	\hline
	\textbf{Nr. lesdoel } & \textbf{Inhoud (timing)}  & \textbf{Organisatie } & \textbf{Media } \\ \hline
	7\newline\newline 8\newline\newline 9& \underline{De wet van Faraday:} \underline{Verbanden (15 minuten)}\newline
	Er is nu aangetoond dat in een gesloten circuit er een inductie stroom is. We kunnen ditzelfde experiment doen, maar in plaats van een milliampèremeter aan de spoel aan te sluiten, sluit ik nu een voltagemeter aan. Hieruit is het mogelijk om de inductiespanning te meten door middel van een fluxverandering. Door verschillende wijzigingen aan de situatie toe te brengen, kunnen de leerlingen zelf bepaalde evenredigheden uit de demonstratie halen. Samen met de leerlingen leid ik de wet van Faraday af.
	&  \underline{Demonstratie + Onderwijsleergesprek}\newline 
	De opstelling bestaat uit een spoel die aangesloten is aan een voltagemeter, die zowel negatieve als positieve spanningen kan meten. Daarna beweeg ik een magneet naar de spoel. Ik zeg niets en vraag aan de leerlingen wat zij beschrijven wat er gebeurt. Hier speel ik op in en treed ik in interactie met de leerlingen om van hen te horen wat zij ervaren wat er gebeurt.	Deze resultaten zullen ze snel begrijpen, gezien we net de situatie van de inductiestroom gezien hebben. Nu verander ik de windingen van de spoel, de flux (door middel van de magneet) en de tijdspanne waarin ik de magneet dichter breng. Deze relaties leiden uiteindelijk tot de wet van Faraday. Hierna vullen we samen op basis van de demo pagina's 9 en 10 in.
	&  Cursus hoofdstuk 5 p9-10\newline\newline Krijtbord
	\\ \hline
\end{tabularx}\vspace{5mm}


\begin{tabularx}{1.56\textwidth}{|p{1.5cm}|p{9cm}|X|p{4cm}|}
	\hline
	\textbf{Nr. lesdoel } & \textbf{Inhoud (timing)}  & \textbf{Organisatie } & \textbf{Media } \\ \hline
	& \underline{Slot (5 minuten)}\newline
	Ik herhaal nog even kort samen met de leerlingen de wet van Faraday. We bespreken samen wat deze voorstelt, een inductiespanning, en waarop deze steunt, een fluxverandering. Op deze manier probeer ik de leerlingen te evalueren.
	&  \underline{Vertellen}\newline 
	Bespreken van de wet van Faraday en fluxverandering
	&  
	\\ \hline
\end{tabularx}




	
\end{landscape}


%\subsection*{Bijlage 5.1: slides introductie}

%
%\subsection*{Bijlage 1.2: bordschema theorie}
%\begin{center}
%	\includegraphics[width=0.9\textwidth]{Bord1a}
%\includegraphics[width=0.9\textwidth]{Bord1b}
%\end{center}
%\newpage
%
%
%\includepdf[scale = 0.8,pages = 17,pagecommand=\subsection*{Bijlage 1.3: opgeloste oefeningen}]{Observaties_OpgelosteOef}
%\includepdf[scale = 0.8,pages =18-20,pagecommand=]{Observaties_OpgelosteOef}
%
%
%
%\includepdf[scale = 0.95,pages = 1,pagecommand=\subsection*{Bijlage 1.4: oefeningenbundel elektromagnetisme}]{OefeningenBundel}
%\includepdf[scale = 0.95,pages =2-,pagecommand=]{OefeningenBundel}	
% !TeX root = Stageportfolio.tex



\begin{landscape}
	Mijn stage werd vroegtijdig beëindigd vanwege de sluiting van de scholen omtrent de maatregelen die voltrokken werden vanwege Covid-19. Hieronder wil ik toch mijn reeds getroffen voorbereidingen plaatsen: de lesvoorbereiding (weliswaar met nog onvolledige beginsituatie en acties) van les 21 en het labo dat ik tijdens les 22 en 23 zou begeleiden.
	
	\subsubsection{Les 21}
	
	\begin{tabularx}{1.56\textwidth}{|p{0.35\textwidth}|X|}\hline
		\textbf{Administratieve gegevens}\newline\newline
		Kevin Truyaert\newline\newline
		technisch secundair onderwijs\newline
		3e graad, 1ste jaar, Techniek-Wetenschappen\newline
		VVKSO: \href{http://ond.vvkso-ict.com/leerplannen/doc/Toegepaste\%20fysica-2014-041.pdf}{http://ond.vvkso-ict.com/leerplannen /doc/Toegepaste\%20fysica-2014-041.pdf} \newline
		\underline{Lesonderwerp}:\newline De transformator en het elektrisch energietransport  & \textbf{Doelstellingen}
		\begin{itemize}[itemsep=0.08\baselineskip]
			\item B28: Met behulp van de wet van Lenz de zin van de inductiespanning vinden.
			\item B29: De algemene inductiewet hanteren.
			\item B31: De transformatorhouding bij de spanningen en de stromen van een ideale transformator toepassen en zijn functie bij het transport van elektrische energie toelichten.
		\end{itemize}
		\underline{Lesdoelen}\newline
		\vspace{-0.75cm}
		\begin{enumerate}[itemsep=0.08\baselineskip]
			\item De leerlingen kunnen het doel van een transformator verwoorden.
			\item De leerlingen kunnen de zin van de inductiestroom bepalen.
			\item De leerlingen kunnen de wet van Faraday-Lenz hanteren.
			\item De leerlingen kennen de transformatieverhouding voor spanningen bij een transformator.
			\item De leerlingen kennen de transformatieverhouding voor stromen bij een transformator.
			\item De leerlingen begrijpen de werking van een transformator.
			\item De leerlingen begrijpen waarom elektrische energie bij hoge spanningen vervoerd wordt. 
			\item De leerlingen kunnen het afgelegde traject tussen centrale en het stopcontact schetsen.
			\item De leerlingen kunnen de transformatieverhouding voor spanningen in oefeningen toepassen.  
			\item De leerlingen kunnen de transformatieverhouding voor stromen in oefeningen toepassen.  
		\end{enumerate} \\\hline
	\end{tabularx}\vfill \textcolor{white}{.} 


	\begin{tabularx}{1.56\textwidth}{|p{0.55\textwidth}|X|}
		\hline
		\multirow{2}{0.55\textwidth}{\textbf{Beginsituatie}\newline  
		Er zijn acht leerlingen binnen 5TW. Er heerst een algemene klassfeer. De leerlingen hebben al theorie gekregen  rond en oefeningen gemaakt op de magnetische krachtwerking. \newline\newline De leerlingen hebben vorige week de wisselspanningsgenerator gezien als een toepassing van magnetische inductie. Daarnaast kregen ze ook al een inleiding tot de transformator \newline\newline NOG AANVULLEN MET LERAARKENMERKEN.} & \textbf{Acties}\newline\newline 
		- \YellowHighlight{De transformator is een stuk fysica die in ons dagelijkse leven onbewust vaak}{15cm} \YellowHighlight{gebruikt wordt.}{3cm} Het zit in alle adapters die we gebruiken. Daarom is het belangrijk dat dit ook in de fysicales besproken wordt om de werking ervan te begrijpen.	 \newline\newline 
		\newline\newline\newline\newline\newline\newline
		
		\\ \cline{2-2}
		  & \textbf{Bronnen}\begin{itemize}
		  	\item Schramme, S. (2018) De stroombalans, labo magnetisme 4
		  	\item Frederiksen (2014), Current Balance 4565.00
		  	\item Giancoli, D. C. (2008). Physics for scientists and engineers. Pearson Education International.
		  \end{itemize}\\ \hline
	\end{tabularx}


\newpage
	
	\begin{tabularx}{1.56\textwidth}{|p{1.5cm}|p{8cm}|X|p{4cm}|}
		\hline
		\textbf{Nr. lesdoel } & \textbf{Inhoud (timing)}  & \textbf{Organisatie } & \textbf{Media } \\ \hline
		1&\underline{Herhaling wat is transformator (5 minuten)}\newline
			De leerlingen herhalen de componenten en het doel van de transformator.
		&  \underline{Onderwijsleergesprek}\newline 
			Ik schets een transformator op het bord en vraag de leerlingen wat ze van de componenten nog kunnen benoemen.
		&   Cursus hoofdstuk 6 p11\newline\newline Krijtbord \newline\newline Zelfgemaakte transfo met spoelen en weekijzeren kern staat op tafel
		\\ \hline
	\end{tabularx}\vspace{5mm}


	\begin{tabularx}{1.56\textwidth}{|p{1.5cm}|p{8cm}|X|p{4cm}|}
		\hline
		\textbf{Nr. lesdoel } & \textbf{Inhoud (timing)}  & \textbf{Organisatie } & \textbf{Media } \\ \hline
		2\newline3\newline4\newline5\newline6&\underline{Werking van de transformator (15 minuten)}\newline
		De leerlingen gebruiken de algemene inductiewet om de transformatorverhoudingen voor de spanningen en de stromen af te leiden. Door deze afleiding zullen ze ook de werking van de transformator begrijpen.
		&  \underline{Onderwijsleergesprek}\newline 
		Ik start  met het ondervragen in verband met de primaire spoel onder wisselspanning: welke fenomenen zullen er hierdoor optreden? Zo leid ik samen met de leerlingen de transformatieverhoudingen voor de spanningen en stromen af. Hierbij verwijs ik naar de tekening en naar de transfo die op tafel staat. Op die manier kunnen de leerlingen verschillende visualisaties bij dit onderwerp krijgen.
		&   Cursus hoofdstuk 6 p12\newline\newline Krijtbord \newline\newline Zelfgemaakte transfo met spoelen en weekijzeren kern staat op tafel
		\\ \hline
	\end{tabularx}\vspace{5mm}

	\begin{tabularx}{1.56\textwidth}{|p{1.5cm}|p{8cm}|X|p{4cm}|}
	\hline
	\textbf{Nr. lesdoel } & \textbf{Inhoud (timing)}  & \textbf{Organisatie } & \textbf{Media } \\ \hline
	1\newline6\newline7\newline8&\underline{Transport van elektrische} \underline{energie (15 minuten)}\newline
	Door het transport van elektrische energie te bespreken, zullen leerlingen een ander aspect van het nut van transformatoren ervaren. Ze zullen inzien dat er minder verlies is bij hoge spanningen, in vergelijking met lage spanningen. Toch kunnen we in onze leefomgeving deze hoge spanningen niet gebruiken, waardoor er transfo's gebruikt worden.
	&  \underline{Groepswerk + klassikale bespreking}\newline 
	Ik laat de leerlingen per twee aan de slag gaan om één kolom in te vullen. Daarna bespreken we klassikaal beide kolommen en besluiten de leerlingen op welke manier zij de energie zouden transporteren. Hierna bespreken we nog even kort het traject dat de energie effectief tussen centrale en klant aflegt.
	
	&   Cursus hoofdstuk 6 p13-14\newline\newline Krijtbord \newline\newline Zelfgemaakte transfo met spoelen en weekijzeren kern staat op tafel
	\\ \hline
	\end{tabularx}\vspace{5mm}

		\begin{tabularx}{1.56\textwidth}{|p{1.5cm}|p{8cm}|X|p{4cm}|}
		\hline
		\textbf{Nr. lesdoel } & \textbf{Inhoud (timing)}  & \textbf{Organisatie } & \textbf{Media } \\ \hline
		9\newline 10&\underline{Oefeningen transfo (12 minuten)}\newline
		Om voor een beter begrip van transfo's te zorgen en om een goede voorbereiding van het labo van morgen te hebben, maken we nog enkele oefeningen.
		&  \underline{Oefeningen + klassikale bespreking}\newline 
		Ik maak via onderwijsleergesprek eerst oefening 1 klassikaal. Hierna laat ik de leerlingen zelfstandig oefeningen 2 t.e.m. 5 maken. Ik probeer hier weer dat ik extra aandacht aan de minder sterke leerlingen schenk.
		&   Cursus hoofdstuk 6 p15\newline\newline Krijtbord \newline\newline Zelfgemaakte transfo met spoelen en weekijzeren kern staat op tafel
		\\ \hline
	\end{tabularx}\vspace{5mm}

\begin{tabularx}{1.56\textwidth}{|p{1.5cm}|p{8cm}|X|p{4cm}|}
	\hline
	\textbf{Nr. lesdoel } & \textbf{Inhoud (timing)}  & \textbf{Organisatie } & \textbf{Media } \\ \hline
	1&\underline{Slot (3 minuten)}\newline
	Ter voorbereiding van het labo herhaal ik nog even de kernbegrippen bij de transfo.
	&  \underline{Doceren / onderwijsleergesprek}\newline 
	Ik bespreek nog kort even de algemene werking van een transformator. Ik leg nog eens de nadruk op de transformatieverhoudingen bij ideale transfo's, maar benadruk dat ideale transfo's niet bestaan en dat ze dit morgen zullen ervaren.
	& Zelfgemaakte transfo met spoelen en weekijzeren kern staat op tafel
	\\ \hline
\end{tabularx}\vspace{5mm}


\end{landscape}
\includepdf[scale = 0.8,pages = 1,pagecommand=\subsubsection{Labo les 22-23}]{M6_DeTransformator1920}
\includepdf[scale = 0.8,pages =2-,pagecommand=]{M6_DeTransformator1920}







%\subsection*{Bijlage 5.1: slides introductie}

%
%\subsection*{Bijlage 1.2: bordschema theorie}
%\begin{center}
%	\includegraphics[width=0.9\textwidth]{Bord1a}
%\includegraphics[width=0.9\textwidth]{Bord1b}
%\end{center}
%\newpage
%
%
%\includepdf[scale = 0.8,pages = 17,pagecommand=\subsection*{Bijlage 1.3: opgeloste oefeningen}]{Observaties_OpgelosteOef}
%\includepdf[scale = 0.8,pages =18-20,pagecommand=]{Observaties_OpgelosteOef}
%
%
%
%\includepdf[scale = 0.95,pages = 1,pagecommand=\subsection*{Bijlage 1.4: oefeningenbundel elektromagnetisme}]{OefeningenBundel}
%\includepdf[scale = 0.95,pages =2-,pagecommand=]{OefeningenBundel}	
	
	
	\section{Bespreking meso-activiteiten}
	Stel per meso-activiteit een verslag op op basis van volgende criteria:
	\begin{itemize}
		\item Korte situering van de drie activiteiten.
		\item Omschrijving van twee aspecten die je voor jezelf geleerd hebt uit de deelname aan de activiteiten
		\item  Toon aan met twee voorbeelden dat de activiteiten een meerwaarde zijn voor de leerkrachten.
		\item Toon aan met twee voorbeelden dat de activiteiten een meerwaarde vormen voor de leerlingen.
		\item Bespreek hoe het komt dat bepaalde activiteiten geen echte meerwaarde hebben voor leerlingen en op welke manier deze aangepast kunnen worden om toch nog functioneel te zijn voor het leerproces van de leerlingen.
	\end{itemize}

	\subsection{Meso-activiteit 1: vakwerkgroepvergadering fysica KU Leuven campus Kortrijk Kulak}
	\subsubsection{Omschrijving van de activiteiten}
	Tijdens de semesteriële vergaderingen worden de educatieve taken van het volgende semester verdeeld en besproken en worden de educatieve taken van het vorige semester geëvalueerd. Er worden ook nieuwe aankopen voorgesteld, nieuwe practica toegelicht en herstellingen van het educatief materiaal besproken. Op niet educatief vlak worden:
	\begin{itemize}
		\item nieuwe collega's voorgesteld en worden de vertrekken van ex-collega's aangehaald,
		\item de kinderuniversiteit besproken (onderwerp, wie \ldots),
		\item Dag van de Wetenschap (wie, onderwerpen, standen \ldots)
		\item varia besproken.
	\end{itemize}  
	
	\subsection{Twee aspecten die ik voor mezelf geleerd heb}
		Deze meetings zijn deel van mijn job en ik heb deze vergaderingen de voorbije drie jaar semesterieel gevolgd. 
	\begin{itemize}
		\item Bij deze vergadering ben ik enkele nieuwe zaken te weten gekomen dat collega's een practicum dat ik mee helpen ontwikkelen heb, willen overnemen. Ze vonden het uitgewerkte experiment rond Doppler voldoen aan hun doelstellingen en vinden het interessanter dat de studenten hierover een practicum maken, dan over een ander onderwerp. Hieruit haal ik dat anderen mijn werk rond het bereiken van bepaalde doelstellingen zeker kunnen appreciëren en zelfs in hun eigen vak willen implementeren. Hier werd ook besproken dat er nieuw materiaal zal aangekocht moeten worden, gezien er niet genoeg opstellingen zijn voor de nieuwe groep waarbij iedereen tegelijkertijd hetzelfde practicum uitvoert.
		\item De vakken die ik de voorbije jaren tijdens het eerste semester verzorgde, blijf ik dit semester behouden. Ik zal dus dezelfde oefenzittingen verzorgen, die ik dit jaar wat probeer aan te passen. Ik zal ook nieuwe ingenieursprojecten moeten uitwerken voor een ander vak dat ik verzorg. Hier proberen we als team steeds voor interessante en relevante topics te kiezen, over een breed gamma van iets fysieks realiseren tot een software programma schrijven en alles daar tussen.
	\end{itemize}


	\subsection{Twee voorbeelden die aantonen dat de activiteiten een meer-waarde zijn voor leerkrachten}
	\begin{itemize}
		\item Tijdens deze vergaderingen worden er zaken omtrent het fysica onderwijs aan de Kulak besproken. Het is zeker interessant om te horen hoe collega's van andere vakken bepaalde zaken aanpakken. Dit kan zowel gaan over hoorcolleges als oefenzittingen en practica. 
		\item Je blijft op de hoogte van activiteiten die de universiteit tot een ander doelpubliek richt. Zo worden de onderwerpen van de kinderuniversiteit en de Dag van de Wetenschap tijdens deze meetings aan de niet-deelnemende collega's afgetoetst. Hun feedback nemen we dan mee om nog eventuele aanpassingen te doen.
	\end{itemize}


	\subsection{Twee voorbeelden die aantonen dat de activiteiten een meer-waarde zijn voor leerlingen}
	\begin{itemize}
		\item Tijdens deze vergaderingen worden de vakken globaal besproken. Als verantwoordelijke voor een vak, kan je dus hulp, commentaar, extra info \ldots{} bij je collega's fysica verkrijgen. Dit zorgt ervoor dat het onderwijs voor de studenten kwalitatiever wordt. 
		\item Nieuw lesmateriaal wordt hier ook besproken. Zowel het vernieuwen van bestaand materiaal als het raadplegen van nieuwe materialen die dan tijdens nieuwe practica gebruikt kunnen worden. Zo zorgen we ervoor dat het onderwijs ook aanschouwelijk blijft. We proberen dit te realiseren door te blijven inzetten om met didactisch materiaal de leerstof op alle drie lesniveau's (hoorcollege, werkzitting, practicum) te ondersteunen.
	\end{itemize}
	
	\subsection{Bespreek hoe het komt dat bepaalde activiteiten geen echte meerwaarde hebben voor leerlingen en op welke manier deze aangepast kunnen worden om toch nog functioneel te zijn voor het leerproces van de leerlingen}
	Hier heb ik geen zaken, aangezien alles wat hier besproken wordt in zake met onderwijs te maken heeft. Enerzijds voor de studenten en anderzijds voor externen (kinderuniversiteit: 8-13 jarigen; Dag van de Wetenschap: bezoekers van alle leeftijden; \ldots).


	
	\subsection{Meso-activiteit 2: kinderuniversiteit Kulak}
	\subsubsection{Omschrijving van de activiteiten}  Op zaterdag 26 oktober ging aan de katholieke universiteit campus kulak kortrijk de kinderuniversiteit door. Tijdens deze dag kunnen jongeren tussen 8 en 13 jaar ofwel de voormiddag, namiddag of hele dag op de universiteit doorbrengen. Per sessie wordt er zowel een lezing (45min) als een workshop (1u30min) aangereikt; de lezing wordt door iedereen gevolgd, waarna de jongeren zich verspreiden om per 20 à 25 een workshop te volgen.\newline
	
	
	De 15e editie van de kinderuniversiteit stond in het teken van `reis door de tijd'. De werknemers van de Kulak voorzagen tien verschillende workshops. Enkele personen binnen de fysica, waartoe ik behoor, bedachten een workshop genaamd `Bouw nu een telescoop en kijk straks naar het Universum van vroeger!'. Hiermee willen we de leerlingen bekend maken met de werking van lenzen, dat je de kleuren van de regenboog uit wit licht kan halen en dat je in het verleden kijkt wanneer je met een telescoop naar de sterren kijkt. De leerlingen krijgen tijdens de workshop eerst een halfuur uitleg van de professor door middel van een presentatie met slides en demonstratiemateriaal. Tijdens deze presentatie begint de professor met uit te leggen hoe licht werkt. Hij toont breking van licht met behulp van een laserstraal, een glazen halve cirkel (om het licht te breken) en wat krijtstof. Om reflectie duidelijk te maken, wordt er een spiegel aan de leerlingen doorgegeven. Daarna legt de prof uit hoe zowel holle als bolle lenzen werken, hoe ze ervoor zorgen dat dingen vergroot en verkleind worden en hoe je lenzen kan gebruiken om naar de ruimte te kijken. Daarna legt de professor nog uit dat het licht wel heel snel gaat, maar niet oneindig snel. Hierdoor zie je sterren zoals ze in het verleden waren. \newline\newline
	
	Na deze uitleg gaan de leerlingen aan de slag met het maken van een minitelescoop. Hiervoor gebruiken ze:
	\begin{itemize}
		\item 2 PVC-buizen met een verschillende diameter die in elkaar schuiven
		\item Twee verschillende lenzen
		\item 3D-geprinte lenshouders
		\item Plakband en versiering.
	\end{itemize}
	\begin{figure}[!h]
		\centering
		\includegraphics[width=0.9\textwidth]{Telescoop}
		\caption{Het materiaal waarmee de telescoop gemaakt wordt.}
		\label{Fig::Telescoop}
	\end{figure}
	Tijdens het maken van hun telescoop werden de leerlingen per drie meegenomen om zelf de eigenschappen van lenzen te ondervinden. Er werd een figuur op doorschijnende folie afgedrukt die een hond voorstelt en wanneer je de figuur ondersteboven houdt een kat toont. De figuur staat hieronder in beide opzichten.
	
	\begin{figure}[!h]
		\centering
		\includegraphics[width=0.5\textwidth]{HondKat}
		\caption{De figuur die gebruikt werd om de  leerlingen de eigenschappen van bolle lenzen te laten ondervinden.}
		\label{Fig::HondKat}
	\end{figure}
	Door middel van een opstelling met bolle lenzen is het mogelijk om beide figuren zichtbaar te maken, aangezien bolle lenzen het beeld kunnen omdraaien. De opstelling werd aan de leerlingen voorgesteld en iedere component werd benoemd. Door aan de  leerlingen de vraag te stellen hoe het mogelijk is dat beide beelden uit het ene beeld voortkomen wisten er sommigen de eigenschappen van bolle lenzen, die ze net gehoord hadden, te gebruiken om dit te verklaren. 
	
	Wanneer alle jongeren hun telescoop gemaakt hebben, kunnen ze op zoek gaan naar hun naam die op een ster geschreven staat. Die sterren hangen een eindje verder, waardoor hun namen niet zichtbaar zijn met het blote oog, maar wel met de gemaakte telescoop. 
	
	\subsection{Twee aspecten die ik voor mezelf geleerd heb}
	\begin{itemize}
		\item Ik vind het echt plezant om fysica uit te leggen. Mijn doelpubliek doet er niet toe. Ik pas mij aan en leg een bepaald onderwerp uit op het niveau van de persoon die tegenover mij zit.
		\item Leerlingen uit het lager onderwijs kunnen ook heel veel vragen hebben over een fysica onderwerp. Ze willen begrijpen hoe de wereld rondom hen draait op zowel conceptueel als reëel vlak. Ze zijn niet bang om verder te vragen, ook al vatten ze niet volledig wat je net uitgelegd hebt.
	\end{itemize}
	
	\subsection{Twee voorbeelden die aantonen dat de activiteiten een meer-waarde zijn voor leerkrachten}
	\begin{itemize}%[labelwidth=3em,leftmargin =\dimexpr\labelwidth+\labelsep\relax]
		\item Ondanks dat de werking van lenzen geen makkelijke materie is, was ik verbaasd van de interpretatie van sommige jongeren bij de proef met de hond-kat. In eerste instantie vonden ze het heel vreemd wat er aan de hand was: ze zagen twee verschillende beelden, maar die kwamen allebei van dezelfde foto. Door als leerkracht hier gerichte vragen te stellen, kun je de leerlingen zelfontdekkend laten leren. Het zijn zijzelf die de link leggen tussen de eigenschappen van lenzen: bolle lenzen draaien je beeld om en maken het reëel, terwijl holle lenzen de oriëntatie van het beeld behouden, maar dat het beeld virtueel wordt. Hierdoor wordt mijn beeld van zelfontdekkend leren  binnen het juiste tijdskader versterkt.
		\item Door eens met een jonger doelpubliek in aanraking te komen, kan ik mij heel even in de schoenen van een leerkracht lager onderwijs leerkracht plaatsen. Mijn respect voor hen om ruim twintig leerlingen te voeden en hun honger naar kennis te stillen is enkel gegroeid. In vergelijking met deze leerlingen zijn de jongeren aan wie ik les zal geven / geef heel vaak maar stil en moet je hen uit hun schulp sleuren. Hieruit wil ik leren om toch net die nieuwsgierigheid ook bij mijn doelpubliek los te trekken.
	\end{itemize}
	
	\subsection{Twee voorbeelden die aantonen dat de activiteiten een meer-waarde zijn voor leerlingen}
	\begin{itemize}
		\item De leerlingen hebben er zeker van genoten om bij te leren over een topic dat nieuw is voor hen. Optica is iets wat in ons dagelijks leven voorkomt, maar pas veel later in het onderwijs aan bod komt. Veel leerlingen droegen een bril en kregen op deze manier iets van die werking te zien. Tegelijkertijd werd optica verbonden met het heelal, een onderwerp dat tot de verbeelding van velen spreekt.
		\item Daarnaast konden de leerlingen ook zelf aan de slag met hun net opgedane kennis om zelf een telescoop te maken. Ze kregen materiaal en een kleine handleiding om die te maken. Daarna konden ze hun telescoop meteen testen door hun naam vanop een afstand met een getal proberen te linken. Ze vonden dit een zeer aangename opdracht.
	\end{itemize}
	
	\subsection{Bespreek hoe het komt dat bepaalde activiteiten geen echte meerwaarde hebben voor leerlingen en op welke manier deze aangepast kunnen worden om toch nog functioneel te zijn voor het leerproces van de leerlingen}
	De leerlingen krijgen hier ook informatie van de prof te verwerken die voor sommigen onder hen waarschijnlijk wat te ingewikkeld is. Dit komt omdat het publiek tussen 8 en 13 jaar is. Dit zou bijvoorbeeld opgelost kunnen worden door twee verschillende sessies uit te werken en de groep in twee op te splitsen. Dit zou echter ook het doel van de kinderuniversiteit veranderen, waarbij ze leerlingen over verschillende leeftijden eenzelfde onderwerp willen aanbieden.
	
	
	
	\subsection{Meso-activiteit 3: infodag KU Leuven campus Kortrijk Kulak}
	\subsubsection{Omschrijving van de activiteiten}
	De jaarlijkse infodag op Kulak vond dit jaar op woensdag 11 maart plaats. Hierbij voorzien alle richtingen een infomoment waarbij de toekomstige studenten vragen kunnen stellen, cursussen kunnen inkijken, projecten kunnen bekijken \ldots Mijn taken tijdens deze infodag zijn:
	\begin{itemize}
		\item alle leerlingen ontvangen in het onderwijslabo, samen met enkele collega's van de richtingen wiskunde en informatica. Leerlingen met vragen omtrent fysica of burgerlijk ingenieur krijg ik doorverwezen en op hun vragen antwoord ik zo goed mogelijk.
		\item Tussen 15u30-15u50 en 17u30-17u50 sta ik in de WeeTKelder de toekomstige studenten te woord omtrent  projectwerken voor de burgerlijk ingenieurs (`Probleemoplossen en ontwerpen, deel 2' en `Probleemoplossen en ontwerpen, deel 3'). Daarna neem ik de in fysica geïnteresseerden mee naar het aanpalende onderzoekslabo, waar er projecten binnen het vak `Grondslagen voor experimentele natuurkunde' tentoon staan en waar er over de integratie van het onderzoek van de doctorandi binnen dit vak gesproken wordt. Ik ben met, onder andere, deze drie vakken verbonden, waardoor ik hiervoor verantwoordelijk ben tijdens de infodag.
	\end{itemize}
	
	\subsection{Twee aspecten die ik voor mezelf geleerd heb}
	\begin{itemize}
		\item Laatstejaars leerlingen uit het middelbaar staan nog steeds sterk open voor wetenschap en innovatie. Hierdoor zijn ze sterk gedreven om aan de opleiding te beginnen en stellen ze interessante en inhoudelijke vragen.
		\item Zowel de toekomstige studenten als hun ouders reageren positief wanneer ik mijn uiteenzetting over het projectwerk en de integratie van onderzoek in ons onderwijs houd. Deze appreciatie is steeds aangenaam en leert me dat we hierop moeten blijven inzetten. 
	\end{itemize}
	
	
	\subsection{Twee voorbeelden die aantonen dat de activiteiten een meer-waarde zijn voor leerkrachten}
	\begin{itemize}
		\item Als goede voorbereiding op de infodag moet je niet enkel jouw vak kennen, maar moet je over de opleiding in het algemeen goed geïnformeerd zijn. Dit houdt in dat jij als `leerkracht' toch ook van de vakken van je collega's op de hoogte bent dankzij de infodag.
		\item Tijdens een infodag kom je al in contact met toekomstige studenten. Zo kan je sommigen onder hen al leren kennen en hun interesses hoog houden voor het vakgebied. Ik vind dat je als lesgever een gezonde, oprechte interesse in je studenten moet hebben. Dit kan je tijdens een infodag al tonen aan je toekomstige studenten.
	\end{itemize}
	
	
	\subsection{Twee voorbeelden die aantonen dat de activiteiten een meer-waarde zijn voor leerlingen}
	\begin{itemize}
		\item Tijdens deze infodag kan je leerlingen helpen om ofwel een beslissing te nemen in verband met hun studiekeuze ofwel om hen nog zekerder te maken van de keuze die ze al genomen hebben. Ze kunnen op dat moment vragen stellen aan zowel professoren, assistenten, doctorandi, medestudenten \ldots over alle mogelijke onderwerpen die bij het academische studentenleven komen kijken. Ze kunnen al eens de cursussen inkijken en vragen stellen omtrent de inhoud van bepaalde vakken.  
		\item Tegelijkertijd staat de dienst studentenvoorzieningen er ook en kunnen ze met hun meer studentikoze vragen zowel bij hen als bij hun toekomstige medestudenten terecht. 
	\end{itemize}
	
	\subsection{Bespreek hoe het komt dat bepaalde activiteiten geen echte meerwaarde hebben voor leerlingen en op welke manier deze aangepast kunnen worden om toch nog functioneel te zijn voor het leerproces van de leerlingen}
	Niet iedereen kan op één van de contactmomenten in de WeeTKelder aanwezig zijn. Daarom is het mogelijk dat toekomstige studenten de uitleg rond projectwerken niet te horen krijgen. Daarom zou ik willen voorstellen om dit frequenter te houden.
	
	
	\newpage
	\includepdf[scale = 0.8,pages = 3,pagecommand=
	\section{Evaluatiedocumenten vakmentor}
	\subsection{Evaluatiedocument mentor Kulak (LIO): eerste geobserveerde les }]{VerslagenDavid2}
	\includepdf[scale = 0.8,pages =4-5,pagecommand=]{VerslagenDavid2}
	
	
	\includepdf[scale = 0.8,pages = 7,pagecommand=\subsection{Evaluatiedocument mentor Kulak (LIO): tweede geobserveerde les (Les 7-9)}]{VerslagenDavid}
	\includepdf[scale = 0.8,pages =8-9,pagecommand=]{VerslagenDavid2}
	
	
	
	\includepdf[scale = 0.8,pages = 10,pagecommand=\subsection{Eindevaluatiedocument mentor Kulak (LIO)}]{VerslagenDavid2}
	
	
	\includepdf[scale = 0.8,pages = 1,pagecommand=\subsection{Evaluatiedocument mentor VISO: les 13-14}]{EvaluatieLes1314}
	\includepdf[scale = 0.8,pages =2-3,pagecommand=]{EvaluatieLes1314}
	
	
	\includepdf[scale = 0.8,pages = 1,pagecommand=\subsection{Evaluatiedocument mentor VISO: les 15}]{EvaluatieLes15}
	\includepdf[scale = 0.8,pages =2-3,pagecommand=]{EvaluatieLes15}
	
	
	\includepdf[scale = 0.8,pages = 1,pagecommand=\subsection{Evaluatiedocument mentor VISO: les 16-17}]{EvaluatieLes1617}
	\includepdf[scale = 0.8,pages =2-3,pagecommand=]{EvaluatieLes1617}
	
	
	\includepdf[scale = 0.8,pages = 1,pagecommand=\subsection{Evaluatiedocument mentor VISO: les 18}]{EvaluatieLes18}
	\includepdf[scale = 0.8,pages =2-3,pagecommand=]{EvaluatieLes18}
	
	
	\includepdf[scale = 0.8,pages = 1,pagecommand=\subsection{Evaluatiedocument mentor VISO: les 19-20}]{EvaluatieLes1920}
	\includepdf[scale = 0.8,pages =2-3,pagecommand=]{EvaluatieLes1920}
	
	
	
	\includepdf[scale = 0.8,pagecommand=\subsection{Eindevaluatiedocument mentor VISO}]{EindevaluatieVISO}

	
	
	
	
	\includepdf[scale = 0.9,pagecommand=\section{Evaluatiedocument klasbezoek stagebegeleider}]{27112019_Feedback}
	\includepdf[scale = 0.8,pagecommand=,pages=2-]{27112019_Feedback}
	
	
	\newpage
	\section{Eindreflectie}
	Stel een eindreflectie op waarin je volgende aspecten behandelt: 
	\begin{enumerate}
		\item Waren er factoren die bevorderend of belemmerend werkten m.b.t. het goed doorlopen van je stage? 
		\item Waarvoor had je graag bijkomende begeleiding gekregen van je vakmentoren? 
		\item Waarvoor had je graag bijkomende begeleiding gekregen van je stagebegeleider? 
		\item Bekijk aandachtig de acties die je in het begin van je stage opstelde in jouw POP. Ga na of je via de acties jouw leerdoelen hebt behaald. Verwijs heel duidelijk naar informatie in je portfolio waar en hoe je deze acties aan bod liet komen. 
		\item  Bestudeer nogmaals het opleidingsprofiel en de basiscompetenties van een leraar (link):  bespreek minstens 5 basiscompetenties die je succesvol hebt behaald tijdens het uitvoeren van je stage. 
	\end{enumerate} 
Jouw eindreflectie is maximaal drie A4-pagina’s lang.\newline\newline
	
Mijn stage in het VISO heeft vanwege de overmacht rond de pandemie spijtig geen slot gekregen. Het was de bedoeling dat ik nog drie lessen ging geven: een introductie rond het onderwerp van de transformator (1 lesuur) en het geven van een labo rond dit onderwerp (2 lesuren). Hierdoor heb ik van de leerlingen geen afscheid kunnen nemen en hen niet kunnen vragen naar hun ervaringen van mij als stageleerkracht. Om een vlottere beëindiging van mijn stage te verzorgen, hebben mijn vakmentor en ik via mail nog zaken geregeld. Ik heb het deel van de cursus dat bij mij lag ingescand zodat de leerlingen dit onderdeel nog konden krijgen van haar. Verder heb ik geen speciale factoren binnen mijn stage ervaren, die ofwel speciaal bevorderend ofwel belemmerend werkten.\newline

Ik heb positieve ervaringen met zowel mijn twee vakmentoren als met mijn stagebegeleidster. Ze hebben mij steeds met terechte feedback te rade gestaan, waaruit ik altijd heb kunnen putten om het in mijn volgende les toch net dat tikkeltje anders te doen dan ik oorspronkelijk van plan was. Op die manier heb ik de leerlingen beter in mijn lessen kunnen betrekken, beter kunnen begeleiden, ondersteunen \ldots{}  Tijdens deze stage heb ik ook een lessenreeks van acht (oorspronkelijk elf) lessen in het secundair onderwijs mogen geven. Dit benadert beter de realiteit dan losse lessen. Door een lessenreeks te mogen geven, toont de vakmentor ook het vertrouwen in mij als stagiair. Zij heeft ook aangegeven dat ze tevreden is over de lessenreeks die ik voor haar leerlingen voorzien heb.\newline
Binnen mijn LIO-context werd ik eerder vrij gelaten. Dit zijn lessen die ik al voor het vierde jaar aan de universiteit verzorg. Net zoals de voorbije drie jaar, heb ik ook dit jaar geprobeerd om mijn lessen te innoveren, met als doel dat de studenten een beter begrip rond fysica hebben. Ik heb doorheen de jaren ervaren dat oefenzittingen aan de universiteit niet op de `klassieke' (lees: meest veronderstelde) manier hoeven te gebeuren. Studenten verwachtten dit enigszins wel, dat je als lesgever aan de universiteit eerder een docerende vorm aanneemt. Wanneer ze echter in een vroeg stadium met een andere manier van lesgeven geconfronteerd werden, zette het merendeel van de groep zich snel in om actief mee te werken aan het lesgebeuren. Ik heb ze zowel in groep, per twee en vier, laten werken als met verbeterfiches, per twee, om te zorgen dat studenten vooral elkaar met de leerstof konden helpen. De studenten leren op die manier van elkaar en van hun eigen uitleg. De vakmentor liet mij op het einde van de lessenreeks weten dat de studenten het vak opnieuw positief geëvalueerd hebben en dat zij zeer positief staan ten opzichte van hun lesgevers, zowel op gebied van vakkennis als van het lesgebeuren (POP lesdoel 3). \newline\newline
Mijn stage in het middelbaar vond plaats in de VISO, Roeselare, waar ik toegepaste fysica aan het vijfde jaar Techniek-Wetenschappen mocht geven. Dit is mijn eerste ervaring met het technisch onderwijs geweest en eveneens ook mijn eerste ervaring met het geven meerdere lessen op verschillende dagen aan een klas in het middelbaar onderwijs. Hier heb ik vooral positieve ervaringen gehad, dankzij ondersteuning van de vakmentor. Haar feedback zorgde ervoor dat ik mijn volgende les telkens beter uitvoerde. Tegelijkertijd was ik ook in staat om te sleutelen aan elementen binnen mijn POP, door deze feedback. Ik heb tijdens deze stage mijn `voelsprieten' (POP lesdoel 1) beter ontwikkeld. Ik stelde de leerlingen vragen en betrok alle leerlingen bij mijn les, ook diegene die niet hun hand opstaken (zie feedback vakmentor VISO, bvb les 19-20). Dit lukte natuurlijk ook uitstekend vanwege de grootte van deze klas, aangezien er slechts acht leerlingen in de groep zitten. Ik heb ook wat verschillende werkvormen in verschillende lessen proberen te gebruiken, zoals een labo in les 13-14 (en wat eigenlijk les 22-23 had moeten zijn), maakten de leerlingen oefeningen in groep en verbeterden we die meestal klassikaal, maar ook via verbetersleutels (les 18). Wanneer er theorie besproken werd, dan kwam ik heel vaak terecht bij een onderwijsleergesprek. Hier kon ik pauzes laten, die de leerlingen aanzette tot actief meedenken omdat ik `weigerde' om het antwoord meteen zelf te zeggen (zie feedback les 2 stagebegeleidster). Tijdens deze lesfases liet ik de leerlingen aan het woord. Bij een correct antwoord herhaalde ik het nog eens om daarna soms nog eens door een andere leerling dit te laten herhalen. Bij een foutief antwoord ging ik verder in op de fout en stelde ik eerst de leerling in kwestie vragen opdat die leerling de fout zou inzien. Dit hielp om die leerling te laten inzien van zijn/haar fout. De leerkracht merkte wel op dat de stukken die ik gegeven heb, tot het moeilijkere deel van alle leerstof behoort en dat het moeilijk is om dit anders aan te pakken. Toch zou ik in mijn toekomst willen verder zoeken om ook deze moeilijke stukken vlot en actief aan te brengen.\newline In mijn beide stagesettings heb ik ook steeds de relevantie van de aangeleerde fysische begrippen te duiden binnen de context van onze leefwereld. De theorie was steeds heel algemeen, heel wiskundig besproken. Toch heb ik steeds de theorie gelinkt met situaties in de alledaagse wereld waarin die theorie aan bod komt (lesdoel 2). Dit ging dan van een dynamo tot een windmolen over een `kooi van Faraday' (waarbij ik de opmerking maakte dat dit een perfect klaslokaal zou zijn, aangezien er dan geen bereik van hun GSM zou zijn).  \newline
Op het zorgniveau van deze acht leerlingen, was er één persoon met verhoogde zorg (fase 1), vanwege anderstaligheid, en één leerling met uitbreiding van zorg (fase 2), vooral vanwege een Autisme spectrum stoornis. Ik vond het een ideaal moment om een eerste maal in praktijk in aanraking te komen met de verschillende facetten van het zorgspectrum binnen het lesgebeuren. Ik heb hier niet bewust naar oplossingen / alternatieven /... gezocht, maar de vakmentor vond dat ik uitstekend reageerde tijdens bepaalde situaties, terwijl ik sommige situaties vaak als niet-problematisch inschatte. Ik heb tijdens deze lessen toch wel een grondige basis praktijkervaring opgedaan wanneer het op een leerling met ASS aankomt. Dit heb ik als een enorme meerwaarde van mijn stage ervaren. \newline\newline
\newline
Tegelijkertijd zijn er nog heel wat werkpunten aan het licht gekomen. Zo is mijn bordschrift vaak warrig en verliezen de leerlingen soms het overzicht omdat ik dingen blijf bijschrijven. Hier zou ik in de toekomst verder aan blijven werken, door beter stil te staan bij mijn bordschema's en die op papier wat blijven uit te werken. Zo zal ik een beter overzicht creëren, iets wat ik ondervonden heb zowel ter voorbereiding van als tijdens mijn laatste lessen. Een tweede werkpunt is dat mijn taal iets meer verzorgd mag. Tijdens het uitleggen van bepaalde zaken is heel duidelijk dat ik van West-Vlaanderen kom. Op zich vind ik dat niet erg, dat er West-Vlaamse klanken aanwezig zijn in mijn taal, het is wie ik ben en draagt bij aan mijn authenticiteit. De leerkrachten waarvan ik destijds zelf het meest heb bijgeleerd konden namelijk ook niet verstoppen dat ze uit West-Vlaanderen komen. Tegelijkertijd moet ik er echter voor zorgen dat alles wat ik zeg voor iedereen begrijpbaar is, wat soms dreigt weg te vallen. Ik merkte wel dat ik er mij van bewust was wanneer ik met de anderstalige persoon in de klas van 5TW sprak en dat ik in die gesprekken bewust minder dialect sprak. Dit zou ik toch wel meer willen integreren als een natuurlijke houding in de toekomst van mijn lesgebeuren. \newline Een laatste iets waar ik aan blijven werken is de manier van overbrengen van bepaalde inhouden. Bij de gegeven lessenreeks heb ik veel tips van de vakmentor gekregen over hoe ik met dit thema beter kan omgaan indien ik het nog eens zou moeten geven. Deze tips wil ik ook meenemen naar andere onderwerpen en daar uitproberen. Dit zijn echter zaken die pas volledig vlot zullen lopen wanneer ik meer geroutineerd ben in het lesgeven van die bepaalde onderwerpen. \newline\newline
`Functioneel geheel 1 - De leraar als begeleider van leer- en ontwikkelingsprocessen' is een basiscompetentie die ik zeker bereikt heb tijdens mijn stage. Ik heb overleg gepleegd met de vakmentor om de zorgprofielen van mijn leerlingen te kennen om van daaruit te zorgen dat de kwaliteit van hun leren hoog is. Ook andere zaken, die ik onder andere hierboven heb aangehaald, tonen dat deze basiscompetentie zeker bereikt is.\newline Tijdens mijn stage was ik zeker ook vlot tegenover collega's van de vakmentor en stelde ik hen ook vragen in verband met materiaal dat ik zocht. Daarnaast overlegde ik vaak met de vakmentor over de gegeven lessen, waaruit ik de feedback gebruikte om in de volgende lessen mee aan de slag te gaan. Op mijn werk overleg ik vaak met de prof en de verantwoordelijke voor de practica. We bespreken nieuwe oefeningen die ik ontwikkeld heb, werken  ideeën voor nieuwe practica uit, die ik dan programmeer, evalueren de evaluatie die de studenten over het vak ingevuld hebben \ldots{} Hieruit kan ik concluderen dat ook `Functioneel geheel 7 - De leraar als lid van een schoolteam' bereikt is.\newline 
Tijdens mijn beide stages, stelden de leerlingen / studenten vragen die moeilijk zijn om op inzichtelijk vlak uit te leggen. Tijdens zo'n momenten merkte ik dat het goed is dat ik inhoudelijk een expert ben, wanneer het op fysica en wiskunde aankomt (`Functioneel geheel 3 - De leraar als inhoudelijk expert'). Het is van essentieel belang dat de lerenden een moeilijk uitziende wiskundige vergelijking niet enkel kunnen uitrekenen, maar dat ze vooral begrijpen wat ieder onderdeel betekent. Hiervoor moet je zelf, als leraar, natuurlijk het werkkader volledig begrijpen.\newline
Op mijn werk aan de universiteit heb ik contacten gehad  met bedrijven en instanties. Enerzijds heb ik, wanneer de Kulak de gastuniversiteit was voor de finale van de fysica olympiade, één van de praktische finaleproeven uitgewerkt. Anderzijds heb ik ook contact met National Instruments om ervoor te zorgen dat we aan studenten practica kunnen aanbieden waarin toestellen gebruikt worden die van hoge kwaliteit zijn. Zij bieden naast onderzoeksmateriaal ook speciaal vervaardigd onderwijsmateriaal aan waarmee we ondertussen verschillende studentengroepen aan de slag laten gaan. Jaarlijks pas ik nog steeds enkele practica aan, zodat meer studentengroepen (burgerlijk ingenieur, fysica, wiskunde, handelsingenieur, informatici, chemie, bio-ingenieur, biologie, biomedische wetenschappen) met technologie die vandaag op de markt is, kunnen kennis maken. De meeste studenten die een onderzoekende ingenieursvacature zullen invullen, zullen met gelijkaardige systemen in contact komen. Op deze manier zetten we in om een brug te slaan tussen onderwijs en de arbeidsmarkt.  Op deze manier vervul ik ook `Functioneel geheel 8 - De leraar als partner van externen' en `Functioneel geheel 5 - De leraar als innovator - de leraar als onderzoeker'.\newline
Een ander aspect van het laatst genoemde functioneel geheel is dat wij als universiteit na ieder semester ondervragen over de kwaliteit van de vakken die ze in dat semester gevolgd hebben. Als onderwijsteam kunnen we zo onze eigen werking aanpassen naar wat de studenten noodzakelijk vinden. Zo kunnen we ons eigen functioneren evalueren en aan de slag gaan om ons lesgebeuren aan te passen naar volgend jaar toe.\newline 
\newpage












\section{Voorbereiding eindassessment}

Om het eindassessment voor te bereiden, kan je gebruik maken van volgende vragen:
\begin{itemize}
	\item Lees jouw eindreflectie goed na en bekijk jouw leerdoelen en uitgewerkte acties. Recapituleer hoe je de stage hebt ervaren. Waarom moet een directeur jou als leerkracht aanwerven? Wat heb jij een schoolteam te bieden? Waar zie je nog uitdagingen voor jezelf? 
	\item Waar heb je nog aanvangsbegeleiding nodig en wie kan jou daarbij helpen (toon je inzicht in vakgroep- en schoolwerking aan)?
	\item Hoe heb je de lerarenopleiding in het algemeen ervaren? Wat vond je positief? Wat heb je gemist tijdens de opleiding?
\end{itemize} 

Zoals hierboven reeds besproken, heb ik mijn stage als positief ervaren. Tijdens mijn stage heb ik ervaren dat ik binnen het secundair onderwijs liever aan een tso richting zou lesgeven dan aan een aso richting. Binnen een tso richting heb ik weerwoord van mijn leerlingen, durven ze zelf vragen te stellen, staan ze open om zelf aan de slag te gaan. Tijdens mijn vorige stage in het aso heb ik dit niet kunnen ervaren.\newline\newline
Dit vind ik meteen één van mijn pluspunten: ik wil echt lesgeven om de leerlingen iets bij te leren; ze de wereld rondom hen beter te laten begrijpen; ze kritisch te laten nadenken, niet enkel over wetenschappen, maar over alle onderwerpen. Als leerkracht wil ik dat bijbrengen aan mijn toekomstige leerlingen. \newline
Als lid van het vakteam, wil ik mijn inbreng doen om te helpen bij het ontwikkelen van nieuwe wetenschapsprojecten en -initiatieven. Als fysicus/sterrenkundige zou ik thema's kunnen maken die iedereen aanspreken, aangezien er heel wat fysica achter alledaagse handelingen zit. Als lid van het schoolteam, zou ik een meerwaarde zijn bij de begeleiding van de GIP's van de leerlingen. Ik begeleid al vier jaar verschillende projectwerken van ingenieursstudenten die één of twee jaar ouder zijn dan de laatstejaarsleerlingen in het secundair. Studenten begeleiden bij een technische probleemstelling brengt aangename uitdagingen met zich mee. Een steeds variërende en evoluerende uitdaging die ik met veel plezier aanga! Die ervaring zou een meerwaarde zijn in mijn rol als fysica leerkracht binnen het GIPteam in een technische school bij de meer theoretische opleidingen als bijvoorbeeld industriële wetenschappen, techniek-wetenschappen, elektriciteit-elektronica, elektromechanica en vliegtuigtechnieken. Tegelijkertijd zal ik van deze projecten veel technische inzichten verwerven, die ik van de leerlingen zelf en collega's binnen het GIPteam aangereikt zal krijgen, iets wat ik als een meerwaarde beschouw.\newline\newline
De mentor van de aanvangsbegeleiding zal mij en mijn mede startende collega's hulp op schoolniveau aanbieden. We zullen enerzijds kennismaken met de werking van de school op vlak van regelgeving en manier van lesgeven. Anderzijds zullen we hier ook de boodschap die de school wil dragen uitvoerig kunnen bespreken. Binnen de werking van de  aanvangsbegeleiding kunnen we ook elkaars lessen bezoeken om hieruit te leren hoe we op klasniveau met bepaalde groepen kunnen omgaan. Dit kan handig zijn voor klassen die ik ook heb, als bij klassen waar ik helemaal niet kom. Naast de aanvangsbegeleiding is het steeds aangenaam om een `senior' vakleerkracht als begeleider te hebben, die je kan bijstaan om manieren te bespreken om bepaalde vakinhouden te geven. Die persoon kan je ook introduceren in het aanwezige materiaal voor labo's en welke materialen in de nabije toekomst nuttig zouden zijn.\newline\newline
Ik heb anderhalf jaar in het CVO de opleiding gevolgd. Daar vond ik het een meerwaarde dat je microteaching moest doen aan medecursisten die allemaal een verschillende achtergrond hebben. Dat vond ik een heel sterk pluspunt, omdat dit de realiteit beter benadert. Als leerkracht geef je les over inhoud die nog ongekend is voor je leerlingen. Door microteaching te doen aan een diverse groep boots je deze situatie na. Voor wetenschapsvakken kom je dan net vaak zelfs in situaties terecht waar je medecursisten eigenlijk zelfs de basis niet (meer) kennen en waar je als leerkracht getriggerd wordt om echt de basis kort en bondig nog eens aan te brengen. Dit komt dan in strijd met de `realistische' lesvoorbereiding die je voor een bepaalde klas voorbereid hebt. Op dat moment moet je als leerkracht ingrijpen, want je wilt niet dat er iemand achterop hinkt doordat die de basis niet (meer) kent, maar je wilt toch nog je lesdoelen bereiken. Dit vond ik steeds uitdagend om te realiseren. Als leerkracht zal je in de toekomst ook in contact komen met leerlingen die na een vakantie terug naar de les komen en alles vergeten zijn, een situatie waarmee ik deze lessen soms kon vergelijken. \newline
De praktijkervaring die je opdoet tijdens je stages is zeker de beste leerschool. Ik vond het aangenaam dat ik mijn stages over verschillende niveaus heb kunnen ervaren. Zo heb ik mijn stage bij DCO in het aso mogen geven en heb ik mijn stagelessen bij DCS op de universiteit (LIO) en in het tso mogen geven. Die ervaring heeft mij ook ervan overtuigd dat ik liever in het tso dan het aso les wil geven. De leerlingen waarmee ik in contact ben mogen komen, zijn actiever in het lesgebeuren dan de aso leerlingen waarmee ik in contact gekomen ben. Dit heb ik ook ervaren bij de fysica experimentendagen die we voor laatstejaars aan Kulak voorzien. De aso leerlingen wiskunde-wetenschappen die dit volgden waren niet actief tijdens die sessies, het tegenovergestelde van de tso-richting die we mochten ontvangen. Ook de technisch-wetenschappelijke eindwerken die in de tso richtingen door de leerlingen gerealiseerd worden, spreken mij zeer sterk aan. \newline
Wat ik gemist heb tijdens mijn opleiding is wat vakdidactische ondersteuning. De ideale situatie zou er één zijn waarin de situatie van de vroegere CVO-opleiding gecombineerd is met de vakdidactische ondersteuning die binnen de universitaire opleiding aanwezig is. Ik kan over de universitaire opleiding echter weinig zeggen, gezien ik in deze context niet zoveel ervaring heb. \newline 
In beide systemen, CVO en Kulak, heb ik enkel positieve ervaringen met mijn stagebegeleiding. Zij hebben mij steeds bijgestaan met terechte, constructieve feedback die ik in mijn volgende stagelessen meteen kon uitproberen.




	
	
	
\end{document}