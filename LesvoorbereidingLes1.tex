% !TeX root = Stageportfolio.tex



\begin{landscape}
	\section{Lesvoorbereidingen en bijhorende media}
	\begin{tabularx}{1.56\textwidth}{|p{0.55\textwidth}|X|}\hline
		\textbf{Administratieve gegevens}\newline\newline
		Kevin Truyaert\newline\newline
		Universiteit\newline
		Handelsingenieur, 2de fase\newline
		Leerplannummer: De inhoud is terug te vinden op de ECTS fiche: \href{https://onderwijsaanbod.kuleuven.be/syllabi/n/D0W55AN.htm}{https://onderwijsaanbod.kuleuven.be/syllabi/n /D0W55AN.htm} \newline
		Lesonderwerp: `Oefenzitting elektromagnetisme: wat zijn de relaties tussen de elektrische kracht, de  elektrische potentiaal, de elektrische flux en de elektrische capaciteit' & \textbf{Doelstellingen}\newline
		\newline\newline 
		\underline{Leerplandoelen}\newline - Elektriciteit: elektrische lading, elektrisch veld (wetten van Coulomb en Gauss), elektrische flux, elektrische potentiaal, energie in een elektrisch veld \newline\newline
		\underline{Lesdoelen}\newline
		\vspace{-0.5cm}
		\begin{enumerate}
			\item De studenten kunnen via de wet van Coulomb de elektrostatische kracht tussen ladingen berekenen.
			\item De studenten kunnen de relatie tussen de elektrostatische kracht, het elektrisch veld en een lading toepassen in een probleem.
			\item De studenten kunnen de elektrostatische kracht binnen de tweede wet van Newton herkennen.
			\item De studenten kunnen een Gaussoppervlak in een situatie opstellen.
			\item De studenten zijn in staat om de elektrische flux te bepalen met gebruik van een Gaussoppervlak.
			\item De studenten kunnen het elektrisch veld en de elektrische flux van een boloppervlak in functie van de afstand afleiden.
			\item De studenten kunnen het elektrisch veld en de elektrische flux van een opgevulde, geleidende bol in functie van de afstand afleiden.
			\item De studenten kunnen.
		\end{enumerate} \\\hline
	\end{tabularx}


	\begin{tabularx}{1.56\textwidth}{|p{0.55\textwidth}|X|}
		\hline
		\multirow{2}{0.55\textwidth}{\textbf{Beginsituatie}\newline De studenten hebben de theorie rond de  begrippen van `Elektrisch veld', `Elektrische potentiaal', `Elektrische flux' en de wet van Coulomb in de week van 12-15 november gezien, twee weken voor de oefenzitting. Hierdoor zullen ze al tijd gehad hebben om de theorie te bekijken, wat aangemoedigd wordt door het maken van een voorbereidende opdracht die ik de week voor de oefenzitting op Toledo plaats.\newline\newline De minderheid van de studenten heeft  interesse bij mechanica, het eerste deel van de cursus, getoond. Het gedeelte over elektromagnetisme ervaren ze meestal interessanter. Er zijn 28 studenten die deze sessie volgen, maar gemiddeld gezien zijn er 25 studenten aanwezig geweest bij de voorbije lessen.\newline\newline Het lokaal kan 30 studenten plaatsen. Er is een dubbel krijtbord ter beschikking en de mogelijkheid tot projectie. Wanneer er geprojecteerd wordt, hangt het projectiescherm grotendeels over beide borden.  }& \textbf{Acties}\newline  - Om de studenten te stimuleren om zelf aan de slag te gaan, wil ik hen in groepjes van vier tot zes studenten aan de slag zetten. Hierdoor kan ik gerichtere feedback geven, aangezien de studenten onderling elkaar kunnen aanzetten tot het vinden van oplossingen. Naast de helpende rol, kan ik ook interacties tussen de studenten onderling volgen en inspringen waar nodig: ofwel bij het maken van een fout, of wanneer ik hun uiteenzetting zeer goed vind en er nog dieper op in wil gaan. Dit wil ik steeds vanuit het onderwijsleergesprek proberen te realiseren.  \newline\newline
		- Bij het begin van de les overloop ik nog even de theorie rond de elektrische grootheden en hun onderlinge relaties. Dit zet ik op één van de twee krijtborden en laat ik gedurende de hele les staan. Zo kunnen de studenten steeds makkelijk teruggrijpen naar de theorie. \newline\newline
		- Ik werk niet met projectie, maar noteer alles op het bord, omdat het projectiescherm voor zo goed als beide borden hangt. Hierdoor houd ik een tempo aan waarop de studenten makkelijker kunnen volgen, doordat ik alles zelf ook neerschrijf.  
		
		\\ \cline{2-2}
		  & \textbf{Bronnen}\begin{itemize}
		  	\item Dudal, D., Temmerman, E., Truyaert, K., Heymans, S. (2019). Slides conceptuele natuurkunde
		  	\item Dudal, D., Temmerman, E., Truyaert, K., Heymans, S. (2019). Oefeningenbundel conceptuele natuurkunde
		  	\item Giancoli, D. C. (2008). Physics for scientists and engineers. Pearson Education International.
		  \end{itemize}\\ \hline
	\end{tabularx}


\newpage
	
	
	
	\begin{tabularx}{1.56\textwidth}{|p{1.5cm}|p{6cm}|X|p{4cm}|}
		\hline
		\textbf{Nr. lesdoel } & \textbf{Inhoud (timing)}  & \textbf{Organisatie } & \textbf{Media } \\ \hline
		&\underline{Inhoudelijke titel (timing)}
	    \textcolor{gray}{(Naast een inhoudelijke titel en de timing, noteer je kort en samenvattend de kerninhoud van de lesfase; uitgebreide informatie/oefeningen/… neem je op in de uitgewerkte media [verwijzen!])}
	    &  \textcolor{gray}{(Naast de benaming van de specifieke werkvorm [bv. placemat-oefening/basis-expertengroep/… en dus níet groepswerk], noteer je kernachtig het organisatorisch verloop van de lesfase. Noteer eveneens belangrijke vragen die je wil stellen.) }
		& 
		\\ \hline
	\end{tabularx}
	
	
	
	
	
	
	
	
\end{landscape}