% !TeX root = Stageportfolio.tex



\begin{landscape}
	\subsubsection{Les 16}
	\begin{tabularx}{1.56\textwidth}{|p{0.35\textwidth}|X|}\hline
		\textbf{Administratieve gegevens}\newline\newline
		Kevin Truyaert\newline\newline
		technisch secundair onderwijs\newline
		3e graad, 1ste jaar, Techniek-Wetenschappen\newline
		VVKSO: \href{http://ond.vvkso-ict.com/leerplannen/doc/Toegepaste\%20fysica-2014-041.pdf}{http://ond.vvkso-ict.com/leerplannen /doc/Toegepaste\%20fysica-2014-041.pdf} \newline
		\underline{Lesonderwerp}:\newline Wet van Lenz \& algemene inductiewet: Faraday-Lenz + oefeningen & \textbf{Doelstellingen}
		\begin{itemize}[itemsep=0.08\baselineskip]
			\item B27: Fluxverandering als oorzaak van inductiespanning toelichten
			\item B28: Met behulp van de wet van Lenz de zin van de inductiespanning vinden
			\item B29: De algemene inductiewet hanteren.
		\end{itemize}
		\underline{Lesdoelen}\newline
		\vspace{-0.75cm}
		\begin{enumerate}[itemsep=0.08\baselineskip]
			\item De leerlingen kunnen de zin van de inductiestroom bepalen.
			\item De leerlingen kunnen de wet van Lenz verwoorden.
			\item De leerlingen zien de samenhang tussen de wet van Faraday en de wet van Lenz in.
			\item De leerlingen kennen de wet van Faraday-Lenz.
			\item De leerlingen kunnen de wet van Faraday-Lenz op een rechte, bewegende geleider toepassen.
			\item De leerlingen kennen de relatie tussen inductiespanning, magnetisch veld, lengte van de geleider, snelheid van de geleider en aantal windingen.
			\item De leerlingen kunnen de algemene inductiewet tijdens oefeningen hanteren.
		\end{enumerate} \\\hline
	\end{tabularx}\vfill \textcolor{white}{.} 


	\begin{tabularx}{1.56\textwidth}{|p{0.55\textwidth}|X|}
		\hline
		\multirow{2}{0.55\textwidth}{\textbf{Beginsituatie}\newline  
		Er zijn acht leerlingen binnen 5TW. Er heerst een algemene klassfeer. De leerlingen hebben al theorie gekregen  rond en oefeningen gemaakt op de magnetische krachtwerking. \newline\newline De leerlingen hebben vorige week de wet van Faraday gezien. Hiermee kunnen ze de grootte van de inductiespanning bepalen. Ook met de begrippen flux en fluxverandering zijn ze gekend. \newline\newline NOG AANVULLEN MET LERAARKENMERKEN.} & \textbf{Acties}\newline\newline 
		- \GreenHighlight{Via demo's wil ik bepaalde onderwerpen starten.}{9cm}	Op die manier kan ik de interesse van de leerlingen wekken en kan ik fysische wetmatigheden hen effectief aantonen. Zo kunnen leerlingen op een klassikale manier zelfstandig dingen ontdekken.	 \newline\newline 
		- Ik wil oefeningen op zo'n wijze brengen dat ze steeds dezelfde structuur hebben. Die structuur bouw ik eerst samen met de leerlingen op, om ze daarna zelfstandig aan de slag te laten gaan met oefeningen die steeds wat complexer worden. \PinkHighlight{Tijdens het zelfstandig maken van de oefeningen probeer ik toch zeker}{13cm} \PinkHighlight{de zwakkere leerlingen in de gaten te houden en hen individueler te coachen bij het}{15cm} \PinkHighlight{maken van oefeningen.}{5cm}
		\newline\newline\newline\newline\newline\newline
		
		\\ \cline{2-2}
		  & \textbf{Bronnen}\begin{itemize}
		  	\item Schramme, S. (2018) De stroombalans, labo magnetisme 4
		  	\item Frederiksen (2014), Current Balance 4565.00
		  	\item Giancoli, D. C. (2008). Physics for scientists and engineers. Pearson Education International.
		  \end{itemize}\\ \hline
	\end{tabularx}


\newpage
	
	\begin{tabularx}{1.56\textwidth}{|p{1.5cm}|p{8cm}|X|p{4cm}|}
		\hline
		\textbf{Nr. lesdoel } & \textbf{Inhoud (timing)}  & \textbf{Organisatie } & \textbf{Media } \\ \hline
		1\newline\newline 2&\underline{De wet van Lenz (20 minuten)}\newline
			De leerlingen bepalen de zin van de inductiestroom  aan de hand van een demo.
		&  \underline{Demo + Onderwijsleergesprek}\newline 
			Doormiddel van een elektromagneet en een spoel met een weekijzeren kern, wordt een stroom geïnduceerd in een ring die rond de weekijzeren kern hangt. De leerlingen zullen die zien bewegen. Samen leiden we af hoe de inductiestroom loopt, op basis van de beweging van de ring t.o.v. het magnetisch veld van de spoel. Zo worden pagina's 11 en 12 in de bundel aangevuld, aan de hand van wat de leerlingen observeren wat er met de ring gebeurt. Daarna besluiten we met de definitie van de wet van Lenz. Ik zorg dat verschillende leerlingen deze eens in eigen woorden ook zeggen, omdat dit een belangrijk gegeven is voor de toepassingen die in hoofdstuk 6 aan bod komen. 
		&   Cursus hoofdstuk 5 p11-12\newline\newline Krijtbord \newline\newline Elekromagneet, weekijzeren kern, hangende ring
		\\ \hline
	\end{tabularx}\vspace{5mm}

	
\begin{tabularx}{1.56\textwidth}{|p{1.5cm}|p{8cm}|X|p{4cm}|}
	\hline
	\textbf{Nr. lesdoel } & \textbf{Inhoud (timing)}  & \textbf{Organisatie } & \textbf{Media } \\ \hline
	3\newline\newline 4\newline\newline &\underline{De algemene inductiewet:} \underline{Faraday-Lenz (5 minuten)}\newline
	We kunnen de wet van Faraday-Lenz besluiten als combinatie van de wet van Faraday en de wet van Lenz.
	&  \underline{Onderwijsleergesprek}\newline 
	De algemene inductiewet schrijf ik nu op bord. Ik vraag aan de leerlingen om ieder deel te verklaren.
	&   Cursus hoofdstuk 5 p13\newline\newline Krijtbord 
	\\ \hline
\end{tabularx}\vspace{5mm}


\begin{tabularx}{1.56\textwidth}{|p{1.5cm}|p{8cm}|X|p{4cm}|}
	\hline
	\textbf{Nr. lesdoel } & \textbf{Inhoud (timing)}  & \textbf{Organisatie } & \textbf{Media } \\ \hline
	5\newline\newline 6&\underline{Faraday-Lenz op een rechte, bewegende} \underline{geleider (5 minuten)}\newline
	Toepassen van Faraday-Lenz op een bewegende geleider.
	&  \underline{Onderwijsleergesprek}\newline 
	Ik schets de situatie van een bewegende, rechte geleider die aan spanningsmeter verbonden is. We leiden de spanning die door de beweging geïnduceerd is af, door middel van de gekende formules. Ik laat de leerlingen hier zelfstandig mee starten, gezien de verschillende stappen al op hun blad aanwezig zijn, waarna ik inpik.
	&   Cursus hoofdstuk 5 p13\newline\newline Krijtbord 
	\\ \hline
\end{tabularx}\vspace{5mm}



\begin{tabularx}{1.56\textwidth}{|p{1.5cm}|p{8cm}|X|p{4cm}|}
	\hline
	\textbf{Nr. lesdoel } & \textbf{Inhoud (timing)}  & \textbf{Organisatie } & \textbf{Media } \\ \hline
    1\newline\newline 4 \newline\newline 7& \underline{Faraday-Lenz: oefeningen (18 minuten)}\newline
    De leerlingen maken oefeningen op Faraday-Lenz.	
	&  \underline{Oefeningen + onderwijsleergesprek}\newline 
	Ik vraag aan de leerlingen om zelf even oefening 1 te maken. Hiervoor moeten ze gebruik maken van de wet van Lenz. Ik treed na een minuut in interactie met de leerlingen om mij het antwoord op de vragen te geven. Daarna maak ik, door middel van vraagstelling aan de leerlingen, oefening 2. Ik bouw alle stappen op die ze bij dit soort oefeningen zullen moeten doen. Daarna maken ze zelf oefeningen 3, 4 en 5.
	&  Cursus hoofdstuk 5 p14-15\newline\newline Krijtbord
	\\ \hline
\end{tabularx}\vspace{5mm}


\begin{tabularx}{1.56\textwidth}{|p{1.5cm}|p{8cm}|X|p{4cm}|}
	\hline
	\textbf{Nr. lesdoel } & \textbf{Inhoud (timing)}  & \textbf{Organisatie } & \textbf{Media } \\ \hline
	& \underline{Slot (2 minuten)}\newline
	Ik herhaal nog even kort de wet van Lenz en de algemene inductiewet. Deze zullen belangrijk zijn bij volgende lessen gezien die nog meer oefeningen hierover bevatten en er ook toepassingen van magnetische inductie besproken zullen worden.	
	&  \underline{Onderwijsleergesprek + vertellen}\newline 
	&  Cursus hoofdstuk 5 \newline\newline Krijtbord
	\\ \hline
\end{tabularx}

	
\end{landscape}


%\subsection*{Bijlage 5.1: slides introductie}

%
%\subsection*{Bijlage 1.2: bordschema theorie}
%\begin{center}
%	\includegraphics[width=0.9\textwidth]{Bord1a}
%\includegraphics[width=0.9\textwidth]{Bord1b}
%\end{center}
%\newpage
%
%
%\includepdf[scale = 0.8,pages = 17,pagecommand=\subsection*{Bijlage 1.3: opgeloste oefeningen}]{Observaties_OpgelosteOef}
%\includepdf[scale = 0.8,pages =18-20,pagecommand=]{Observaties_OpgelosteOef}
%
%
%
%\includepdf[scale = 0.95,pages = 1,pagecommand=\subsection*{Bijlage 1.4: oefeningenbundel elektromagnetisme}]{OefeningenBundel}
%\includepdf[scale = 0.95,pages =2-,pagecommand=]{OefeningenBundel}