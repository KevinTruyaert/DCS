% !TeX root = Stageportfolio.tex



\begin{landscape}
	
	\subsection{Les 6-8}
	\begin{tabularx}{1.56\textwidth}{|p{0.55\textwidth}|X|}\hline
		\textbf{Administratieve gegevens}\newline\newline
		Kevin Truyaert\newline\newline
		Universiteit\newline
		Handelsingenieur, 2de fase\newline
		\underline{ECTS-fiche}: De inhoud is terug te vinden op de ECTS fiche: \href{https://onderwijsaanbod.kuleuven.be/syllabi/n/D0W55AN.htm}{https://onderwijsaanbod.kuleuven.be/syllabi/n /D0W55AN.htm} \newline
		\underline{Lesonderwerp}: `DC netwerken met weerstanden' & \textbf{Doelstellingen}\newline\vspace{0.5cm}
		\underline{Punt op de ECTS-fiche}
		\vspace{-0.5cm}\newline  - DC netwerken, wetten van Kirchhoff, elektrische meettoestellen \newline - toepassing: elektrische veiligheid en elektrische huisinstallatie \newline
		\underline{Lesdoelen}\newline
		\vspace{-0.5cm}
		\begin{enumerate}[itemsep=0.08\baselineskip]
			\item De studenten kennen de wetten van Kirchhoff.
			\item De studenten kunnen de wetten van Kirchhoff conceptueel uitleggen.
			\item De studenten kunnen de wetten van Kirchhoff opstellen voor een open netwerk met bronnen en weerstanden.
			\item De studenten kunnen de wetten van Kirchhoff opstellen voor een gesloten netwerk met bronnen en weerstanden.
			\item De studenten kunnen interpreteren dat een voltmeter gebruiken zorgt voor een wijziging in de spanningsval over een component. 
		    \item De studenten kunnen in groep over de oefening discussiëren en samen oplossingsgericht werken.
		\end{enumerate} \\\hline
	\multicolumn{2}{c}{ }\\
	\multicolumn{2}{c}{ }\\
	\multicolumn{2}{c}{ }\\
	\multicolumn{2}{c}{ }\\
	\multicolumn{2}{c}{ }\\
	\multicolumn{2}{c}{ }\\
	\multicolumn{2}{c}{ }\\
	\multicolumn{2}{c}{ }\\
	\multicolumn{2}{c}{ }\\
	\multicolumn{2}{c}{ }\\
	\end{tabularx}


	\begin{tabularx}{1.56\textwidth}{|p{0.55\textwidth}|X|}
		\hline
		\multirow{2}{0.55\textwidth}{\textbf{Beginsituatie}\newline De studenten hebben de theorie rond de wetten van Kirchhoff drie weken voor de oefenzitting gezien. Hierdoor zullen ze al tijd gehad hebben om de theorie te bekijken. Rond deze tijd hebben de studenten echter meerdere deadlines voor andere vakken en een examen Frans. Hierdoor plaats ik geen voorbereidende oefening online, maar vraag ik hen om enkel de wetten van Kirchhoff nog eens goed te bekijken. \newline\newline Er zijn 28 studenten die deze sessie volgen, maar vorige sessie waren slechts 18 studenten aanwezig. \newline\newline Het lokaal kan 30 studenten plaatsen. Ik splits de groep in zeven tafels van vier personen. Er is een dubbel krijtbord ter beschikking en de mogelijkheid tot projectie. Wanneer er geprojecteerd wordt, hangt het projectiescherm grotendeels over beide borden.  }& \textbf{Acties}\newline  - Net zoals tijdens de eerste lessenreeks, wil ik de studenten in \GreenHighlight{groepjes van vier studenten}{5cm} aan de slag zetten. Als examenvraag stel ik namelijk een oefening op rond de wetten van Kirchhoff, die aansluit bij wat ze deze en volgende les te zien krijgen. Ik vind het van essentieel belang dat ze de wetten van Kirchhoff niet allen goed en veel kunnen oefenen, maar dat ze die ook conceptueel begrijpen. Bij de eerste lessenreeks merkte ik op dat ik op deze manier gerichtere feedback aan de studenten kon geven. Ik ervoer ook dat ze gemotiveerd waren om per twee `beter' te doen dan hun overburen, terwijl ze toch steevast elkaar hielpen wanneer de andere vast zaten. Ik wil hier opnieuw een steunende rol spelen tijdens hun leer- en ervaringsproces.  \newline\newline
		
		- Bij het begin van de les overloop ik samen met de studenten de wetten van Kirchhoff. Zij reiken mij de twee wetten aan, die ik op het bord neerschrijf. Verder noteer ik ook samen met hen een stappenplan om dit soort oefeningen op te lossen. Dit laat ik op het bord staan. Zo kunnen de studenten steeds makkelijk teruggrijpen naar de theorie. \newline\newline
		- Ik werk niet met projectie, maar noteer alles op het bord, omdat het projectiescherm voor zo goed als beide borden hangt. Hierdoor houd ik een tempo aan waarop de studenten makkelijker kunnen volgen, doordat ik alles zelf ook neerschrijf.  
		
		\\ \cline{2-2}
		  & \textbf{Bronnen}\begin{itemize}
		  	\item Dudal, D., Temmerman, E., Truyaert, K., Heymans, S. (2019). Slides conceptuele natuurkunde
		  	\item Dudal, D., Temmerman, E., Truyaert, K., Heymans, S. (2019). Oefeningenbundel conceptuele natuurkunde
		  	\item Giancoli, D. C. (2008). Physics for scientists and engineers. Pearson Education International.
		  \end{itemize}\\ \hline
	\end{tabularx}


\newpage



\newpage

\begin{tabularx}{1.56\textwidth}{|p{1.5cm}|p{6cm}|X|p{3cm}|}
	\hline
	\textbf{Nr. lesdoel } & \textbf{Inhoud (timing)}  & \textbf{Organisatie } & \textbf{Media } \\ \hline
	1\newline 2 &\underline{Herhaling theorie (20 minuten)}\newline
	De theorie rond de wetten van Kirchhoff worden door de studenten aangereikt. Zij interpreteren ook wat de vergelijkingen voorstellen en delen dit met hun medestudenten. Hierna bouw ik samen met de studenten een stappenplan op om dit soort oefeningen aan te pakken. We bespreken ook nog kort even welke voorwaarden voldaan moeten zijn om een stroom te hebben (gesloten kring, geen condensatoren). 
	&  \underline{Onderwijsleergesprek}\newline 
	Ik start deze les met aan de studenten te vragen om mij de twee wetten van Kirchhoff uit te leggen. Ik noteer de wiskunde vertaling hiervan op bord. Ik probeer verschillende studenten aan het woord te laten.\newline Hierna stel ik samen met de studenten een stappenplan op om DC schakelingen te kunnen interpreteren. Ik vermoed dat de studenten dit zich niet meer goed zullen herinneren vanuit de theorieles. Daardoor zal ik zelf eerst een hint per stap geven of de stap(pen) zelf op het bord zetten, waarna ik telkens nog eens een student aan het woord laat om deze stap uit te leggen in eigen woorden. Hierna schets ik een kleine kring op het bord waar we dit klassikaal op toepassen.\newline
	Ik focus mij ook hier weer op het correct interpreteren van beide vergelijkingen. Dit is goed mogelijk door een vergelijking te maken waarbij de stroom een rij mensen of een rij auto's is en een spanningsverschil een helling. De eerste wet wordt dan  dat je bij ieder kruispunt slecht één richting kan kiezen waardoor het totaal aantal inkomende mensen/auto's hetzelfde moet zijn aan het totaal vertrekkende auto's. De tweede wet van Kirchhoff stelt voor dat je in iedere kring op hetzelfde niveau moet terugkomen: als je een kring doorlopen hebt, dan ben je terug op dezelfde hoogte.
	\newline 
	Hierna noteer ik de oefeningen op bord die gemaakt kunnen worden. Dit zijn oefeningen 66, 65, 69 en 71 in die volgorde. Ik verwacht dat de eerste drie oefeningen door iedereen gemaakt kunnen worden en de laatste door de betere studenten.\newline Ik zal de nadruk tijdens deze les vooral leggen op het zelfstandig inoefenen van dit soort oefeningen. Na het stappenplan op het bord genoteerd te hebben en met een minimaal voorbeeld gelinkt te hebben, heb ik uit ervaring van de voorbije jaren gemerkt dat de studenten er geen meerwaarde aan hebben om nog eerst een extra oefening te maken. Daarom laat ik hen meteen aan de slag gaan met de oefeningenreeks.
	& Krijtbord (Bordschema in bijlage)
	\\ \hline
\end{tabularx}




\begin{tabularx}{1.56\textwidth}{|p{1.5cm}|p{6cm}|X|p{4cm}|}
	\hline
	\textbf{Nr. lesdoel } & \textbf{Inhoud (timing)}  & \textbf{Organisatie } & \textbf{Media } \\ \hline
	&\underline{Inhoudelijke titel (timing)}
	\textcolor{gray}{(Naast een inhoudelijke titel en de timing, noteer je kort en samenvattend de kerninhoud van de lesfase; uitgebreide informatie/oefeningen/… neem je op in de uitgewerkte media [verwijzen!])}
	&  \textcolor{gray}{(Naast de benaming van de specifieke werkvorm [bv. placemat-oefening/basis-expertengroep/… en dus níet groepswerk], noteer je kernachtig het organisatorisch verloop van de lesfase. Noteer eveneens belangrijke vragen die je wil stellen.) }
	& 
	\\ \hline
\end{tabularx}






\begin{tabularx}{1.56\textwidth}{|p{1.5cm}|p{6cm}|X|p{4cm}|}
	\hline
	\textbf{Nr. lesdoel } & \textbf{Inhoud (timing)}  & \textbf{Organisatie } & \textbf{Media } \\ \hline
		&\underline{Pauze}\newline
	
	
	&    De studenten krijgen 15 minuten pauze en mogen het lokaal verlaten. Op deze manier kunnen ze het laatste uur weer met volle aandacht werken. Dit is nodig aangezien ik wel grotere problemen bij oefeningen 54 en 55 voorzie.\newline
	Wanneer de eerste studenten het lokaal terug binnen sijpelen, sla ik een praatje met hen, waarbij ik niet over de leerstof begin. 
	& 
	\\ \hline
\end{tabularx}

	
	\begin{tabularx}{1.56\textwidth}{|p{1.5cm}|p{6cm}|X|p{4cm}|}
		\hline
		\textbf{Nr. lesdoel } & \textbf{Inhoud (timing)}  & \textbf{Organisatie } & \textbf{Media } \\ \hline
		&\underline{Inhoudelijke titel (timing)}
	    \textcolor{gray}{(Naast een inhoudelijke titel en de timing, noteer je kort en samenvattend de kerninhoud van de lesfase; uitgebreide informatie/oefeningen/… neem je op in de uitgewerkte media [verwijzen!])}
	    &  \textcolor{gray}{(Naast de benaming van de specifieke werkvorm [bv. placemat-oefening/basis-expertengroep/… en dus níet groepswerk], noteer je kernachtig het organisatorisch verloop van de lesfase. Noteer eveneens belangrijke vragen die je wil stellen.) }
		& 
		\\ \hline
	\end{tabularx}
	
	
	
	
	
	
	
	
\end{landscape}