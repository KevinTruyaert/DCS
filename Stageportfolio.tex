\documentclass[a4paper,12pt,twoside]{article}%twoside
\usepackage[utf8]{inputenc}
\usepackage[dutch]{babel}
\usepackage{fancyhdr, amsmath, color, graphicx, enumitem, tabularx, hyperref, longtable, multirow, placeins, apacite, subcaption,marvosym,multicol}
\usepackage[framemethod=tikz]{mdframed}
 
  \usepackage[margin=2.5cm,headheight=68pt]{geometry}
 %\usepackage[total={16cm, 22cm}]{geometry}
 
\pagestyle{fancy}
\fancyhf{}
\fancyhead[LE,RO]{Specifieke Lerarenopleiding voor CVO-studenten}%'E': even page, 'O': odd page
\fancyhead[RE,LO]{Didactische Competentie Stage}
\fancyfoot[RE,CO]{}
\fancyfoot[LE,RO]{KU Leuven campus Kortrijk Kulak   \thepage}
%\renewcommand{\labelitemi}{$\circ$}
 
\definecolor{CVO}{RGB}{232, 0, 97}
\setlength\parindent{0pt}
\title{Stageportfolio}
\author{Kevin Truyaert}
\date{}
 
 
 % %FOOTER IN MDFRAMED
 \usepackage{footnote} 
 \newenvironment{mdframedwithfoot}
 {   
     \savenotes
     \begin{mdframed}
     \stepcounter{footnote}
     \renewcommand{\thefootnote}{\arabic{footnote}}
     }
 {
     \end{mdframed}
     \spewnotes
 }
 
 
 %FOOTER IN PARBOX
 \makeatletter
 \newcommand{\global@insert}[2]% #1=box number, #2=vertical list
 {\bgroup
   \setbox\@tempboxa=\box#1
   \global\setbox#1=\vbox{\unvbox\@tempboxa #2}
 \egroup}
 
 \long\def\@footnotetext#1{\global@insert\footins{%
  \reset@font\footnotesize
  \interlinepenalty\interfootnotelinepenalty
  \splittopskip\footnotesep
  \splitmaxdepth \dp\strutbox \floatingpenalty \@MM
  \hsize\columnwidth \@parboxrestore
  \protected@edef\@currentlabel{%
  \csname p@footnote\endcsname\@thefnmark
  }%
  \color@begingroup
  \@makefntext{%
  \rule\z@\footnotesep\ignorespaces#1\@finalstrut\strutbox}%
  \color@endgroup}}%
 \makeatother
 %%%%%%%%%%%%%%%%%%%%%%%%%
 
 %Strikeout and highlight text
  \usepackage{soul}
  \usepackage{tikz} % only to get \foreach
  
  %\definecolor{yellow}{RGB}{255,255,0}
  \sethlcolor{yellow}

  \newcommand*{\YellowHighlight}[1]{{\hl{~#1~}}}
  % % % % % %
  
  \usepackage{tabularx,pdflscape,pdfpages}
  
  \newcolumntype{C}[1]{>{\centering\let\newline\\\arraybackslash\hspace{0pt}}m{#1}}
 
 \begin{document}
\maketitle


\section*{Identificatiegegevens}
\begin{center}
	\begin{tabular}{ll}
	\hline
	Naam: & Kevin Truyaert\\ \hline
	Adres: & Bolle-Akkerweg 4\\
		& 8800 Roeselare\\\hline
	Telefoon: & 0032495/928460\\\hline
	Mail: & kevin.truyaert@student.kuleuven.be\\\hline
	Naam stagebegeleider: & Cato De Baets\\ \hline
\end{tabular}
\end{center}

\newpage
\tableofcontents
\newpage


% !TeX root = Stageportfolio.tex

\begin{landscape}
\section{Observatie- en stageplanning}

\subsection{Observatieplanning}

\begin{tabularx}{1.56\textwidth}{|C{0.15\textwidth}|C{0.14\textwidth}|C{0.14\textwidth}|C{0.1\textwidth}|C{0.1\textwidth}|C{0.05\textwidth}|C{0.35\textwidth}|X|}
	
	
\end{tabularx}
	

\subsection{Actieve stage}
\begin{table}[]
	\begin{tabularx}{1.56\textwidth}{|X|}
		\hline
		Naam stagair:  Kevin Truyaert  \\
		Tel.: 0495/928460 \hspace{3cm} e-mail: kevin.truyaert@student.kuleuven.be  \\
		Naam en adres opleidingsinstituut:  KU Leuven Campus Kulak Kortrijk, Etienne-Sabbelaan 53, 8800 Kortrijk  \\
		Naam directie: \\
		Naam stagecoördinator:  David Dudal \\
		\hline
	\end{tabularx}
\end{table}
		
	\begin{tabularx}{1.56\textwidth}{|C{0.15\textwidth}|C{0.14\textwidth}|C{0.14\textwidth}|C{0.1\textwidth}|C{0.1\textwidth}|C{0.05\textwidth}|C{0.35\textwidth}|X|}
		\hline
		\textbf{Datum} & \textbf{Vestiging} & \textbf{\begin{tabular}[C]{@{}l@{}}Aantal\\ stage-uren\end{tabular}} & \textbf{\begin{tabular}[C]{@{}l@{}}Uur \end{tabular}}    & \textbf{Lokaal}& \textbf{\begin{tabular}[C]{@{}l@{}}AV\\TV\\PV\\KV\end{tabular}}& \textbf{\begin{tabular}[C]{@{}l@{}}Onderwijsvorm\\ graad en lj\\ Vak en lesonderwerp\end{tabular}}  &  \textbf{\begin{tabular}[C]{@{}l@{}}Naam vakmentor\\ + handtekening\end{tabular} } \\ \hline
		27/11/2019 & Kulak & 1-3 & 10:30-13:00 & A352 & AV & Universiteit\newline 2e jaar Handelsingenieur\newline Conceptuele natuurkunde\newline werkzitting elektromagnetisme & \\ \hline
		4/12/2019 & Kulak & 4-6 & 10:30-13:00 & A352 & AV & Universiteit\newline 2e jaar Handelsingenieur\newline Conceptuele natuurkunde\newline werkzitting elektromagnetisme & \\ \hline
		11/12/2019 & Kulak & 7-9 & 10:30-13:00 & A352 & AV & Universiteit\newline 2e jaar Handelsingenieur\newline Conceptuele natuurkunde\newline werkzitting elektromagnetisme & \\ \hline
	\end{tabularx}
\begin{tabularx}{1.56\textwidth}{|C{0.15\textwidth}|C{0.14\textwidth}|C{0.14\textwidth}|C{0.1\textwidth}|C{0.1\textwidth}|C{0.05\textwidth}|C{0.35\textwidth}|X|}
	\hline
		18/12/2019 & Kulak & 10-12 & 10:30-13:00 & A352 & AV & Universiteit\newline 2e jaar Handelsingenieur\newline Conceptuele natuurkunde\newline werkzitting elektromagnetisme & \\ \hline
		 &  &  &  &  &  &  & \\ \hline
	\end{tabularx}
		


		
\end{landscape}		
		


\section{Persoonlijk ontwikkelingsplan}

\begin{tabularx}{\textwidth}{|p{0.15\textwidth}|p{0.795\textwidth}|}
	\hline
	\textbf{Lesdoel 1} & 
	\underline{FG 1: de leraar als begeleider van leer- en}\newline \underline{ontwikkelingsprocessen}\newline
	
	1.8 De leraar kan observatie en evaluatie voorbereiden en uitvoeren met het oog op bijsturing en remediëring als onderdeel van het leerproces van een lerende(n) en kan die observatie-en evaluatiegegevens gebruiken om zijn eigen didactische handelen in vraag te stellen en bij te sturen waar nodig.\\ \hline
	Actie 1 & Tijdens het lesgeven wil ik veel in interactie treden. Dit zou ik met zoveel mogelijk leerlingen willen doen en niet steeds dezelfde leerlingen aan bod laten komen. Door hen gerichte vragen te stellen, kan ik kijken waar er mogelijke problemen zijn met de leerstof en van daaruit werken om zoveel mogelijk begrijpelijk te maken voor alle leerlingen. \\ \hline
	Actie 2 & Na het verbeteren van een toets, wil ik die met de leerlingen overlopen door de meest voorkomende fouten te bespreken. Zo kan ik hen bijsturen en kan ik de belangrijkste punten aanhalen waar er problemen waren. Tegelijkertijd kom ik zo te weten waar ik te weinig nadruk gelegd heb tijdens de les. Hier kan ik nu mee aan de slag om mijn toekomstige lessen aan te passen en om te verhinderen dat hetzelfde soort fouten bij soortgelijke zaken minder gemaakt worden. \\ \hline
\end{tabularx}

\vspace{0.5cm}
\begin{tabularx}{\textwidth}{|p{0.15\textwidth}|p{0.795\textwidth}|}
\hline
\textbf{Lesdoel 2} & \underline{FG5:  de leraar als innovator - de leraar als onderzoeker}\newline
5.1 De leraar kan de kwaliteit van zijn onderwijs verder ontwikkelen. De leraar kan zijn eigen onderwijspraktijk en zijn eigen functioneren in vraag stellen en bijsturen (verbeteren) door te innoveren om zijn eigen praktijk te verbeteren.\\ \hline
Actie 1 & Ik verzorg reeds drie jaar oefenzittingen aan de universiteit. Dit jaar wil ik iets nieuws proberen en de studenten actiever de oefeningen laten maken. Ik wil hen in groep aan de oefeningen laten werken, waardoor ze met elkaar in interactie kunnen treden om de oefeningen samen tot een goed eind te kunnen brengen. Op die manier wil ik tijdens mijn oefenzittingen voor innovatie bij lessen in het hoger onderwijs zorgen. \\ \hline
Actie 2 & Bij de lessen die ik in het middelbaar zal verzorgen, wil ik terugkoppelen naar mijn stagelessen die ik bij DCO deed. Hier gaf ik telkens de introductieles van een nieuw stuk theorie. Die gaf ik relatief `klassiek', waarbij ik als leerkracht veel aan bod kwam. Ik wil nu proberen om de leerlingen zal actiever aan de slag te zetten bij de start van een nieuw stuk. Ik zie dit nu ook meer zitten, omdat ik meer dan één les(blok) per klas zal brengen. Dit zal als gevolg hebben dat ik een groter plan kan uitwerken en zo proberen om mijn eigen lesgeven te innoveren.    \\ \hline
\end{tabularx}

\vspace{0.5cm}
\begin{tabularx}{\textwidth}{|p{0.15\textwidth}|p{0.795\textwidth}|}
\hline
\textbf{Lesdoel 3} & \\ \hline
Actie 1 & \\ \hline
Actie 2 & \\ \hline
\end{tabularx}

\begin{landscape}
	
\section{Bespreking lesobservaties}

\begin{tabularx}{1.56\textwidth}{|C{0.25\textwidth}|C{0.1\textwidth}|C{0.25\textwidth}|X|}\hline
	\textbf{Naam student: Kevin Truyaert} & & Aandachtspunten (o.b.v. POP) & Reflectie:\newline -Wat leerde ik uit mijn observatie over mijn aandachtspunten? \newline -Wat doe ik ermee tijdens mijn stage?\\\hline
	Observatieles \newline Datum: & 1 & & \\
	Klas: \newline Lesonderwerp: & 2 & & \\\hline
	Observatieles \newline Datum: & 1 & & \\
	Klas: \newline Lesonderwerp: & 2 & & \\\hline
	
\end{tabularx}
\end{landscape}

% !TeX root = Stageportfolio.tex



\begin{landscape}
	\section{Lesvoorbereidingen en bijhorende media}
	\begin{tabularx}{1.56\textwidth}{|p{0.35\textwidth}|X|}\hline
		\textbf{Administratieve gegevens}\newline\newline
		Kevin Truyaert\newline\newline
		Universiteit\newline
		Handelsingenieur, 2de fase\newline
		Onderwijskoepel\newline
		Leerplannummer\newline
		Lesonderwerp & \textbf{Doelstellingen}\newline\newline
		\underline{Leerplandoelen}\newline\newline
		\underline{Lesdoelen}\newline\newline \\\hline
		\multirow{2}{0.35\textwidth}{\textbf{Beginsituatie}} & \textbf{Acties} \\ \cline{2-2}
		 & \textbf{Bronnen}\\\hline
		
	\end{tabularx}
	
	
	
\begin{table}[h]
	\begin{tabularx}{1.56\textwidth}{|p{1.5cm}|p{6cm}|X|p{4cm}|}
		\hline
		\textbf{Nr. lesdoel } & \textbf{Inhoud (timing)}  & \textbf{Organisatie } & \textbf{Media } \\ \hline
		&\underline{Inhoudelijke titel (timing)}
	    \textcolor{gray}{(Naast een inhoudelijke titel en de timing, noteer je kort en samenvattend de kerninhoud van de lesfase; uitgebreide informatie/oefeningen/… neem je op in de uitgewerkte media [verwijzen!])}
	    &  \textcolor{gray}{(Naast de benaming van de specifieke werkvorm [bv. placemat-oefening/basis-expertengroep/… en dus níet groepswerk], noteer je kernachtig het organisatorisch verloop van de lesfase. Noteer eveneens belangrijke vragen die je wil stellen.) }
		& 
		\\ \hline
	\end{tabularx}
\end{table}		
	
	
	
	
	
	
	
	
\end{landscape}

\section{Bespreking meso-activiteiten}
Stel per meso-activiteit een verslag op op basis van volgende criteria:
\begin{itemize}
	\item Korte situering van de drie activiteiten.
 	\item Omschrijving van twee aspecten die je voor jezelf geleerd hebt uit de deelname aan de activiteiten
	\item  Toon aan met twee voorbeelden dat de activiteiten een meerwaarde zijn voor de leerkrachten.
	\item Toon aan met twee voorbeelden dat de activiteiten een meerwaarde vormen voor de leerlingen.
	\item Bespreek hoe het komt dat bepaalde activiteiten geen echte meerwaarde hebben voor leerlingen en op welke manier deze aangepast kunnen worden om toch nog functioneel te zijn voor het leerproces van de leerlingen.
\end{itemize}


\section{Evaluatiedocumenten vakmentor}

\section{Evaluatie document klasbezoek stagebegeleider}

\section{Eindreflectie}
Stel een eindreflectie op waarin je volgende aspecten behandelt: 

1) Waren er factoren die bevorderend of belemmerend werkten m.b.t. het goed doorlopen van je stage? 
2) Waarvoor had je graag bijkomende begeleiding gekregen van je vakmentoren? 
3) Waarvoor had je graag bijkomende begeleiding gekregen van je stagebegeleider? 
4) Bekijk aandachtig de acties die je in het begin van je stage opstelde in jouw POP . Ga na of je via de acties jouw leerdoelen hebt behaald. Verwijs heel duidelijk naar informatie in je portfolio waar en hoe je deze acties aan bod liet komen. 
5)  Bestudeer nogmaals het opleidingsprofiel en de basiscompetenties van een leraar (link):  bespreek minstens 5 basiscompetenties die je succesvol hebt behaald tijdens het uitvoeren van je stage. 
Jouw eindreflectie is maximaal drie A4-pagina’s lang. 



\section{Voorbereiding eindassessment}

Om het eindassessment voor te bereiden, kan je gebruik maken van volgende vragen:
• Lees jouw eindreflectie goed na en bekijk jouw leerdoelen en uitgewerkte acties. Recapituleer hoe je de stage hebt ervaren. Waarom moet een directeur jou als leerkracht aanwerven? Wat heb jij een schoolteam te bieden? Waar zie je nog uitdagingen voor jezelf? 
• Waar heb je nog aanvangsbegeleiding nodig en wie kan jou daarbij helpen (toon je inzicht in vakgroep- en schoolwerking aan)?
• Hoe heb je de lerarenopleiding in het algemeen ervaren? Wat vond je positief? Wat heb je gemist tijdens de opleiding? 

















 \end{document}
