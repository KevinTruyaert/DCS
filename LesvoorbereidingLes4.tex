% !TeX root = Stageportfolio.tex



\begin{landscape}	
	\subsubsection{Les 9-10}
	\begin{tabularx}{1.56\textwidth}{|p{0.55\textwidth}|X|}\hline
		\textbf{Administratieve gegevens}\newline\newline
		Kevin Truyaert\newline\newline
		Universiteit\newline
		Handelsingenieur, 2de fase\newline
		\underline{ECTS-fiche}: De inhoud is terug te vinden op de ECTS fiche: \href{https://onderwijsaanbod.kuleuven.be/syllabi/n/D0W55AN.htm}{https://onderwijsaanbod.kuleuven.be/syllabi/n /D0W55AN.htm} \newline
		\underline{Lesonderwerp}: `DC netwerken met weerstanden en condensatoren' & \textbf{Doelstellingen}\newline\vspace{0.5cm}
		\underline{Punt op de ECTS-fiche}
		\vspace{-0.5cm}\newline  - DC netwerken, wetten van Kirchhoff, elektrische meettoestellen \newline - toepassing: elektrische veiligheid en elektrische huisinstallatie \newline
		\underline{Lesdoelen}\newline
		\vspace{-0.5cm}
		\begin{enumerate}[itemsep=0.08\baselineskip]
			\item De studenten kunnen de wetten van Kirchhoff wiskundig formuleren.
			\item De studenten kunnen de werking en de invloed van een condensator in een elektrische schakeling conceptueel uitleggen.
			\item De studenten kunnen de wetten van Kirchhoff opstellen voor een gesloten netwerk met bronnen en condensatoren.
			\item De studenten kunnen de equivalente capaciteit van condensatoren in serie en parallel berekenen. (Herhaling lesdoel 2, les 4-5)
			\item De studenten kunnen de wetten van Kirchhoff opstellen voor een gesloten netwerk met bronnen, weerstanden en condensatoren.
			\item De studenten kunnen in duo/trio over de oefening discussiëren en samen oplossingsgericht werken.
		%	\item De studenten kunnen de wetten van Kirchhoff individueel gebruiken om een elektrische schakeling uit te werken.
		\end{enumerate} \\\hline
		\multicolumn{2}{c}{ }\\
		\multicolumn{2}{c}{ }\\
		\multicolumn{2}{c}{ }\\
		\multicolumn{2}{c}{ }\\
		\multicolumn{2}{c}{ }\\
		\multicolumn{2}{c}{ }\\
		\multicolumn{2}{c}{ }\\
		\multicolumn{2}{c}{ }\\
		\multicolumn{2}{c}{ }\\
	\end{tabularx}
	
	
	\begin{tabularx}{1.56\textwidth}{|p{0.55\textwidth}|X|}
		\hline
		\multirow{2}{0.55\textwidth}{\textbf{Beginsituatie}\newline De studenten hebben vorige week een oefenzitting rond de wetten van Kirchhoff gehad. Tijdens deze les hebben ze de wiskundige uitdrukking herhaald en hebben ze de wetten van Kirchhoff toegepast in schakelingen met bronnen en weerstanden. Rond deze tijd hebben de studenten echter meerdere deadlines voor andere vakken en een examen Frans. Hierdoor plaats ik geen voorbereidende oefening online, maar vraag ik hen om enkel eigenschappen van condensatoren nog eens goed te bekijken. \newline\newline Er zijn 28 studenten die deze sessie volgen, maar vorige sessie waren 22 studenten aanwezig. \newline\newline Het lokaal kan 30 studenten plaatsen. Ik laat de banken in drie rijen van tien staan. Er is een dubbel krijtbord ter beschikking en de mogelijkheid tot projectie. Wanneer er geprojecteerd wordt, hangt het projectiescherm grotendeels over beide borden.  }& \textbf{Acties}\newline  -  Als examenvraag stel ik een oefening op rond de wetten van Kirchhoff, die aansluit bij wat ze deze en vorige les gezien hebben. Ik vind het van essentieel belang dat ze de wetten van Kirchhoff niet allen goed en veel kunnen oefenen, maar dat ze die ook conceptueel begrijpen. Vorige lessenreeks zette ik er op in dat zoveel mogelijk studenten de basis van het toepassen van de wetten van Kirchhoff onder de knie hadden. Bij deze les wil ik er voor zorgen dat de studenten individueler ook aan de slag kunnen gaan bij deze soort oefeningen. Ze kunnen wel nog steeds ten rade bij hun buren of bij mij wanneer er problemen opduiken. \newline\newline
		
		- Bij het begin van de les overloop ik samen met de studenten de wetten van Kirchhoff. Zij reiken mij de twee wetten aan, die ik op het bord neerschrijf. Verder noteer ik ook samen met hen een stappenplan om dit soort oefeningen op te lossen. Dit laat ik op het bord staan. Zo kunnen de studenten steeds makkelijk teruggrijpen naar de theorie. \newline\newline
		- Ik werk niet met projectie, maar noteer alles op het bord, omdat het projectiescherm voor zo goed als beide borden hangt. Hierdoor houd ik een tempo aan waarop de studenten makkelijker kunnen volgen, doordat ik alles zelf ook neerschrijf.  
		
		\\ \cline{2-2}
		& \textbf{Bronnen}\begin{itemize}
			\item Dudal, D., Temmerman, E., Truyaert, K., Heymans, S. (2019). Slides conceptuele natuurkunde
			\item Dudal, D., Temmerman, E., Truyaert, K., Heymans, S. (2019). Oefeningenbundel conceptuele natuurkunde
			\item Giancoli, D. C. (2008). Physics for scientists and engineers. Pearson Education International.
		\end{itemize}\\ \hline
	\end{tabularx}
\newpage
	

\begin{tabularx}{1.56\textwidth}{|p{1.5cm}|p{6cm}|X|p{3cm}|}
	\hline
	\textbf{Nr. lesdoel } & \textbf{Inhoud (timing)}  & \textbf{Organisatie } & \textbf{Media } \\ \hline
	1\newline 2 &\underline{Herhaling theorie (15 minuten)}\newline
	Ik vraag net zoals vorige les opnieuw aan de studenten om de wetten van Kirchhoff zowel in hun eigen woorden als wiskundig te formuleren. Verder bespreken we nog de eigenschappen van een condensator en wat dit betekent wanneer ze in een schakeling geplaatst worden. Ik leg er de nadruk op dat de studenten in oefeningen enkel maar schakelingen met condensatoren in stationaire toestand moeten kunnen oplossen. Dit wil zeggen dat de condensator ofwel volledig opgeladen ofwel volledig ontladen is.
	&  \underline{Onderwijsleergesprek}\newline 
	Ik start deze les met aan de studenten te vragen om mij de twee wetten van Kirchhoff nog eens opnieuw uit te leggen, conceptueel en de wiskundige vertaling ervan. Ik probeer verschillende studenten aan het woord te laten. Daarna laat ik de studenten nog eens het stappenplan herhalen en schrijf ik dit ook op het bord ter referentie voor de oefeningen. \newline Hierna overloop ik met de studenten de eigenschappen van een condensator in stationaire toestand: geen stroom meer door een tak met een condensator, betekenis van het potentiaalverschil bij een opgeladen condensator, het ontladen van een condensator (conceptueel). Hierna schets ik een kleine kring (bron-condensator) op het bord waar we dit klassikaal op toepassen.\newline 
	Hierna noteer ik de oefeningen op bord die gemaakt kunnen worden. Dit zijn oefeningen 68, 70 en 67 in die volgorde. Ik verwacht dat de eerste drie oefeningen door iedereen gemaakt kunnen worden en de laatste door de betere studenten.\newline Ik zal de nadruk tijdens deze les vooral leggen op het zelfstandig inoefenen van dit soort oefeningen. 
	& Krijtbord 
	\\ \hline
\end{tabularx}

\begin{tabularx}{1.56\textwidth}{|p{1.5cm}|p{6cm}|X|p{4cm}|}
	\hline
	\textbf{Nr. lesdoel } & \textbf{Inhoud (timing)}  & \textbf{Organisatie } & \textbf{Media } \\ \hline
	3 \newline 4\newline 6	&\underline{Oefening 68 (20 minuten)}\newline
	Tijdens deze oefening ervaren studenten om spanningsverschillen over condensatoren te berekenen. Hiervoor zullen ze eerst een equivalente capaciteit voor de condensatoren moeten berekenen. Op dit moment evalueer ik of lesdoel 2 van les 4-5 wel degelijk bereikt is (hier lesdoel 4). Ik verwacht geen problemen met de evaluatie van vorig lesdoel. 
	
	&   \underline{Oplossingensleutel}\newline
		De studenten kunnen hun antwoord controleren aan de hand van een controlesleutel. Deze zijn zo opgesteld dat alle stappen benoemd zijn, maar niet uitgewerkt. Bij vragen kunnen de studenten mij raadplegen.\newline
		De grootste problemen zullen ontstaan bij het berekenen van het potentiaalverschil over iedere condensator. De studenten zullen niet meteen inzien dat de lading die per condensator opgeslagen is dezelfde moet zijn. Dit zal ik beantwoorden door hen te vragen naar de werking van een opladende condensator in een circuit.
	&  Oplossingensleutel (Bijlage)\newline Oefeningenbundel + cursuspapier
	\\ \hline
\end{tabularx}




\begin{tabularx}{1.56\textwidth}{|p{1.5cm}|p{6cm}|X|p{4cm}|}
	\hline
	\textbf{Nr. lesdoel } & \textbf{Inhoud (timing)}  & \textbf{Organisatie } & \textbf{Media } \\ \hline
	5\newline 6	&\underline{Oefening 70 (30 minuten)}\newline
	Deze schakeling bevat zowel weerstanden als condensatoren als verbruiker. Deze oefening is de eerste gecombineerde oefening die de studenten krijgen.  
	
	&   \underline{Oplossingensleutel}\newline
	De studenten kunnen hun antwoord controleren aan de hand van een controlesleutel. Deze zijn zo opgesteld dat alle stappen benoemd zijn, maar niet uitgewerkt. Bij vragen kunnen de studenten mij raadplegen.\newline
	De grootste problemen bij deze oefening zullen ontstaan omdat er geen rekening gehouden wordt met een bepaalde eigenschap van de condensator. Wanneer deze volledig opgeladen is, dan is de stroom doorheen die tak gelijk aan $0$~A. Sommige studenten zullen dit niet meteen inzien, waardoor er een vergelijking tekort is. Bij deze vraag stel ik aan de studenten de vraag om eens na te denken over de eigenschappen van condensatoren. Een andere vraag kan zijn waarom de weerstand van R$_3$ niet gekend is. Ook voor die vraag geldt hetzelfde antwoord.
	&  Oplossingensleutel (Bijlage)\newline Oefeningenbundel + cursuspapier
	\\ \hline
\end{tabularx}




\begin{tabularx}{1.56\textwidth}{|p{1.5cm}|p{6cm}|X|p{4cm}|}
	\hline
	\textbf{Nr. lesdoel } & \textbf{Inhoud (timing)}  & \textbf{Organisatie } & \textbf{Media } \\ \hline
	&\underline{Pauze}\newline
	
	
	&    De studenten krijgen 15 minuten pauze en mogen het lokaal verlaten. \newline
	Wanneer de eerste studenten het lokaal terug binnen sijpelen, sla ik een praatje met hen, waarbij ik niet over de leerstof begin. 
	& 
	\\ \hline
\end{tabularx}



\begin{tabularx}{1.56\textwidth}{|p{1.5cm}|p{6cm}|X|p{4cm}|}
	\hline
	\textbf{Nr. lesdoel } & \textbf{Inhoud (timing)}  & \textbf{Organisatie } & \textbf{Media } \\ \hline
	 5\newline 6 & \underline{Oefening 67 (30 minuten)}\newline Deze schakeling bevat zowel weerstanden als condensatoren als verbruiker. Deze schakeling is complexer dan de vorige.
	 
	 
	 &   \underline{Oplossingensleutel}\newline
	 De studenten kunnen hun antwoord controleren aan de hand van een controlesleutel. Deze zijn zo opgesteld dat alle stappen benoemd zijn, maar niet uitgewerkt. Bij vragen kunnen de studenten mij raadplegen.\newline
	 Opnieuw hier geldt dat wanneer de condensator volledig opgeladen is, de stroom doorheen die tak gelijk aan $0$~A is. Sommige studenten zullen dit opnieuw niet meteen inzien, waardoor er opnieuw een vergelijking tekort is. Bij deze vraag stel ik dan ook opnieuw dezelfde vraag aan de studenten.  Mogelijks ontstaan er ook problemen bij het sluiten van de schakelaar waardoor er een extra lus ontstaat.
	 &  Oplossingensleutel (Bijlage)\newline Oefeningenbundel + cursuspapier
	 \\ \hline
\end{tabularx}



\begin{tabularx}{1.56\textwidth}{|p{1.5cm}|p{6cm}|X|p{4cm}|}
	\hline
	\textbf{Nr. lesdoel } & \textbf{Inhoud (timing)}  & \textbf{Organisatie } & \textbf{Media } \\ \hline
	5\newline 6	&\underline{Vragen + extra oefening (1 uur)}\newline 
	Ik projecteer eerst een extra oefening (oude examenvraag) die over een DC netwerk gaat. Hierna kunnen de studenten mij individueel tijdens deze lesfase vragen stellen. Ondertussen kunnen de overige studenten de oude examenoefening proberen op te lossen.
	
	&  Ik projecteer een oude examen oefening met de numerieke oplossing op het projectiescherm. De studenten kunnen deze oplossen. Ondertussen is er mogelijkheid tot vragen in verband met alle delen van de cursus. Ik vraag de studenten om mij geen vragen te stellen over de examenoefening gedurende de eerste 40 minuten, tenzij er geen vragen over de cursus meer zouden komen. Ik meld hen ook dat ik enkel meer mondelinge feedback zal geven over deze oefening. 
	& Projectiescherm \newline cursuspapier 
	\\ \hline
\end{tabularx}


	
\begin{tabularx}{1.56\textwidth}{|p{1.5cm}|p{6cm}|X|p{4cm}|}
	\hline
	\textbf{Nr. lesdoel } & \textbf{Inhoud (timing)}  & \textbf{Organisatie } & \textbf{Media } \\ \hline
	&\underline{Afsluiten (5 minuten)}\newline 
	&  \underline{Afsluiten (5 minuten)}\newline
	Ik herhaal nog even kort wat er van de studenten verwacht werd tijdens deze les en wat ze bijgeleerd hebben. Ik herhaal nog eens de info in verband met de oefeningen voor het examen, het deel waarvoor ik verantwoordelijk ben. Ik herhaal hen ook nog eens de afspraken rond het stellen van vragen voor het examen en dat er tijdens de kerstperiode zowel niet door de prof als door mij geantwoord zal worden.  Ik wens de studenten een fijn oudjaar en veel succes bij het studeren.
	& 
	\\ \hline
\end{tabularx}
	
	
	
	
	
	
	
\end{landscape}


\subsection*{Bijlage 4.1: bordschema theorie}
\subsection*{Bijlage 4.2: opgeloste oefeningen}


\subsection*{Bijlage 4.3: oplossingssleutels}
De oplossingssleutels zijn hieronder bijgevoegd. Deze zijn telkens zo opgesteld dat de studenten niet de oplossing uitgewerkt krijgen, maar dat er enkele gerichte vragen of hints zijn die hen op de weg kunnen helpen wanneer ze vast zitten. Wanneer ze de oefening correct hebben, kunnen ze stilstaan bij die vragen en controleren of ze de oefening ook begrijpen.\newline
De oplossingssleutels zijn hier samen gevoegd. In realiteit werden meerdere sleutels per blad afgedrukt en daarna uitgeknipt.

\subsection*{68}
\underline{Methode}
\begin{enumerate}
\item Hoe bereken je de equivalente capaciteit opnieuw? (Oefening 59) 
\item Wat is er indentiek bij  volledig opgeladen condensatoren in serie?
\item Je weet nu de lading en de capaciteit. Hoe haal je hieruit het potentiaalverschil?
\end{enumerate}

\subsection*{70}
\underline{Methode}
\begin{enumerate}
\item Waarom is de numerieke waarde van R$_3$ niet gekend? Dit is bewust.
\item Wat gebeurt er met een tak die een volledig opgeladen condensator bevat?
\end{enumerate}

\underline{Extra vragen}
\begin{enumerate}
\item Heeft de tak met de condensator nog een invloed op de schakeling eens die volledig opgeladen is? Verklaar.
\end{enumerate}




\subsection*{67}
\underline{Methode}
\begin{enumerate}
	\item Wat gebeurt er met een tak die een volledig opgeladen condensator bevat?
	\item Gebruik opnieuw de hoogte als equivalent voor spanning. Welke verbruikers leveren hier een spanningsverschil? Welke niet? Gebruik dit om conceptueel na te gaan welk punt, b of c, een hogere potentiaal heeft.
	\item Wanneer de schakelaar gesloten is, hebben de condensatoren een \textbf{nieuw} evenwicht gevonden! Er zal dus een ladingsverschil zijn met de `open' situatie.
\end{enumerate}


\underline{Extra vragen}
\begin{enumerate}
\item Is dit ladingsverschil steeds gelijk aan elkaar? Waarom is dit hier zo?
\item Wanneer de schakelaar gesloten is, wat is nu het potentiaalverschil tussen b en c? Verklaar.
\item Beschrijf conceptueel wat er in de eerste tijdseenheden na het sluiten van de schakelaar  gebeurt, na lange tijd open te zijn geweest.
\end{enumerate}



\subsection*{Bijlage 4.4: examenoefening 2018-2019}
	
Onderstaande figuur toont een Kirchhoff netwerk bestaande uit vier weerstanden, een spanningsbron, een condensator en een schakelaar. De waardes van de verschillende componenten zijn: $\varepsilon_1=15$~V, $C=4$~$\mu F$ en $R=5~\Omega$. De waarde van de weerstand $\tilde{R}$ is ongekend. 

In de situatie waar de schakelaar al lange tijd \textbf{gesloten} is, weet je dat de stroom in de tak met $\varepsilon_1$ gelijk is aan $I_1 = 1A$. De stroom $I_1$ loopt zoals op de figuur aangegeven.

\begin{enumerate}
	\item Waar ligt de hoogste potentiaal bij de condensator, in punt a of b?
	\item Bereken de lading op de condensator wanneer de schakelaar lange tijd gesloten is.
	\item Bereken het vermogen dat over $\tilde{R}$ verloren gaat wanneer de schakelaar gesloten is.
	\item De condensator is volledig opgeladen op het moment dat de schakelaar \textbf{open} gezet wordt. De condensator zal beginnen te ontladen. In de theorie werd gezien dat de lading op de condensator in een RC-keten exponentieel daalt als: \[Q=Q_0e^{-\frac{t}{R_{eq}C}},\] waarbij $R_{eq}$ de equivalente weerstand van het systeem is en $Q_0$ is de beginlading op de condensator. Bepaal de stroom in de tak van de condensator in het resterende systeem in functie van de tijd.
\end{enumerate}
Formuleer telkens  een duidelijke antwoordzin. 

\def\x{6}
\def\y{6}
% Size of the bridge
\def\dx{3}
\def\dy{3}

\begin{figure}[h]
	\centering
	\begin{circuitikz}[american resistors, european voltages]
		%	% Voltage source
		\draw (0,\y) to [battery1, l_=$\varepsilon_1$,i=$I_1$,*-*]
		(0, 0) to [cspst] (\dx,0) to (\x,0) to (\x,1) to ({\x/6*4},1) to [R,l_=$3R$,*-*] ({\x/6*4},{\y/2}); \draw  ({\x/6*4},{\y/2}) to [R=$R$,*-*] ({\x/6*4},5) to (\x,5) to (\x,\y) to (0,\y);
		\draw (\x,5) to ({\x/6*8},5) to [R, l_=$\tilde{R}$,*-*] ({\x/3*4}, {\y/2});
		\draw ({\x/3*4}, {\y/2}) to [R,l_=$2R$,*-*] ({\x/3*4}, 1) to (\x,1);
		\draw ({\x/6*4},{\y/2}) node[left]{a} to [C,l_=$C$,*-] ({\x/3*4},{\y/2}) node[right]{b};
	\end{circuitikz}
	\caption*{Een Kirchoff netwerk met vier weerstanden, een bron, een condensator en een schakelaar.}
	\label{Fig:KichoffNetwerk1V4R}
\end{figure}

\underline{\textbf{Antwoord:}}
\begin{enumerate}
	\item punt a ligt 8.75~V hoger dan punt b
	\item 35$\mu$C
	\item 3.125W 
	\item $I = -0.5091\exp(-14545t)~A$
\end{enumerate}

