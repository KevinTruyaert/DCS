% !TeX root = Stageportfolio.tex

\section{Persoonlijk ontwikkelingsplan}
\begin{tabularx}{\textwidth}{|p{0.15\textwidth}|p{0.795\textwidth}|}
	\hline
	\textbf{Lesdoel 1} & 
	\underline{FG 1: de leraar als begeleider van leer- en}\newline \underline{ontwikkelingsprocessen}\newline
	
	1.8 De leraar kan observatie en evaluatie voorbereiden en uitvoeren met het oog op bijsturing en remediëring als onderdeel van het leerproces van een lerende(n) en kan die observatie-en evaluatiegegevens gebruiken om zijn eigen didactische handelen in vraag te stellen en bij te sturen waar nodig.\\ \hline
	Actie 1 & Tijdens het lesgeven wil ik veel in interactie treden. Dit zou ik met zoveel mogelijk leerlingen willen doen en niet steeds dezelfde leerlingen aan bod laten komen. Door hen gerichte vragen te stellen, kan ik kijken waar er mogelijke problemen zijn met de leerstof en van daaruit werken om zoveel mogelijk begrijpelijk te maken voor alle leerlingen. \\ \hline
	Actie 2 & Na het verbeteren van een toets, wil ik die met de leerlingen overlopen door de meest voorkomende fouten te bespreken. Zo kan ik hen bijsturen en kan ik de belangrijkste punten aanhalen waar er problemen waren. Tegelijkertijd kom ik zo te weten waar ik te weinig nadruk gelegd heb tijdens de les. Hier kan ik nu mee aan de slag om mijn toekomstige lessen aan te passen en om te verhinderen dat hetzelfde soort fouten bij soortgelijke zaken minder gemaakt worden. \\ \hline
\end{tabularx}


\vspace{0.5cm}
\begin{tabularx}{\textwidth}{|p{0.15\textwidth}|p{0.795\textwidth}|}
	\hline
	\textbf{Lesdoel 2} & \underline{FG3: de leraarals inhoudelijk expert} \newline\newline 3.3 De leraar beheerst de kennis en vaardigheden met betrekking tot de (vak)didactiek van zijn onderwijsopdracht. Hij kan die  actualiseren, verbreden en verdiepen. \\ \hline
	Actie 1 & Ik wil als leraar in staat zijn om de theorie interessant over te kunnen brengen. Dit wil ik doen door actuele zaken als voorbeeld van die theorie te gebruiken. Een voorbeeld hiervan: ieder jaar wordt een flitsmarathon aangekondigd. De flitscamera werkt volgens het Dopplereffect. Dus wanneer ik dat moet uitleggen aan de leerlingen, kan ik de flitscamera als voorbeeld gebruiken. Door actuele thema's en alledaagse voorwerpen te linken met fysische verschijnselen, hoop ik dat de leerlingen de wereld rond hen beter begrijpen. \\ \hline
	Actie 2 & Als leerkracht vind ik het belangrijk dat je de leerstof die je aan het bespreken bent, goed begrijpt en dat je de achtergrond ervan ook kent, ook al behandel je die niet in de les. Ik vind dat je als leerkracht de leerstof enkele niveaus dieper moet beheersen dan dat je ze moet overbrengen. Op die manier kan je beter begrijpen vanwaar alles komt en zou je meerdere invalswegen moeten hebben om de te geven leerstof aan je leerlingen over te brengen. \\ \hline
\end{tabularx}


\vspace{0.5cm}
\begin{tabularx}{\textwidth}{|p{0.15\textwidth}|p{0.795\textwidth}|}
	\hline
	\textbf{Lesdoel 3} & \underline{FG5:  de leraar als innovator - de leraar als onderzoeker}\newline\newline
	5.1 De leraar kan de kwaliteit van zijn onderwijs verder ontwikkelen. De leraar kan zijn eigen onderwijspraktijk en zijn eigen functioneren in vraag stellen en bijsturen (verbeteren) door te innoveren om zijn eigen praktijk te verbeteren.\\ \hline
	Actie 1 & Ik verzorg reeds drie jaar oefenzittingen aan de universiteit. Dit jaar wil ik iets nieuws proberen en de studenten actiever de oefeningen laten maken. Ik wil hen in groep aan de oefeningen laten werken, waardoor ze met elkaar in interactie kunnen treden om de oefeningen samen tot een goed eind te kunnen brengen. Op die manier wil ik tijdens mijn oefenzittingen voor innovatie bij lessen in het hoger onderwijs zorgen. \\ \hline
	Actie 2 & Bij de lessen die ik in het middelbaar zal verzorgen, wil ik terugkoppelen naar mijn stagelessen die ik bij DCO deed. Hier gaf ik telkens de introductieles van een nieuw stuk theorie. Die gaf ik relatief `klassiek', waarbij ik als leerkracht veel aan bod kwam. Ik wil nu proberen om de leerlingen zal actiever aan de slag te zetten bij de start van een nieuw stuk. Ik zie dit nu ook meer zitten, omdat ik meer dan één les(blok) per klas zal brengen. Dit zal als gevolg hebben dat ik een groter plan kan uitwerken en zo proberen om mijn eigen lesgeven te innoveren.    \\ \hline
\end{tabularx}